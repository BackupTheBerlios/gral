\xname{VertexGridRange}
\begin{Label}{VertexGridRange}
\conceptsection{Vertex (Edge, Cell, ...) Grid Range Concept}
\end{Label}

\conceptsubsection{Description}
A {\em Vertex Grid Range\/} is a sequence of objects of type {\tt V}
which is a model of 
\sectionlink{\concept{Grid Vertex}}{GridVertex}.

Virtually the same definitions can be made for the other element types,
replacing {\tt Vertex} with {\tt Edge, Cell} etc.
Therefore, Vertex Grid Range is chosen to stand {\em sui generis\/} for 
Edge Grid  Range, Cell  Grid Range and so on.

The Vertex Grid Range concept is seldom useful as such;
its primary use is to
define further refinements of the basic 
\sectionlink{\concept{Grid Range}}{GridRange} concept: 
Useful concrete grid types generally
will be models of (at least) two different range concepts,
such as Vertex Grid Range and Grid Cell Sequence.

\conceptsubsection{Refinement of}
\sectionlink{\concept{ Grid Range }}{GridRange}

\conceptsubsection{Notation}
{\tt R} is a type which is a model of Vertex Range 
\\
{\tt r} is an object of type {\tt R::grid\_type}
\\
{\tt v} is an object of type  {\tt R::Vertex}
\\
{\tt vi} is an object of type  {\tt R::VertexIterator}

\conceptsubsection{Associated types}
\begin{tabularx}{14cm}{llR} \\ 
  \hline
  \bf  Name  &\bf  Expression  &\bf  Description  \\ \hline
  Vertex & 
  {\tt R::Vertex} &
  type of vertex (model of \sectionlink{\concept{Grid Vertex}}{GridVertex})
  \\ 
  \hline
       
  handle  & 
  {\tt R::vertex\_handle} &
  \sectionlink{\concept{ handle}}{GridElementHandle} type 
  corresponding to {\tt R::Vertex} 
  \par
   model of \sectionlink{\concept{ Grid Vertex Handle }}{GridVertexHandle}
  \par
   ({\tt R::Vertex::handle\_type} is identical to  {\tt R::vertex\_handle}) 
   \\ 
   Vertex iterator  & 
   {\tt R::VertexIterator} &
   type of the  \sectionlink{\concept{Sequence Iterator}}{GridSequenceIterator}
   corresponding to {\tt R::Vertex}
   \par
   model of \sectionlink{\concept{ Grid Vertex Iterator }}{GridVertexIterator}
   \par
    ({\tt R::Vertex} is identical to  {\tt R::VertexIterator::Vertex}) 
   \\ 
   \hline
   \\
 \end{tabularx}

 \conceptsubsection{Valid Expressions}
 \begin{tabularx}{14cm}{RlRR} \\
   \hline
   \bf  Name  &\bf  Expression  &\bf  Type requirements  & \bf  return type  \\ 
   \hline
   start of sequence &
   {\tt vi = r.FirstVertex()} & 
   ~ & 
   {\tt VertexIterator}  
   \\
   end of sequence & 
   {\tt vi = r.EndVertex()} & 
   ~ & 
   {\tt VertexIterator} 
   \\
   size of sequence  & 
   {\tt r.NumOfVertices()} & 
   ~ &
   {\tt size\_type} 
   \\
   handle-to-element conversion &
   {\tt v = r.vertex(h)} & 
   ~ &
   {\tt Vertex} 
   \\
   element-to-handle  conversion   &
   {\tt h = r.handle(v)} & 
   ~ &
   {\tt vertex\_handle} 
   \\
   \hline
   \\
 \end{tabularx}

 \conceptsubsection{Expression semantics}

\T\begin{small}
 \begin{tabularx}{16cm}{RRRRR} \\
   \hline
   \bf  Name    &
   \bf  Expression &
   \bf  Precondition&
   \bf   Semantics &
   \bf   Postcondition
   \\ 
   \hline
   start of  sequence &
   {\tt v = r.FirstVertex()} &
   ~    &
   return iterator pointing to the first vertex of {\tt s} &
   {\tt v} is \footlink{valid}{valid}, or  {\tt v == r.EndVertex()}  
   \\
   end of sequence &
   {\tt v = r.EndVertex()} &
   ~    &
   ~    &
   {\tt v} is past-the-end.
   \\
   size of sequence & 
   {\tt n = r.NumOfVertices()} & 
   ~ &
   get the number of vertices of {\tt s} &
   {\tt n == \Stllink{distance}(r.FirstVertex(), r.EndVertex())} 
   \\
   handle-to-element &
   v = r.vertex(h) &
   {\tt h} is a handle of {\tt r.TheGrid()} &
   Construct the vertex corresponding to the handle {\tt h}  & 
   {\tt r.handle(v) == h} 
   \\
   element-to-handle &
   h = r.handle(v) &
   {\tt v} is a vertex of {\tt r.TheGrid()} &
   Construct the vertex corresponding to the handle {\tt h}  & 
   {\tt r.element(h) == v} 
   \\
   \hline
   \\
 \end{tabularx}
\T\end{small}

 \conceptsubsection{Complexity guarantees}
 The {\tt r.NumOfVertices()} operation has complexity at most O(V), where
 V is  the number of vertices of {\tt r} \noteref{note-complexity}
 
 \conceptsubsection{Models}
 Virtually every concrete grid  or grid range:
 \\
 \sectionlink{{\tt enumerated\_grid\_range}}{EnumeratedGridRange}
 \\
 \sectionlink{\concept{Complex}2D}{Complex2D}
 
 \conceptsubsection{Notes}
 \begin{enumerate}
 \item \notelabel{note-complexity}
   The reason why a general Vertex Range cannot guarantee $O(1)$ complexity for
   this operation is the following: 
   Such a sequence  might arise from a grid range
   given by a simple enumeration
   of its cells 
   (or their handles,
   see for example   \sectionlink{{\tt enumerated\_grid\_range}}{EnumeratedGridRange}).
   The elements of lower dimension belonging to {\tt r} are then implicitely defined
   by the closure (in a topological sense) of the set of cells in the underlying grid.
   The determination of, for example, the set of vertices can be done in expected time
   O(V) (using \sectionlink{\concept{partial grid functions}}{PartialGridFunction}).
   Requiring to do this determination at time of construction of the range would 
   impose the cost of doing so also to clients not interested in this functionality.
 \end{enumerate}
  

