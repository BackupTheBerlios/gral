\xname{AdaptingToInterface}
\begin{Label}{AdaptingToInterface}
\introsection{Creating an Interface Adaption to Your Own Grid Data Structure}
\end{Label}


If you want to use GrAL, you typically want to use its generic
components, but not necessarily its concrete grid data structures
-- at least that's how it is intended.
Therefore, you will need to create an interface layer 
satisfiying the syntax specified in the \Sectionlink{Concepts} section.
Actual working code along the lines of this section 
can be found in the \gralmodulelink{triang2d} module.

For sake of simplicity, let us assume that your grid data structure
is a plain  good old C (or Fortran) one:
Cell-vertex incidences are stored in a plain array \texttt{cells},
such that \texttt{cells[3*c +v]} gives the index of the \texttt{v}'th vertex
of cell \texttt{c}. 
Likewise, an array \texttt{geom} 
holds the coordinates of vertex \texttt{v}
at indices \texttt{2*v} and \texttt{2*v+1}.

\begin{Label}{triang2d-combi}
\introsubsection{Defining the Combinatorial Layer}
\end{Label}

Let's call our grid wrapper class \texttt{Triang2D}:
\begin{example}
class Triang2D \{
  int * cells;

  // to be continued ...
\};
\end{example}
We are going to implement as much of the GrAL interface for \code{Triang2D}
as possible.
From the incidence information present in the array \texttt{cells},
we can define  the following types more or less directly:
\texttt{Vertex, VertexIterator, Cell, CellIterator, VertexOnCellIterator}.
We will implement them as nested classes of \texttt{Triang2D}
(note: the actual code used separate classes).
Also, in order to save some work, we will implement the
\Conceptlink{Vertex} and \Conceptlink{VertexIterator} by one and the same class
\texttt{Triang2D::Vertex},
and the similarly for \Conceptlink{Cell}.


Moreover, as show below, we can also implement edges and related iterators.
So, in sum, we can support the following concepts,
without adding any new data structure
(only the EdgeIterator will have an internal table to keep track of visited edges):

\begin{center}
  \begin{tabular}{lll}
    \Conceptlink{Vertex}      & \Conceptlink{VertexIterator} & \Conceptlink{VertexOnCellIterator} \\
    \Conceptlink{Edge} 
    (= \Conceptlink{Facet}) &   \Conceptlink{EdgeIterator} & \Conceptlink{EdgeOnCellIterator} \\
    \Conceptlink{Cell} &  \Conceptlink{CellIterator} & \\
    & & \Conceptlink{VertexOnFacetIterator} 
  \end{tabular}
\end{center}


\begin{Label}{triang2d-cellvertexinput}
\introsubsubsection{Implementing the basic layer: Vertices and Cells}
\end{Label}


Element handles could be simply integers in this case;
however, we chose to make them distinguishable types,
which makes it easier to catch misuse errors, e.g.
taking vertex handles for cell handles mistakingly.
\begin{example}
class Triang2D \{

  // typedef int vertex_handle; // simplest possibility
  typedef vertex_handle_int<Triang2D> vertex_handle;
  typedef vrll_handle_int  <Triang2D> cell_handle;

  class Vertex \{
     // to be continued ...
  \};
  typedef Vertex VertexIterator;

  class Cell \{
     // to be continued ...
  \};
  typedef Cell   CellIterator;

  class VertexOnCellIterator \{
     // to be continued ...
  \};
\};
\end{example}
We first show the definition of \texttt{Triang2D::Cell},
and then \texttt{Triang2D::VertexOnCellIterator}.

\begin{example}
class Triang2D \{
   class Cell \{
     typedef Cell     self;
     typedef Triang2D grid_type;
   private:
      cell_handle      c;
      grid_type const* g;
   public:
      Cell() : g(0) \{\}
      Cell(grid_type const& gg, cell_handle cc = 0) : g(&gg), c(cc) \{\}  
   
      self      & operator++() \{ ++c; return *this;\}
      self const& operator* () const \{ return *this;\}
      bool IsDone() const \{ return c >= TheGrid.NumOfCells();\}
  
      cell_handle handle() const \{ return c;\}     
      grid_type   const& TheGrid() const \{ return *g; \} 

      unsigned NumOfVertices() const \{ return 3; \}
      VertexOnCellIterator FirstVertex() const 
       \{ return VertexOnCellIterator(*this);\}
   \};

\};
\end{example}
(In pactice, we would have to define \texttt{FirstVertex()} outside
of \texttt{Triang2D::Cell}.)
The class \texttt{Triang2D::Vertex} is of course very similar and 
not shown here. 

Now for the incidence iterator
\code{Triang2D::VertexOnCellIterator} 
which models \Conceptlink{VertexOnCellIterator}:

\begin{example}
class Triang2D \{
 
  class VertexOnCellIterator \{
    // typedefs omitted ...
  private:
    cell_handle c;
    int         lv;
    grid_type   g;
  public:
    VertexOnCellIterator(Cell const& cc) : c(CC.c), lv(0), g(CC.g) \{\}
     
    self&  operator++() \{ ++lv; return *this; \}
    Vertex operator*() const \{ return Vertex(*g, handle());\}
    bool   IsDone() const \{ return (lv >= 3);\}
    
    vertex_handle       handle()  const \{ return g->cells[3*c+lv];\}
    Cell                TheCell() const \{ return Cell(*g,c);\}
    grid_type const&    TheGrid() const \{ return *g;\}
  \};

\}
\end{example}
Now we have defined enough types to satisfy the 
\Conceptlink{Cell-VertexInputGridRange} model.
It suffices to serve a argument for a generic copy 
operation, see the  \Sectionlink{ConstructGrid} operation.

It remains to fill in the \texttt{grid\_types<>} template:
\begin{example}
template<>
struct grid_types<Triang2D> \{
  typedef Triang2D grid_type;

  typedef Triang2D::vertex_handle vertex_handle;
  
  //  etc.
\};
\end{example}

\begin{Label}{triang2d-edge}
\introsubsubsection{Going further: Edge, EdgeIterator and EdgeOnCellIterator}
\end{Label}

We can, however, support a larger portion of the GrAL interface.
Even if we do not store edges, they are implicitely present
as sides of triangles, and we can define an \Conceptlink{Edge}
type for \code{Triang2D}, as well as \Conceptlink{EdgeIterator}
and {EdgeOnCellIterator}. The types \Conceptlink{Facet}
etc. will be just typedefs to the corresponding edge concepts.

First, we note that we have two possibilities of representing 
an edge: First, by its two vertices, 
and second, by an incident cell and the local edge number
in that cell, much like in \code{Triang2D::VertexOnCellIterator}.

Now, the second representation gives readily access 
to the two vertices of the edge, given our data structure,
but not the other way around. 
So, we choose to represent edges (and edge-on-cell-iterators)
by (cell, local edge number) pairs.
If we have to compare two edges for equality,
we must compare the vertex sets, as our primary
representation is not unique (each interior edge
is visited from both sides).

As the internal representations for 
\code{Triang2D::Edge}
and \code{Triang2D::EdgeOnCellIterator} are identical,
we will implement the first in terms of the latter.
Starting with \code{Triang2D::EdgeOnCellIterator}, we see
that it is very similar to  \code{Triang2D::VertexOnCellIterator}:

\begin{example}
class Triang2D \{
 
  class EdgeOnCellIterator \{
    typedef  EdgeOnCellIterator self;
    // other typedefs omitted ...
  private:
    cell_handle c;
    int         le;
    grid_type   g;
  public:
    EdgeOnCellIterator(Cell const& cc) : c(CC.c), le(0), g(CC.g) \{\}
     
    self&  operator++() \{ ++le; return *this; \}
    Edge   operator*() const \{ return Edge(*this);\}
    bool   IsDone() const \{ return (le >= 3);\}

    Vertex V1() const \{ return Vertex(*g,g->cells[3*c+le]);\}
    Vertex V2() const \{ return Vertex(*g,g->cells[3*c+(le+1)\%3]);\}
    

    edge_handle         handle()  const \{ return edge_handle(c,le);\}
    Cell                TheCell() const \{ return Cell(*g,c);\}
    grid_type const&    TheGrid() const \{ return *g;\}

    friend bool operator==(self a, self b)
    \{ return (a.c == b.c && a.le == b.le);\}
  \};

\}
\end{example}
The type \code{Triang2D::edge\_handle} is 
a simple struct containing cell handle and local number.

The class \code{Triang2D::Edge} 
just contains an object of type 
\code{Triang2D::EdgeOnCellIterator}.
The only major difference is the comparison operator
\code{==}:

\begin{example}
class Triang2D \{
  
  class Edge \{
    EdgeOnCellIterator e;
  public:
    Edge(EdgeOnCellIterator ee) : e(ee) \{\}
    // ... most things forwarded to e

    friend bool operator==(Edge a, Edge b)
     \{ return (  (a.V1() == b.V1() && a.V2() == b.V2())
               || (a.V1() == b.V2() && a.V2() == b.V1())); \}
  \};
\};
\end{example}

So far, so good. Now, how can we iterate over all
edges of \code{Triang2D} grids?
We can do so be iterating over all cells and iteration over
all edges incident to those cells.
As every internal edge is visited twice,
we must keep track of already visited ones.
This can be done with an hash table, taking edges as keys
(for the definition of  a hash function on edges, see below).
So, \code{Triang2D::EdgeIterator} contains 
\code{Triang2D::EdgeOnCellIterator} and this hash table.
As this strategy is often employed, it has been
implemented in a generic component:

\begin{example}
class Triang2D \{

   class EdgeIterator 
    : public facet_set_of_cells_iterator<Triang2D::Cell>
   \{
     typedef facet_set_of_cells_iterator<Triang2D::Cell> base;
    public: 
     EdgeIterator() : base(CellIterator()) \{\}
     EdgeIterator(CellIterator  const& c) : base(c) \{\}
     EdgeIterator(grid_type const& g)  : base(CellIterator(g)) \{\}
  \};
\};
\end{example}
BTW, the class \gralclasslink{facet\_set\_of\_cells\_iterator}{base}
can also be used for finding the facets of (the closure of)
an arbitrary subset of the cells of any grid.


Now, our combinatorial layer is complete, 
the iterators that are missing cannot be supported efficiently
with the given data.
If we want to use algorithms which require more (e.g. cell-cell 
adjacencies), we have to calculate the additional information.
Probably, the most comfortable thing to do would be
to use a view on \code{Triang2D} which supports the 
required functionality, reusing what is already there.
At present, there is no view in GrAL 
for adding cell-cell adjacencies, but it would certainly be 
a useful addition.

\begin{Label}{triang2d-geom}
\introsubsection{Defining the Geometrical Layer}
\end{Label}


If we want to use geometric algorithms,
we need to define a geometry for \texttt{Triang2D}.
At first sight, this looks rather simple,
but there is a subtle problem, 
relating to the fact that vertex coordinates are stored 
in a plain array of doubles:

\begin{example}
class geom_Triang2D_base {
   double* coords;
   typedef ... coord_type;  // 2D point type

   ??? const& coord(Vertex v) const;
   ???      & coord(Vertex v);
};
\end{example}

What should we return from 
the non-const \texttt{coord(Vertex v)}?
On the one hand, we want allow code like
\texttt{coord\_type p = geom.coord(v)},
on the other hand, an assignment like
\texttt{geom.coord(v) = coord\_type(1,0)} should change the
array \texttt{coords}.
Thus, we cannot return \texttt{coord\_type \&}.

The solution is to define a proxy type:

\begin{example}
class geom_Triang2D_base {

   class coord_proxy {
     double * coo;
     void operator=(coord_type const& p) 
     {  coo[0] = p[0]; coo[1] = p[1]; }
   };

   coord_proxy        coord(Vertex v) 
   { return coord_proxy(coords[2*v.handle()]);}
};
\end{example}


For the `real' \texttt{coord\_type}, 
we are free to choose any suitable implementation
of a 2D geometric point.
In order to get a bunch of useful geometric functionality
(like volumes or centers), 
we can use a generic 2D geometry:

\begin{example}
 typedef geometry2d<geom_Triang2D_base>  geom_Triang2D;
\end{example}

The efficiency of this generic implementation will depend
on how well the compiler can optimize out the overhead.
If it turns out to be too inefficient, 
parts have to be specialized for \texttt{Triang2D}
using direct  access to its data. 
This does not break the generic paradigm,
as grid geometries are part of the kernel.


\begin{Label}{triang2d-gf}
\introsubsection{Defining the Grid Function Layer}
\end{Label}

This part is actually quite easy, as predefined components
for the common cases exist.
In the case of vertices and cells, which are
numbered consecutively, we take the class
template \gralclasslink{grid_function_vector}{base}
as our implementation:

\begin{example}
template<class T>
class grid_function<Triang2D::Vertex, T>
 : public grid_function_vector<Triang2D::Vertex, T> 
{
 // repeat constructors here.
};
\end{example}
The same is done for \code{Triang2D::Vertex}.
The actual code can be found in 
\gralfilelink{grid-functions}{h}{triang2d}.

For edges, there is no consecutive numbering we can exploit.
Thus, we have to use a different approach for grid functions
on edges. One viable way is to use hash tables. 
This is so common (it is also used for partial grid functions),
that we can again use a generic component:
\begin{example}
template<class T>
class grid_function<Triang2D::Edge, T> 
  : public grid_function_hash<Triang2D::Edge, T> 
{
  // repeat constructors here.
};
\end{example}
In order to make it work this way, we have to somehow
pass a hash function to the \code{grid_\function\_hash<>}
template. This is done by specializing the traits class
\code{element\_traits<>} for \code{Triang2D::Edge}
and providing a \code{hasher\_type}:
\begin{example}
template<>
struct element_traits<Triang2D::Edge> 
  : public \gralstructlink{element_traits_edge_base}{base}<Triang2D> 
{
  struct hasher_type : public grid_types_Triang2D {
    enum { p= 17}; // somewhat arbitrary
    size_t operator()(Triang2D::Edge const& e) const { 
      vertex_handle v1 = e.V1().handle();
      vertex_handle v2 = e.V2().handle();
      return (v1 > v2 ? p*v1 + (v2\%p) : p*v2 + (v1\%p));
    }
  };
};
\end{example}
The actual code is found in 
\gralfilelink{element-traits}{h}{triang2d}.
We do the same for \code{Triang2D::Vertex} and \code{Triang2D::Cell}.
Here the hash function just returns the handle.

For defining partial grid function, there effectively does not remain
anything to do (given the provision of a \code{hasher_type} by
\code{element\_traits} above), because they are already defined in a generic
way. Thus, in \gralfilelink{partial-grid-functions}{h}{triang2d},
we just include the generic \gralfilelink{partial-grid-function-hash}{h}{base}.

\begin{Label}{triang2d-mutating}
\introsubsection{Defining the Mutating primitives}
\end{Label}

If we just need a wrapper for read-only access to an existing
dat structure, we are fine with what we have achieved so far.
With the interface created above, we are able to use all generic
algorithms on \code{Triang2D} which do not change the grid.
Provided, of course, the incidence information supplied by 
\code{Triang2D} is sufficient for the algorithm.
If not, we would have to create additional data structures
-- the interface we have defined gives access to all information
inherent in the data structure.

Now, we might for example want to read from a file format
for which there is a GrAL input adapter:

\begin{example}
Triang2D t;
InputWeirdFileFmt In("mygrid.weird");
ConstructGrid0(t, In);
\end{example}

This will copy the connectivity information from the file
\code{mygrid.weird} to the triangulation \code{t}.
In order for this to work, 
we need to specialize the 
\code{ConstructGrid0} primitive for \code{Triang2D}.
(Suffixes like 0 at the end of \code{ConstructGrid0}  
are just technical hacks
for distinguishing different overloaded versions.)

\begin{example}
template<class GRID>
void ConstructGrid0(Triang2D     & t,
                    GRID    const& g_src);
\end{example}
In the general case, we need some relationship between
\code{g\_src} and \code{t} (associative copy),
so we use that version as a master implementation
which all other versions of the \code{ConstructGrid} family use:

\begin{example}
template<class GRID, class VTX_CORR, class CELL_CORR>
void ConstructGrid0(Triang2D     & t,
                    GRID    const& g_src,
                    VTX_CORR     & src2t_v,
                    CELL_COOR    & src2t_c);
\end{example}
Afterwards, \code{src2t\_v} will map the vertices of \code{g\_src}
to those of \code{t} in a 1:1 fashion, and \code{src2t\_c}
will do the same for cells.

The actual implementation is rather simple in our case:
\begin{enumerate}
\item The vertices of \code{g\_src} are mapped to those of
\code{t}, i.e., numbered consecutively from 0 to nv-1,
where nv = \code{g\_src.NumOfVertices()}.
This creates the mapping \code{src2t\_v}.
\item Next, storage is allocated for the \code{cells} array,
using \code{g\_src.NumOfCells()}.
\item Finally, the cells of \code{t} are created, by using
the cell and the vertex-on-cell iterator types of \code{GRID},
and the vertex correspondence of \code{src2t\_v}.
 At the same time, the mapping \code{src2t\_c} is filled in.
\end{enumerate}

The complete code can be found in 
\gralfilelink{construct}{C}{triang2d}
and \gralfilelink{copy}{C}{triang2d}.

\begin{Label}{triang2d-enlarge}
\introsubsubsection{Defining the \code{EnlargeGrid} primitive}
\end{Label}

To be written.