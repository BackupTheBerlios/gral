\xname{ConstructGrid}
\begin{Label}{ConstructGrid}
\algosection{ConstructGrid}
\end{Label}

\needswork{This documentation is still incomplete!}

\algosubsection{Declaration}
This is actually a family of related functions:
\begin{example}
[1]
template<class G1, class G2>
inline void 
ConstructGrid0(G1            & destG, 
               G2       const& srcG);

[2]
template<class G1, class G2, class VertexMap, class CellMap>
extern void 
ConstructGrid0VC(G1            & destG, 
                 G2       const& srcG,
                 VertexMap     & V21,
                 CellMap       & C21); 

[3]
template<class G1, class Geom1, class G2, class Geom2>
extern void 
ConstructGrid(G1         & destG,
              Geom1 const& destGeom,
              G2    const& srcG, 
              Geom2 const& srcGeom);

[4]
template<class G1, class Geom1, 
         class G2, class Geom2, class VertexMap>
extern void 
ConstructGridV(G1            & destG, 
               Geom1         & destGeom,
               G2       const& srcG,
               Geom2    const& srcGeom,
               VertexMap     & V21); 

[5]
template<class G1, class Geom1, 
         class G2, class Geom2, class VertexMap, class CellMap>
extern void 
ConstructGridVC(G1            & destG, 
                Geom1         & destGeom,
                G2       const& srcG,
                Geom2    const& srcGeom,
                VertexMap     & V21,
                CellMap       & C21);
\end{example}

\algosubsection{Description}
The \code{ConstructGridXXX} family of template functions
make an \Glossarylink{associative copy} 
of a source grid \variable{Gsrc} to a destination grid \variable{Gdest}.
The  \code{ConstructGrid0XXX} functions ([1], [2])
do only a copy of combinatoric grids,
the other copy also geometry information.

The templates above are actually not really fully generic ---
they must be specialized to the type of the first parameter, \type{G1}.
The more general grids the \type{G1} grid type can represent,
the more useful these functions are.
For very specialized grids, like Cartesian ones, 
they do not make much sense.

This algorithm is in some sense the analogue to the STL
\stllink{\type{copy}}{copy} algorithm for sequences,
only that it cannot be fully generic on the destination argument.
\algosubsection{Definition}
Declared in \gralcodelink{construct-grid.h}{base}.
\\
Specialized in 
\gralcodelink{complex2d-construct.C}{complex2d}
for \sectionlink{\type{Complex2D}}{Complex2D}
and
\gralcodelink{construct.C}{triang2d}

\algosubsection{Template parameters}

\algosubsection{Type requirements}
\type{Geom1} is a model of \conceptlink{Mutable Vertex Grid Geometry}{VertexGridGeometry}
\type{G2} is a model of \conceptlink{Cell-Vertex Input Grid Range}{Cell-VertexInputGridRange}
\\ 
\type{Geom2} is a model of \conceptlink{Vertex Grid Geometry}{VertexGridGeometry}
\\
\type{VertexMap} is a model of \sectionlinkUNDEF{Mutable Mapping}{MutableMapping}
from \type{G2::vertex\_handle} to \type{G1::vertex\_handle}
\\
\type{CellMap} is a model of \sectionlinkUNDEF{Mutable Mapping}{MutableMapping}
from \type{G2::cell\_handle} to \type{G1::cell\_handle}

\algosubsection{Preconditions}
\variable{srcGeom} is \glossarylink{bound}{bound to a grid} to 
\variable{srcG}.

\algosubsection{Postconditions}
\variable{destG} contains a copy of \variable{srcG},
and ([3]-[5]) \variable{destGeom} contains a copy of 
\variable{srcGeom},
such that the mappings 
\variable{V21} and \variable{C21} induce a 
\Glossarylink{grid morphism}
from \variable{srcG} to \variable{destG}.

\algosubsection{Complexity}
Linear in the size of \variable{srcG}.

\algosubsection{Example}
\begin{example}
\sectionlinkfoot{RegGrid2D}{RegGrid2D} R(10,10);  // 10x10 Cartesian grid
\sectionlinkUNDEF{mapped_geometry_reg2d<Linear2D>}{MappedGeometryReg2D}  geomR(R, Linear2D::identity); // 
\sectionlink{Complex2D}{Complex2D} G;         // empty grid
\sectionlinkUNDEF{stored_geometry_complex2d}{StoredGeometryComplex2D} geomG(G);
ConstructGrid(G,geomG, R,geomR); // copy R to G;
assert(G.NumOfVertices() == R.NumOfVertices());
assert(G.NumOfCells()    == R.NumOfCells());
typedef grid\_types<Complex2D> gt; // 'namespace' for Complex2D-related types 
for(gt::CellIterator c(G), ! c.IsDone(); ++c)
  cout << "Cell " << c.handle() << "  "
       << "has "  << (*c).NumOfVertices() << " vertices";
\end{example}

\algosubsection{Uses}

\algosubsection{Used by}
The operation of copying is so fundamental that
it is used virtually everywhere a grid has to be created.
For example, the algorithm is used for implementing 
the member function \code{Complex2D::read(istream\&)}, 
defined in \gralcodelink{complex2d.C}{complex2d}
Here an adapter \sectionlink{\type{IstreamComplex2DFmt}}{istream-complex2d-fmt}
for a grid stored in a file is used.
\algosubsection{Notes}

\algosubsection{See also}
\sectionlinkUNDEF{EnlargeGrid}{EnlargeGrid}

