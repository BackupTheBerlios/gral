
\xname{DataAssociationLayer}
\begin{Label}{DataAssociationLayer}
\introsection{Data associations}
\end{Label}

      In any serious application, we need to deal with values attached to k-elements:
      for example, state variables living on vertices in a typical FEM-program,
      costs on edges in network-flow algorithms or state variables on cells
      in a cell-centered Finite-Volume-code.
      There are two different aspects of this necessity:
      
      On the one hand, such mappings are often passed to algorithms.
      Then, algorithms should make only minimal assumptions on the nature 
      of the mapping, in particular, they should not commit to a particular representation
      of such data associations.

      On the other hand, algorithms themselves often need to attach values to
      element temporarily. In this case, they have to be able to create such data
      associations themselves, and perhaps to decide on some performance-critical
      features of the implementations.
      
      The underlying mathematical model of the first usage scenario 
      is that of a function $f$ from the set 
      of k-Elements of a Grid \T${\cal G}$ \W$G$ to values of an arbitrary type {\em T\/}:

      \begin{center}
        \begin{figure}[h]
          \W\rhtml{<TABLE> <TR>}
          \T\begin{minipage}{6cm}
            \T\includegraphics[width=5cm]{bilder/gf-map-cell}
            \W\rhtml{<TD>}
            \W\htmlimage[ALT="Grid function on cells"]{\img{gf-map-cell}}
            \caption{grid function on cells}
            \W\rhtml{</TD>}
            \T\end{minipage}  \T\hspace{1cm}
          \T\begin{minipage}{6cm}
            \T\includegraphics[width=5cm]{bilder/gf-map-edge}
            \W\rhtml{<TD>}
            \W\htmlimage[ALT="Grid function on edges"]{\img{gf-map-edge}}
            \caption{Grid function on edges}
            \W\rhtml{</TD>}
            \T\end{minipage}
          \W\rhtml{</TR> </TABLE>}
        \end{figure}
      \end{center}



      The grid function concept \sectionlink{(C)}{GridFunction},
      is a natural interface
      which is close to the  mathematical model.
      This interface can be implemented by many different representations,
      including calculation on-the-fly, and corresponds to the first
      requirement (data abstraction).
      
      Beyond this, the second scenario (temporary values / local grid functions)
      requires a  container component,
      which extends this interface with
      methods for creation of grid functions. 
      The corresponding concept 
      is {\em container grid function\/}  \sectionlink{(C)}{ContainerGridFunction}.

      There is a interesting and important choice left open now:
      Either, we can reserve memory for the value of each element in the grid,
      and considering each value not explicitly set on an element an error.
      Or, we could determine a default value, and consider each element whose
      valeu is not explicitely set having this default value.
      The first option leads to the {\em total\/} grid function concept
      \sectionlink{ (C) }{TotalGridFunction},
      storing a value for every element.
      
      The second option gives rise to  {\em partial\/}
      grid functions     \sectionlink{ (C) }{PartialGridFunction}
      which stores values  only on a part of the grid elements,
      assigning a default value to the others.
      
      Depending on the choice, there are slightly different semantics of the constructors,
      as well as same new operations in the case of  partial grid functions.
      
      Owing to the great importance of both total and partial grid function containers
      for the implementation of many algorithms, these entities should be 
      available under a uniform syntax, if they are defined for a concrete grid type.
      More precisely, the following class templates are then available: 
     {\small ({\tt E} is constrained to be an element type \sectionlink{(C)}{GridElement} 
      associated to this grid type, {\tt T} can be chosen arbitrary.)}
     
\begin{example}
template<class E, class T> 
class grid\_function; // total grid function 
template<class E, class T>
class partial\_grid\_function; // partial grid function
\end{example}

There are examples of grids types without such associated grid functions.
These are typically used for special restricted purposes only,
like grid adapters for input from a file, which offer a clean way 
to encapsulate specific file formats from the rest of the world.
An example is 
\sectionlink{{\tt IstreamComplex2DFmt}}{istream-complex2d-fmt}.


\begin{Label}{refine-example}
{\bf Example:}
We can use  a partial grid function to mark cells for refinement:
\begin{example}
partial\_grid\_function<Cell,bool> refine(G,false);
....
for(CellIterator c = G.FirstCell(); ! c.IsDone(); ++c)
 if(error(*c) > tolerance())
   refine[*c] = true;
\end{example}
\end{Label}
