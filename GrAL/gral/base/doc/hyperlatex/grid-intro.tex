
\xname{GridIntro}
\begin{Label}{GridIntro}
\introsection{Overview over the Grid Category}
\end{Label} 

  In the \sectionlinkshort{preceding section}{GridsAlgorithmsAndReuse},
  we have seen that the major obstacle for reusing grid-related software
  is direct access to low-level representation details,
  together with to inavoidable high variability of such details
  in the field of geometric data structures.

  In order to  decouple implementations of algorithms 
  (or other data structures)
  from these details, we will develop in the following 
  a uniform, abstract interface layer for grids.

  How can one obtain such an interface? 
  Let us look again at the simple 
  \xlink{algorithm for surface calculation}{GridsAlgorithmsAndReuse.html\#algo-surface}:

   \begin{figure}[h]
     \begin{center}
       \begin{Label}{algo-surface-2}
       \T\includegraphics{bilder/algo-surface}
       \W\htmlimage[ALT="A simple algorithm calculating cell surfaces"]{%
          \img{algo-surface}}
        \end{Label}
     \end{center}
   \end{figure}

   What kind of functionality does this algorithm need?
   \begin{enumerate}
   \item \label{iter-cell} Iteration over all cells of a grid
   \item \label{iter-facet-on-cell} Iteration over all facets of a cell
   \item \label{gf-real-on-cell} Storage of real numbers on cells
   \item \label{geom-volume} Calculation of volumes of  facets 
     (i.\ e.\ area in 3D, length in 2D)
   \end{enumerate}
  
   Items \ref{iter-cell} and \ref{iter-facet-on-cell} 
   can be summarized as iteration or combinatorial functionality.
   Item \ref{gf-real-on-cell} obeys to the general pattern of storing
   data on (or associating data to) grid elements.
   And item \ref{geom-volume} attaches geometric meaning to the combinatorial
   entitiy ``facet''.

   Thus, in this case, functionality needed by  the algorithm falls
   into three classes. This, however, is no accident due to the simplicity
   of the algorithm, but a pattern found also when investigating 
   more complex algorithms.
  To sum up, we distinguish the following three fundamental layers 
  of functionality:
  \begin{itemize}
  \item  the \sectionlink{combinatorial layer}{CombinatorialLayer}
  \item the \sectionlink{data association layer}{DataAssociationLayer}
  \item the \sectionlink{geometrical layer}{GeometricalLayer} 
  \end{itemize}
  
  The {\em combinatorial layer\/} views a grid 
  simply as a \glossarylinkemph{lattice},
  or, more generally, a \glossarylinkemph{poset},
  given by the {\em incidence\/} relationship.
  The entities involved are of  a purely discrete nature;
  there is not notion of coordinates of vertices or the like.
  The notion of dimension used here is a combinatorial or {\em intrinsic\/}
  dimension.
  
  The {\em grid function\/} layer adds the possibility to associate data 
  with the discrete grid elements, 
  for example to put marks on vertices,
  or to store solution values on cells.
  Mathematically, it captures the notion of a function from
  the discrete set of grid elements of any fixed dimension 
  to entities of some given type, for example reals or integers.

  
  The {\em geometrical layer\/} provides an embedding into a concrete geometrical space of 
  an ({\em external\/}) dimension which is at least the intrinsic dimension. 
  For example, a grid of (intrinsic) dimension 2 could be embedded in 3-space,
  like the surface of a torus.
  It corresponds to the notion of a \glossarylinkemph{geometric realization}
  of a combinatorial complex.

  
  We will discuss these layers in an informal way 
  in the introductory sections on the
  \sectionlink{combinatorial}{CombinatorialLayer}, 
 \sectionlink{grid function}{DataAssociationLayer}
 and \sectionlink{geometrical}{GeometricalLayer} layers, respectively.

 A more formal description of the
 \glossarylink{concepts}{concept} involved starts in 
 \sectionlink{the concept section.}{Concepts}.
 Here C++ syntax is prescribed, in a way analogous to
 the STL documentation.
 We will refer to them with a trailing (C).
 This part is of cours crucial for actually implementing generic components;
 it is not so important for understanding the bib picture
 (because it contains inevitably many arbitrary decisions).

 In a short lookahead, here is the generic version of the 
 algorithm  \xlink{presented before}{GridIntro.html\#algo-surface-2}:

 \begin{example}
\conceptlinkfoot{grid_function}{TotalGridFunction}<\Conceptlinkfoot{Cell},double> surface(Grid);
for(\Conceptlinkfoot{CellIterator} c(Grid); !c.IsDone(); ++c) {
  surface[*c] = 0.0; 
  for(\Conceptlinkfoot{FacetOnCellIterator} fc(c); !fc.IsDone(); ++fc)
    surface[*c] += \conceptlinkfoot{Geometry}{VolumeGridGeometry}.volume(*fc);
}
 \end{example}

 In the \sectionlinkshort{third part}{Components} of the documentation,
 concrete \sectionlink{components}{Components}
 are described, sorted by functionality 
 (algorithms, grid ranges, iterators and so on).
 These components separate into two layers, 
 depending on whether they are  \emph{leaf components\/}
 (for example basic grid implementations),
 or \emph{generic extensions}, that is, 
 classes which provide additional functionality 
 for each grid component that is a \Glossarylink{model} of a certain \Glossarylink{concept}.
 In fact, most components belong to this second layer,
 which evidently greatly enhances reuse.
  
  \bigsep 

  In the present framework, a mathematical grid is not represented by a {\em single\/}
  entity (e.g. a class), 
  rather, {\em different\/} classes cooperate to this aim.
  This is already visible by the dissection into the three fundamental categories,
  namely combinatorics, data association and geometry.
  
  Moreover, each of these categories introduces many sub-categories 
  and refined concepts, 
  especially the combinatorial one.
  In order to understand how grids are represented by these concepts,
  it is important to understand their mutual relationship.

  For each concrete grid type, there is one class implementing
  the abstract \sectionlink{grid concept}{Grid}.
  {\small (In the STL parlance, it is \glossarylink{{\em models\/}}{model} of the concepts.)}
  To actually use this grid type, one has to access a number of
  different classes accounting for a representation 
  of this mental "The Grid" thing one may have in mind.
  Thus, there are seperate classes for 
  \link{grid elements}{element-intro} (vertices, edges, cells),
  \sectionlink{iterators}{GridSequenceIterator},
  and \sectionlink{grid functions}{DataAssociationLayer}.
  The concrete grid types act as a sort of {\em central entry point\/} 
  giving access to these {\sl associated types\/},
  which one can think of  as  forming a sort of `halo'
  around the central grid type.

  The number of supported types  will not be the same for all grids.
  The basic \conceptlink{Grid}{Grid} concepts requires any types at all
  --- everything useful comes in through refinements of this concept.
  A grid even does not need to define all possible element types ---
  there are grid types used only for I/O 
  (see \sectionlink{\type{IstreamComplex2DFmt}}{istream-complex2d-fmt})
  which do with a quite
  minimalistic set of associated types.
  Most profoundly, grids differ in the amount of
  \link{incidence iteration}[~(see section \Ref)~]{incidence-it-intro}
  they can support --- a trade-off between storage consumption and functionality.


  At first reading, 
  one can consider this type association as fixed.
  But in principle, it is possible to use different `halo' types 
  with one and the same grid type.
  This technique allows a fine control of grid functionality
  which is not attainable with a monolithic approach.

  \label{intro-grid-types}
  An important  technical question is now  
  how to access `halo types' in a consistent and uniform manner. 

  The way chosen here is to collect all non-generic grid-related types into
  a dedicated entity called {\tt grid\_types},
   associated to a concrete grid type.
  (With `non-generic', we mean types that have to be specialized for each
  concrete grid type.) 
  In practice, this is just a class template that has to be
  \glossarylink{specialized}{partial specialization} for concrete grids.

  This mechanism provides algorithms with a single entry point to 
  type information (and perhaps other) related to a grid parameter.
  It also offers a clean possibility to pass a different `halo',
  that is, a customized set of types which behave differently.
  For example, one could use special iterators for doing 
  limited algorithm animation in an automatic way.
  Currently, only few algorithms are actually parameterized by 
  grid types, an example is the \type{CGT} parameter of 
  \sectionlink{CalculateNeighborCells}{cell-neighbor-search}.
  
  We may contrast this  with the corresponding mechanisms for STL containers:
  These, too, are associated with some service types, namely the 
  iterator type(s), which are accessed by means of nested typedefs:
  \begin{example}
    Container C;
    typedef Container::iterator iter_type;
    iter_type i = C.begin();
  \end{example}
  In our approach, an analogous action would look like the following:
  \begin{example}
    MyGrid g;
    typedef grid_types<MyGrid> gt; // default set of related types
    gt::VertexIterator vi = g.FirstVertex();
  \end{example}
  A {\em quantitative\/} difference to the STL is, that in the grid category,
  many more types are necessary. 
  The total functionality making up a grid is also more broadly
  distributed among these types.

  An important {\em qualitative\/} difference is the parameterization of algorithms.
  STL algorithms are never parameterized by containers but always by iterators
  forming a logical range {\tt [begin, end)}. This captures well the only aspect of 
  containers the algorithms are interested in, namely that they represent linear 
  sequences.
  In contrast, algorithms for grids are parameterized by grids, and often also
  by grid functions or grid geometries, which correspond  in some sense
  to the STL containers. 
  This is due to the fact that grids in this case act as placeholders
  for the underlying mathematical structures on top of which the algorithm
  performs its tasks.

  The set of types used by these algorithms is rich, corresponding to
  the richer structures of the domain. Moreover, these types are not
  independent, by closely relate to each other.
  Therefore, the responsibility for bundling the right types together
  has to be concentrated somewhere.
  This is in essence what the {\tt grid\_types} mechanism does:
  It maps grid types to a set of related types.
  If one want to control the way  algorithms access grid-related types,
  one has to parameterize them with such mappings, 
  in addition to their regular parameters.
 

  

