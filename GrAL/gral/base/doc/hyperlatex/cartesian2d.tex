\xname{RegGrid2D}
\begin{Label}{RegGrid2D}
\typesection{{\tt RegGrid2D} Grid Type}
\end{Label}

\needswork{This documentation is still incomplete!}

\typesubsection{Description}
The type {\tt RegGrid2D} is a model of \sectionlink{\concept{Grid}}{Grid}.
It allows representation of Cartesian grids, that is, regular $m \times n$
tensor product grids.

\typesubsection{Example}

\begin{example}
#include "Grids/Reg2D/cartesian-grid2d.h"

\sectionlinkfoot{RegGrid2D}{RegGrid2D} R(10,10);  // 10x10 Cartesian grid
assert(R.NumOfVertices() == 10*10);
assert(R.NumOfCells()    == 9*9);
typedef grid\_types<RegGrid2D> gt; // 'namespace' for RegGrid2D-related types 
for(gt::CellIterator c(R), ! c.IsDone(); ++c)
  cout << "Cell " << c.handle() << "  "
       << "has "  << (*c).NumOfVertices() << " vertices\\n";
\end{example}

\typesubsection{Definition}

Defined in \gralfilelink{cartesian-grid2d}{h}{cartesian2d}
\texorhtml{}{(see also \xlink{cartesian2d}{\GRALINC{cartesiand2d}{index}})}.
  
\typesubsection{Model of}
\sectionlink{\concept{Grid-With-Boundary}}{Grid-With-Boundary}
\\
\sectionlink{\concept{Vertex Grid Range}}{VertexGridRange}
\\
\sectionlink{\concept{Edge Grid Range}}{VertexGridRange}
\\
\sectionlink{\concept{Facet Grid Range}}{VertexGridRange}
\\
\sectionlink{\concept{Cell Grid Range}}{VertexGridRange}

\typesubsection{Members}
\begin{tabularx}{15cm}{XXX} \hline
  \bf  Member   &
  \bf  Where defined &
  \bf  Description 
  \\ \hline
  \multicolumn{3}{c}{{\bf \em Types}}  \\ \hline
  \multicolumn{3}{c}{{\em handle types\/}}  \\ \hline
  \type{vertex\_handle} &
  \sectionlinkshort{\concept{Vertex Grid Range}}{VertexGridRange} &
  \sectionlinkshort{ handle }{GridElementHandle}   type
  corr. to \type{Vertex} 
  \\
  \type{edge\_handle} &
  \sectionlinkshort{\concept{Edge Grid Range}}{VertexGridRange} &
  \sectionlinkshort{handle}{GridElementHandle}   type
  corr.\ to \type{Edge}
  \\
  \type{facet\_handle} &
  \sectionlinkshort{\concept{Facet Grid Range}}{VertexGridRange} &
  \sectionlinkshort{ handle }{GridElementHandle}   type
  corr.\ to \type{Facet}
  \\
  \type{cell\_handle} &
  \sectionlinkshort{\concept{Cell Grid Range}}{VertexGridRange} &
  \sectionlinkshort{handle}{GridElementHandle}   type
  corr.\ to \type{Cell}
  \\ \hline
  \multicolumn{3}{c}{{\em element types\/}} \\ \hline
  \sectionlinkshort{\type{Vertex}}{RegGrid2DVertex} &
  \sectionlinkshort{\concept{Vertex Grid Range}}{VertexGridRange}&
  The \sectionlinkshort{\concept{Vertex}}{GridVertex} element type 
  \\
  \sectionlinkshort{\type{Edge}}{RegGrid2DEdge} &
  \sectionlinkshort{\concept{Edge Grid Range}}{VertexGridRange}&
  The \sectionlinkshort{\concept{Edge}}{GridEdge} element type 
  \\
  \sectionlinkshort{\type{Facet}}{RegGrid2DFacet} &
  \sectionlinkshort{\concept{Facet Grid Range}}{VertexGridRange}&
  The \sectionlinkshort{\concept{Facet}}{GridFacet} element type 
  \\
  \sectionlinkshort{\type{Cell}}{RegGrid2DCell} &
  \sectionlinkshort{\concept{Cell Grid Range}}{VertexGridRange} &
  The \sectionlinkshort{\concept{Cell}}{GridCell} element type 
  \\ \hline
  \multicolumn{3}{c}{{\em sequence iterator types\/}} \\ \hline
  \type{VertexIterator} &
  \sectionlinkshort{\concept{Vertex Grid Range}}{VertexGridRange} &
  The  \sectionlinkshort{\concept{ Sequence Iterator}}{GridSequenceIterator}   
  type for \type{Vertex}
  \\
  \type{EdgeIterator} &
  \sectionlinkshort{\concept{Edge Grid Range}}{VertexGridRange}&
  The  \sectionlinkshort{\concept{ Sequence Iterator}}{GridSequenceIterator}   
  type for \type{Edge}
  \\
  \type{FacetIterator} &
  \sectionlinkshort{\concept{Facet Grid Range}}{VertexGridRange}&
  The  \sectionlinkshort{\concept{ Sequence Iterator}}{GridSequenceIterator}   
  type for \type{Facet}
  \\
  \type{CellIterator} &
  \sectionlinkshort{\concept{Cell Grid Range}}{VertexGridRange}&
  The  \sectionlinkshort{\concept{ Sequence Iterator}}{GridSequenceIterator}   
  type for \type{Cell}
  \\ 
  \hline
  \multicolumn{3}{c}{{\bf \em Functions\/}} \\ 
  \hline
  \multicolumn{3}{c}{{\em constructors\/}}  \\ 
  \hline
  RegGrid2D(int m, int n) &
  ~ &
  construct cartesian grid with $n$ vertices in $x$ direction
  and $m$ vertices in $y$ direction. 
  \\
  \hline
  \\
  \multicolumn{3}{c}{{\em sequence iteration\/}}  
  \\ 
  \hline
  \type{VertexIterator} \par \type{FirstVertex()} &
  \sectionlinkshort{\concept{Vertex Grid Range}}{VertexGridRange} &
  Iterator pointing to the first vertex 
  \\ 
  \multicolumn{3}{c}{ ---  {\em same for\/} \type{Edge}, \type{Facet}, \type{Cell} {\em types\/} --- } \\ \hline
  \multicolumn{3}{c}{{\em sequence sizes\/}}  \\ \hline
  \type{int NumOfVertices()} &
  \sectionlinkshort{\concept{Vertex Grid Range}}{VertexGridRange} &
  number of vertices 
  \\
  \multicolumn{3}{c}{ --- {\em same for\/} \type{Edge}, \type{Facet}, \type{Cell} {\em types\/} --- } \\ \hline
\end{tabularx}

\typesubsection{See also}
\sectionlinkUNDEF{\concept{ }\type{Triang2D} }{Triang2D} ~
\sectionlink{\concept{ }\type{Complex2D} }{Complex2D} ~
    
  

