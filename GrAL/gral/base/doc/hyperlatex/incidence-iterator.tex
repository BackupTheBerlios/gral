 \xname{GridIncidenceIterator}
\begin{Label}{GridIncidenceIterator}
 \conceptsection{Grid Incidence Iterator Concept}
\end{Label}

 \conceptsubsection{Description}
 A {\em Grid Incidence Iterator\/} allows to access all elements of a given type incident
 to an given element (the {\em anchor\/}), 
 for example  all vertices incident to a given cell,
 or all edges incident to a given vertex. 
 
 For each combination of 
 \sectionlink{\concept{element types}}{GridElement}, 
 there is a subconcept of incidence iterator,
 for example,  the 
 \sectionlink{\concept{Vertex}-On-Cell Iterator}{Vertex-On-CellIterator}.

 \conceptsubsection{Refinement of}
 STL \xlink{Forward Iterator}{http://www.sgi.com/Technology/STL/ForwardIterator.html}
 \\
 \sectionlink{\concept{Grid Entity}}{GridEntity}
 
 \conceptsubsection{Notation}
 {\tt I} is a model of incidence iterator
 \\ 
 {\tt i, j} are  objects of type  {\tt I}
 
 \conceptsubsection{Associated types}
 \begin{tabularx}{12cm}{llR}\\
   \hline
   \bf  Name  &\bf  Expression  &\bf  Description   \\ 
   \hline
   grid type  &  
   {\tt I::grid\_type} &
   type of the iterators underlying grid (range)   \\ 
   anchor type  & 
   {\tt I::anchor\_type} &
   type of the iterators anchor element  \\ 
   element type  &
   {\tt I::element\_type} &
   type of the \sectionlink{\concept{Element}}{GridElement} that the iterator references   \\ 
   value type  & 
   {\tt I::value\_type} &
   synomym for {\tt I::element\_type} \\ 
   \hline
   \\
 \end{tabularx}
 

 \conceptsubsection{Valid Expressions}
 \begin{tabular}{llll} \\
   \hline
   \bf  Name  &\bf  Expression  &\bf  Type requirements  & \bf  return type  \\
   \hline
   prefix increment  &
   {\tt ++it;} &
   ~ &
   {\tt I\&} 
   \\
   dereference  &
   {\tt *it;} & 
   ~ &
   {\tt I::element\_type} 
   \\
   equality comparison  &
   {\tt i == j} & 
   \ &
   {\tt bool} 
   \\
   validity check  &
   {\tt i.IsDone();}\noteref{note-isdone} & 
   ~ &
   {\tt bool} \\
   \hline
   \\
 \end{tabular}

   
 \conceptsubsection{Expression semantics}

\begin{tabularx}{15cm}{RlRRR} \\
  \hline      
  \bf  Name       &
  \bf  Expression &
  \bf  Precondition&
  \bf  Semantics &
  \bf  Postcondition
  \\ 
  \hline
  constructor  &
  {\tt i(a);} &
  {\tt e} is \link{valid}[~(see sec.\ \Ref)~]{valid} &
  set i to  {\tt e}'s first incident element of type {\tt I::element\_type}  & 
  {\tt i.TheAnchor() == a} 
  \\ 
  prefix increment  &
  {\tt ++i;} &
  {\tt ! i.IsDone()} &
  move iterator forward  & 
  {\tt i.IsDone()} or {\tt *i} is a \link{valid}{valid}
  grid element 
  \\ 
  dereference  &
  {\tt e = *it;} & 
  {\tt ! it.IsDone()} &  
  access the element {\tt it} points to &
  {\tt E == (*it);}
  \\ 
  equality comparison  &
  {\tt i == j} & 
  {\tt \&(i.TheGrid()) == \&(j.TheGrid())} &  
  true if i and j reference the same element:  {\tt *i == *j}  &
  ~
  \\ 
  validity check  &
  {\tt i.IsDone();} & 
  ~ &  
  true iff {\tt i} is past-the-end. &
  ~ 
  \\ 
  \hline
  \\
\end{tabularx}

\W\conceptsubsection{Complexity guarantees}

\conceptsubsection{Refinements}
\sectionlink{\concept{Vertex}-On-Cell Iterator}{Vertex-On-CellIterator}
\\
\sectionlink{\concept{Edge}-On-Cell Iterator}{Edge-On-CellIterator}
\\
    
\W\conceptsubsection{Models}

\conceptsubsection{Notes}
\begin{enumerate}
\item 
\notelabel{note-isdone}
For circular sequences, such as the vertices around a cell, there is no natural,
predefined past-the-end value. Therefore it is more natural to let the iterator
itself decide when it is invalid, instead of the standard comparison with a
past-the-end iterator (which is defined as well, in order to be able to use
STL algorithms).
\end{enumerate}

\conceptsubsection{See also}
\sectionlink{\concept{Grid}}{Grid} ~
\sectionlink{\concept{Grid Element}}{GridElement} ~
\sectionlink{\concept{Grid Sequence Iterator}}{GridSequenceIterator} ~


  

