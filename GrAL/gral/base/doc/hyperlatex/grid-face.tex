\xname{GridFace}
\begin{Label}{GridFace}
\conceptsection{Grid Face Concept}
\end{Label}

\conceptsubsection{Description}
A  {\em Grid Face\/} represents the mathematical concept of a Face 
--- a 2-dimensional entity 
in a  \sectionlink{\concept{Grid}}{Grid}.
If the grid is 2-dimensional, a \concept{Grid Face} my
coincide with a \concept{Grid Cell},
in a 3-dimensional grid, it may coincide with a \concept{Grid Facet}.

\conceptsubsection{Refinement of}
\sectionlink{\concept{Grid Element}}{GridElement}
 
\conceptsubsection{Notation}
{\tt F} is a type which is a model of \concept{Grid Face}.
\\
{\tt f} is an object of type {\tt F}
\\
{\tt C} is  a shorthand for the {\tt F::Cell} type
\\
{\tt c} is an object of type {\tt C}
\\
{\tt g} is an object of type {\tt F::grid\_type}
\\
{\tt h} is an object of type {\tt F::face\_handle}
\\
{\tt vf} is an object of type {\tt F::VertexIterator}

\conceptsubsection{Associated types}

{\em NOTE:\/} The types and expression involving 
\sectionlink{\concept{ Incidence Iterators}}{GridIncidenceIterator} are given below for the
case of \sectionlink{vertex-on-face iteration}{Vertex-On-CellIterator}.
Analogous types and expressions can be defined for the other element types.

The tables are to be understood in the following sense:
\\
{\em If\/} a Face defines the incidence iterator over vertices,
{\em then\/} the requirements under {\em Optional part\/} apply.
Analogous requirements take effect if `vertex' is replaced by another element type.

\noindent
\begin{tabularx}{12cm}{llX} \hline
  \bf  Name  &\bf  Expression  &\bf  Description   
  \T \\ \hline
  handle type & 
  {\tt F::face\_handle} &
  type of the corresponding \sectionlink{\concept{Face Handle}}{GridVertexHandle} 
  \\ 
  cell type &
  \code{F::Cell} &
  Cell type of the Face, 
  short for \code{F::grid\_type::Cell}.
  \\ 
  \hline
  \multicolumn{3}{c}{\bf \em Optional part (as example) }
  \\
  \hline
  vertex-on-face iterator &
  {\tt F::VertexIterator}&
  type of the corr. VertexOnFaceIterator
  \T \\   \hline
\end{tabularx}
    
\conceptsubsection{Valid Expressions}

\noindent
\begin{tabularx}{14cm}{RRRR} 
  \T \\ \hline
  \bf  Name  &\bf  Expression  &\bf  Type requirements  & \bf  return type \\ 
  \hline
  handle & 
  {\tt f.handle()} &
  ~ &
  {\tt F::face\_handle} 
  \\
  first cell &
  \code{c = f.C1()} &
  ~ &
  \type{Cell}
  \\
  second cell &
  \code{c = f.C2()} &
  ~ &
  \type{Cell}
  \\
  \hline
  \multicolumn{4}{c}{\bf \em Optional part (as example) }  
  \\
  \hline
  vertex-on-face iteration start & 
  {\tt f.FirstVertex()} &
  ~ &
  {\tt F::VertexIterator} 
  \\
  number of incident vertices & 
  {\tt F.NumOfVertices()} &
  ~ &
  {\tt int} 
  \T \\  \hline  \\
\end{tabularx}
 
\T\begin{small}
\conceptsubsection{Expression semantics}
\begin{tabularx}{15cm}{XXXXX} 
  \T \\ \hline
  \bf  Name     &
  \bf  Expression &
  \bf  Precondition&
  \bf  Semantics &
  \bf  Postcondition
  \\ 
  \hline
  handle &
  {\tt h = f.handle();} &
  f is \footlink{valid}{valid} &
  shorthand for {\tt h = f.TheGrid(). handle(f)} &
  {\tt f == f.TheGrid(). Face(h)}  
  \\ 
  \hline
  \multicolumn{5}{c}{\bf \em Optional part (as example) }
  \\
  \hline
  Vertex-on-Face iteration start & 
  {\tt vf = f.FirstVertex()} &
  {\tt f} is \link{valid}{valid}  &
  let {\tt fi} point to  the first vertex incident to {\tt f}  & 
  {\tt vf.TheEdge() == vf.TheAnchor() == f} 
   and 
  {\tt vf.TheGrid() == f.TheGrid()}
  \\ 
  number of incident vertices & 
  {\tt n =  f.NumOfVertices()} &
  {\tt f} is \link{valid}{valid}  &
  n is the number of vertices that are incident to {\tt f} &
  ~ 
  \T \\ \hline  \\
\end{tabularx}
\T\end{small}
    
\conceptsubsection{Complexity guarantees}
All operations are amortized constant time\noteref{note-face-amort}.

\conceptsubsection{Refinements}

\conceptsubsection{Models}
\sectionlink{\type{Complex2D::Face}}{Complex2D} 
defined in
\gralfilelink{cell2d}{h}{complex2d}
(identical to \type{Complex2D::Cell})
\conceptsubsection{Notes}
\begin{enumerate}
\item \notelabel{note-face-amort}
  Amortization is understood to involve calling the operations for all
  Faces of a grid.
\end{enumerate}

\conceptsubsection{See also}
\sectionlink{\concept{Grid}}{Grid} ~
\sectionlink{\concept{Grid Element Handle}}{GridElementHandle} ~
\sectionlink{\concept{Face Handle}}{GridVertexHandle}~
\sectionlink{\concept{Grid Element }}{GridElement} ~
\sectionlink{\concept{Grid Cell}}{GridCell} ~
\sectionlink{\concept{Sequence Iterator}}{GridSequenceIterator} ~
\sectionlink{\concept{Incidence Iterator}}{GridIncidenceIterator} ~

  

