\xname{GridSequenceIterator}
\begin{Label}{GridSequenceIterator}
\conceptsection{Grid Sequence iterator Concept}
\end{Label}

\conceptsubsection{Description}
A {\em Grid Sequence Iterator\/}
lets a \sectionlink{\concept{Grid Range}}{GridRange}
be seen as a sequence of the iterators element type.

\conceptsubsection{Refinement of}
STL \stllink{Forward Iterator}{ForwardIterator}
\noteref{note-forward}
\\
\sectionlink{\concept{Grid Entity}}{GridEntity}


\conceptsubsection{Notation}
{\tt I} is a model of sequence iterator
\\ 
{\tt i, j} are  objects of type  {\tt I}

\conceptsubsection{Associated types}
\begin{tabularx}{12cm}{llR} \\
  \hline
  \bf  Name  &\bf  Expression  &\bf  Description  \\ 
  \hline
  grid type  & 
  {\tt I::grid\_type} &
  type of the  underlying grid, 
  model of  \sectionlink{\concept{Grid}}{Grid}
  \\ 
  anchor type  & 
  {\tt I::anchor\_type} &
  type of the underlying grid range,
   model of \sectionlink{\concept{Grid Range}}{GridRange}
  \\ 
  element type  & 
  {\tt I::element\_type} &
  type of the underlying element, 
  model of \sectionlink{\concept{Grid Element}}{GridElement}
  \\ 
  value type  &
  {\tt I::value\_type} &
  synomym for {\tt I::element\_type}   
  \\ 
  \hline
  \\
\end{tabularx}
 

\conceptsubsection{Valid Expressions}
\begin{tabular}{llll} \\
  \hline
  \bf  Name  &\bf  Expression  &\bf  Type requirements  & \bf  return type  \\ 
  \hline
  prefix increment  &
  {\tt ++it;} &
  ~ &
  {\tt I\&} 
  \\ 
  dereference  &
  {\tt *it;} & 
  ~ &
  {\tt I::element\_type} 
  \\
  equality comparison  &
  {\tt i == j} & 
  ~ &
  {\tt bool} 
  \\ 
  validity check  &
  {\tt i.IsDone();} & 
  ~ &
  {\tt bool} 
  \\ 
  \hline
  \\
\end{tabular}

   
\conceptsubsection{Expression semantics}

\begin{tabularx}{15cm}{RlRRR} \\
  \hline
  \bf  Name     &
  \bf  Expression &
  \bf  Precondition&
  \bf  Semantics &
  \bf  Postcondition
  \\ 
  \hline
  prefix increment  &
  {\tt ++i;} &
  {\tt ! i.IsDone()} &
  move iterator forward  & 
  {\tt i.IsDone()} or {\tt *i} is a valid grid element 
  \\ 
  dereference  &
  {\tt e = *it;} & 
  {\tt ! it.IsDone()} &  
  access the element {\tt it} points to &
  {\tt E == (*it);} 
  \\ 
  equality comparison  &
  {\tt i == j} & 
  {\tt \&(i.TheGrid()) == \&(j.TheGrid())} &  
  true if i and j reference the same element:  {\tt *i == *j}  &
  ~ 
  \\ 
  validity check  &
  {\tt i.IsDone();} & 
  ~ &  
  true iff {\tt i} is past-the-end. &
  ~ 
  \\ 
  \hline
  \\
\end{tabularx}

\W\conceptsubsection{Complexity guarantees}

\conceptsubsection{Refinements}
\sectionlink{\concept{Grid Vertex Iterator}}{GridVertexIterator}
\\
\sectionlink{\concept{Grid Edge Iterator}}{GridVertexIterator}
    
\W\conceptsubsection{Models}

\conceptsubsection{Notes}
\begin{enumerate}
\item 
  \notelabel{note-forward}
  For the existing models, this is currently only partially implemented: postfix increment is missing
  (it is inefficient because of the temporary object involved), and
  \xlink{{\tt iterator\_traits<>}}{\STLPATH{iterator_traits}} 
  is not (yet) specialized.
  However, the main property of Forward Iterator, namely support of multiple passes,
  is satisfied.
\end{enumerate}

\conceptsubsection{See also}
\sectionlink{\concept{Grid}}{Grid} ~
\sectionlink{\concept{Grid Element}}{GridElement} ~
\sectionlink{\concept{Grid Incidence Iterator}}{GridIncidenceIterator} 

  

