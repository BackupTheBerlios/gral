\xname{Complex2D}
\begin{Label}{Complex2D}
\typesection{{\tt Complex2D} Grid Type}
\end{Label}

\needswork{This documentation is still incomplete!}

\typesubsection{Description}
The type {\tt Complex2D} is a model of \sectionlink{\concept{Grid}}{Grid}.
It allows representation of  general regular $2$-dimensional finite CW-complexes.
That is, the cells are combinatorially equivalent to  simple polygons.

\typesubsection{Example}

\begin{example}
Complex2D G;         // empty grid
\sectionlinkfoot{RegGrid2D}{RegGrid2D} R(10,10);  // 10x10 Cartesian grid
\sectionlinkfoot{ConstructGrid0}{ConstructGrid}(G,R); // copy R to G;
assert(G.NumOfVertices() == R.NumOfVertices());
assert(G.NumOfCells()    == R.NumOfCells());
typedef grid\_types<Complex2D> gt; // 'namespace' for Complex2D-related types 
for(gt::CellIterator c(G), ! c.IsDone(); ++c)
  cout << "Cell " << c.handle() << "  "
       << "has "  << (*c).NumOfVertices() << " vertices\\n";
\end{example}

\typesubsection{Definition}

Defined in \gralfilelink{complex2d}{h}{complex2d}.
  
\typesubsection{Model of}
\sectionlink{\concept{Grid}}{Grid}
\\
\sectionlink{\concept{Vertex Grid Range}}{VertexGridRange}
\\
\sectionlink{\concept{Edge Grid Range}}{VertexGridRange}
\\
\sectionlink{\concept{Facet Grid Range}}{VertexGridRange}
\\
\sectionlink{\concept{Cell Grid Range}}{VertexGridRange}

\typesubsection{Members}
\begin{tabularx}{15cm}{XXX} \hline
  \bf  Member   &
  \bf  Where defined &
  \bf  Description 
  \\ \hline
  \multicolumn{3}{c}{{\bf \em Types}}  \\ \hline
  \multicolumn{3}{c}{{\em handle types\/}}  \\ \hline
  \type{vertex\_handle} &
  \sectionlinkshort{\concept{Vertex Grid Range}}{VertexGridRange} &
  \sectionlinkshort{ handle }{GridElementHandle}   type
  corr. to \type{Vertex} 
  \\
  \type{edge\_handle} &
  \sectionlinkshort{\concept{Edge Grid Range}}{VertexGridRange} &
  \sectionlinkshort{handle}{GridElementHandle}   type
  corr.\ to \type{Edge}
  \\
  \type{facet\_handle} &
  \sectionlinkshort{\concept{Facet Grid Range}}{VertexGridRange} &
  \sectionlinkshort{ handle }{GridElementHandle}   type
  corr.\ to \type{Facet}
  \\
  \type{cell\_handle} &
  \sectionlinkshort{\concept{Cell Grid Range}}{VertexGridRange} &
  \sectionlinkshort{handle}{GridElementHandle}   type
  corr.\ to \type{Cell}
  \\ \hline
  \multicolumn{3}{c}{{\em element types\/}} \\ \hline
  \sectionlinkUNDEF{\type{Vertex}}{Complex2DVertex} &
  \sectionlinkshort{\concept{Vertex Grid Range}}{VertexGridRange}&
  The \sectionlinkshort{\concept{Vertex}}{GridVertex} element type 
  \\
  \sectionlinkUNDEF{\type{Edge}}{Complex2DEdge} &
  \sectionlinkshort{\concept{Edge Grid Range}}{VertexGridRange}&
  The \sectionlinkshort{\concept{Edge}}{GridEdge} element type 
  \\
  \sectionlinkUNDEF{\type{Facet}}{Complex2DFacet} &
  \sectionlinkshort{\concept{Facet Grid Range}}{VertexGridRange}&
  The \sectionlinkshort{\concept{Facet}}{GridFacet} element type 
  \\
  \sectionlinkUNDEF{\type{Cell}}{Complex2DCell} &
  \sectionlinkshort{\concept{Cell Grid Range}}{VertexGridRange} &
  The \sectionlinkshort{\concept{Cell}}{GridCell} element type 
  \\ \hline
  \multicolumn{3}{c}{{\em sequence iterator types\/}} \\ \hline
  \type{VertexIterator} &
  \sectionlinkshort{\concept{Vertex Grid Range}}{VertexGridRange} &
  The  \sectionlinkshort{\concept{ Sequence Iterator}}{GridSequenceIterator}   
  type for \type{Vertex}
  \\
  \type{EdgeIterator} &
  \sectionlinkshort{\concept{Edge Grid Range}}{VertexGridRange}&
  The  \sectionlinkshort{\concept{ Sequence Iterator}}{GridSequenceIterator}   
  type for \type{Edge}
  \\
  \type{FacetIterator} &
  \sectionlinkshort{\concept{Facet Grid Range}}{VertexGridRange}&
  The  \sectionlinkshort{\concept{ Sequence Iterator}}{GridSequenceIterator}   
  type for \type{Facet}
  \\
  \type{CellIterator} &
  \sectionlinkshort{\concept{Cell Grid Range}}{VertexGridRange}&
  The  \sectionlinkshort{\concept{ Sequence Iterator}}{GridSequenceIterator}   
  type for \type{Cell}
  \\ 
  \hline
  \multicolumn{3}{c}{{\bf \em Functions\/}} \\ 
  \hline
  \multicolumn{3}{c}{{\em sequence iteration\/}}  \\ 
  \hline
  \type{VertexIterator} \par \type{FirstVertex()} &
  \sectionlinkshort{\concept{Vertex Grid Range}}{VertexGridRange} &
  Iterator pointing to the first vertex 
  \\ 
  \multicolumn{3}{c}{ ---  {\em same for\/} \type{Edge}, \type{Facet}, \type{Cell} {\em types\/} --- } \\ \hline
  \multicolumn{3}{c}{{\em sequence sizes\/}}  \\ \hline
  \type{int NumOfVertices()} &
  \sectionlinkshort{\concept{Vertex Grid Range}}{VertexGridRange} &
  number of vertices 
  \\
  \multicolumn{3}{c}{ --- {\em same for\/} \type{Edge}, \type{Facet}, \type{Cell} {\em types\/} --- } \\ \hline
\end{tabularx}

\typesubsection{See also}
\gralclasslink{Triang2D}{triang2d} ~
\gralclasslink{RegGrid2D}{cartesian2d}

    
  

