\documentclass{report}
%\documentclass{article}
\usepackage[compact]{titlesec}
\usepackage{hyperlatex}
\usepackage{tabularx}
\usepackage{xspace}
\usepackage[dvips]{graphicx}

%\W\usepackage[simple]{frames}
\W\usepackage{math}

\htmltitle{Documentation of the Grid Category Components}
\title{Documentation of the Grid Category Components}
\author{Guntram Berti}
\htmladdress{\xlink{Guntram Berti}{mailto:berti@math.tu-cottbus.de}}
\htmlattributes{TABLE}{BORDER}
\htmldepth{5}
\htmldirectory{html}

\T\newcommand{\usc}{\_}
%\W\newcommand{\usc}{_}
%\setcounter{secnumdepth}{6}

\newcommand{\htmlimgpath}{bilder/}
\newcommand{\htmlimgsuffix}{.jpeg}
\newcommand{\makepath}[3]{{#1}#2#3}
\newcommand{\img}[1]{\makepath{\htmlimgpath}{#1}{\htmlimgsuffix}}

% Pfade
\newcommand{\NMWRROOT}{../}
\newcommand{\NMWRINC}[1]{\NMWRROOT/include/#1}
\newcommand{\STLURL}{http://www.sgi.com/Technology/STL}
\newcommand{\STLPATH}[1]{\STLURL/#1}
\newcommand{\traitsURL}{}

%\T\parindent0cm
%\T\parskip0.25cm

\newcommand{\htmlfill}{\ifhtml{%
    \begin{rawhtml}
      <BR> <BR> <BR> <BR> <BR> <BR> <BR> <BR> <BR> <BR> <BR> <BR>
      <BR> <BR> <BR> <BR> <BR> <BR> <BR> <BR> <BR> <BR> <BR> <BR>
      <BR> <BR> <BR> <BR> <BR> <BR> <BR> <BR> <BR> <BR> <BR> <BR>
    \end{rawhtml}}}

\T\newcommand{\nix}{\rule{0cm}{0cm}}
\W\newcommand{\nix}{}
\W\newcommand{\smallskip}{}

%%%%%%%%%%%%%%%%%%%%%%%%%%
%%%    Referenzen     %%%%
%%%%%%%%%%%%%%%%%%%%%%%%%%

\newcounter{notecounter}[section]
%\newcommand{\notelabel}[1]{\refstepcounter{notecounter}{\label{#1}}}
\newcommand{\notelabel}[1]{{\label{#1}}}
%\newcommand{\noteref}[1]{\link*{\arabic{\value}{notecounter}}[{(see note \arabic{\value{notecounter}})}]{#1}}

\newcommand{\noteref}[1]{\link*{\htmlref{#1}}[~{\small(see note \ref{#1})}]{#1}}
%\newcommand{\noteref}[1]{\link*{\ref{#1}}[{\footnote{see note \ref{#1}}}]{#1}}

\newcommand{\sectionlink}[2]{\xlink{#1}[~(section \ref{#2})~]{#2.html}}
\newcommand{\sectionlinkfoot}[2]{\xlink{#1}[\footnote{see section \ref{#2}}]{#2.html}}
\newcommand{\sectionlinkweb}[2]{\xlink{#1}{#2.html}}
\newcommand{\sectionlinkshort}[2]{%
 \xlink{#1}[{{\small\xspace{} $\rightarrow$ S \ref{#2}}}]{#2.html}}
\newcommand{\Sectionlink}[1]{\sectionlink{#1}{#1}}
\newcommand{\Sectionlinkfoot}[1]{\sectionlinkfoot{#1}{#1}}
\newcommand{\Sectionlinkshort}[1]{\sectionlinkshort{#1}{#1}}

\newcommand{\footlink}[2]{\link{#1}[\footnote{see \ref{#2}}]{#2}}

\newcommand{\glossarylink}[2]{%
 \link{#1}[~($\rightarrow$ p.\ \pageref{glossary-#2})~]{glossary-#2}}
\newcommand{\Glossarylink}[1]{\glossarylink{#1}{#1}}
\newcommand{\glossarylinkfoot}[2]{%
  \link{#1}[\footnote{see page  \pageref{glossary-#2}}]{glossary-#2}}
\newcommand{\Glossarylinkfoot}[1]{\glossarylinkfoot{#1}}

\newcommand{\nmwrcodelink}[2]{\xlink{{\tt #1}}{\NMWRINC{#2}/#1}}
\newcommand{\stllink}[2]{\xlink{\concept{#1}}{\STLURL/#2.html}}
\newcommand{\stltypelink}[2]{\xlink{\type{#1}}{\STLURL/#2.html}}
\newcommand{\Stllink}[1]{\stllink{#1}{#1}}
\newcommand{\Stltypelink}[1]{\stltypelink{#1}{#1}}
\newcommand{\conceptlink}[2]{\sectionlink{\concept{#1}}{#2}}
\newcommand{\Conceptlink}[1]{\conceptlink{#1}{#1}}
\newcommand{\conceptlinkfoot}[2]{\sectionlinkfoot{\concept{#1}}{#2}}
\newcommand{\Conceptlinkfoot}[1]{\conceptlinkfoot{#1}{#1}}

\newcommand{\webref}[3]{\xlink{#1}[~\cite{#3}~]{#2}}

%\newcommand{\htmlcolor}[2]{%
%  \begin{rawhtml}<FONT COLOR="#\end{rawhtml}#1
%  \begin{rawhtml}> \end{rawthml} #2 \begin{rawhtml} </FONT>  \end{rawthml}}

\newcommand{\htmlcolor}[2]{#2}

\newcommand{\undefcb}{\begin{rawhtml}<FONT COLOR="008F00">\end{rawhtml}}
\newcommand{\undefce}{\begin{rawhtml}</FONT>\end{rawhtml}}

\newcommand{\sectionlinkUNDEF}[2]{\texorhtml{#1}{\xlink{#1}{not-yet-done.html}}}
%\newcommand{\xlinkUNDEF}[2]{\texorhtml{#1}{\xlink{#1}{not-yet-done.html}}}

\newcommand{\xlinkUNDEF}[2]{\texorhtml{#1}{\xlink{\undefcb{}#1\undefce}{not-yet-done.html}}}

\newcommand{\bigsep}{%
\T\bigskip
\W\htmlrule[width=70% align=center]
}

 
\W\begin{iftex}
\newenvironment{Glossar}{%
  \begin{list}{}{%
    \renewcommand{\makelabel}[1]{{\bf ##1} \label{glossary-##1}}  %
   }
}{\end{list}}
\W\end{iftex}

\begin{ifhtml}
\newenvironment{Glossar}{\HlxBlk\html{DL}\begingroup
  \newcommand{\item}[1][]{ \label{glossary-##1} \HlxBlk\html{DT}\textbf{##1}\html{DD}}%
  \ignorespaces}{\endgroup\HlxBlk\html{/DL}}
\end{ifhtml}


% Tabellen
\T\newcolumntype{R}{>{\raggedright\arraybackslash}X}
\W\newcolumntype{R}{X}

\renewcommand{\arraystretch}{1.3}
\T\setlength{\extrarowheight}{1pt}

% Gliederung
\newcommand{\RelativeSection}[1]{\subsection{#1}}
\newcommand{\RelativeSubsection}[1]{\subsubsection{#1}}
%\renewcommand{\RelativeSubsection}[1]{\paragraph*{#1} \nix  \par\noindent }
\T\titlespacing*{\paragraph}{0pt}{*2}{*1}
\T\titleformat{\paragraph}[hang]{\bf\sl }{}{0pt}{}{}
\renewcommand{\RelativeSubsection}[1]{\paragraph{#1}}

\newcommand{\typesection}[1]{\RelativeSection{#1}}
\newcommand{\typesubsection}[1]{\RelativeSubsection{#1}}
\newcommand{\conceptsection}[1]{\RelativeSection{#1}}
\newcommand{\conceptsubsection}[1]{\RelativeSubsection{#1}}
\T\newcommand{\algosection}[1]{\RelativeSection{{\sc #1} Algorithm}}
\W\newcommand{\algosection}[1]{\RelativeSection{{\tt #1} Algorithm}}
\newcommand{\algosubsection}[1]{\RelativeSubsection{#1}}
\newcommand{\datasection}[1]{\RelativeSection{{\tt #1} Type}}
\newcommand{\datasubsection}[1]{\RelativeSubsection{#1}}
\newcommand{\itersection}[1]{\RelativeSection{{\tt #1} Type}}
\newcommand{\itersubsection}[1]{\RelativeSubsection{#1}}


\newcommand{\introsection}[1]{\section{#1}}
\newcommand{\introsubsection}[1]{\par \noindent{\bf\large #1} \par\noindent}

\newcommand{\xchapter}[2]{\xname{#2}\begin{Label}{#2} \chapter{#1} \end{Label}}

% Fonts
\newcommand{\type}[1]{{\tt #1}}
\newcommand{\pcode}[1]{{\tt #1}}
\newcommand{\concept}[1]{\texorhtml{{\sf #1}}{{\em{}#1}}}
\newcommand{\function}[1]{{\tt #1}}
\newcommand{\variable}[1]{{\tt #1}}
\newcommand{\Var}[1]{{\tt #1}}
\newcommand{\templateparam}[1]{{\tt #1}}

% images
\newcommand{\Image}[1]{%
 \texorhtml{\includegraphics{#1.eps}}{\htmlimage{#1.gif}}}

% Status
\newcommand{\needswork}[1]{ {\begin{center}{\LARGE #1} \end{center}}}
\newcommand{\Needswork}[1]{ Sorry, this page is still incomplete!}

\begin{document}

%\W\maketitle
\W\htmlmenu{4}


\chapter{Introduction}

\xname{GridIntro}
\begin{Label}{GridIntro}
\introsection{Overview over the Grid Category}
\end{Label} 

  In the \sectionlinkshort{preceding section}{GridsAlgorithmsAndReuse},
  we have seen that the major obstacle for reusing grid-related software
  is direct access to low-level representation details,
  together with to inavoidable high variability of such details
  in the field of geometric data structures.

  In order to  decouple implementations of algorithms 
  (or other data structures)
  from these details, we will develop in the following 
  a uniform, abstract interface layer for grids.

  How can one obtain such an interface? 
  Let us look again at the simple 
  \xlink{algorithm for surface calculation}{GridsAlgorithmsAndReuse.html\#algo-surface}:

   \begin{figure}[h]
     \begin{center}
       \begin{Label}{algo-surface-2}
       \T\includegraphics{bilder/algo-surface}
       \W\htmlimage[ALT="A simple algorithm calculating cell surfaces"]{%
          \img{algo-surface}}
        \end{Label}
     \end{center}
   \end{figure}

   What kind of functionality does this algorithm need?
   \begin{enumerate}
   \item \label{iter-cell} Iteration over all cells of a grid
   \item \label{iter-facet-on-cell} Iteration over all facets of a cell
   \item \label{gf-real-on-cell} Storage of real numbers on cells
   \item \label{geom-volume} Calculation of volumes of  facets 
     (i.\ e.\ area in 3D, length in 2D)
   \end{enumerate}
  
   Items \ref{iter-cell} and \ref{iter-facet-on-cell} 
   can be summarized as iteration or combinatorial functionality.
   Item \ref{gf-real-on-cell} obeys to the general pattern of storing
   data on (or associating data to) grid elements.
   And item \ref{geom-volume} attaches geometric meaning to the combinatorial
   entitiy ``facet''.

   Thus, in this case, functionality needed by  the algorithm falls
   into three classes. This, however, is no accident due to the simplicity
   of the algorithm, but a pattern found also when investigating 
   more complex algorithms.
  To sum up, we distinguish the following three fundamental layers 
  of functionality:
  \begin{itemize}
  \item  the \sectionlink{combinatorial layer}{CombinatorialLayer}
  \item the \sectionlink{data association layer}{DataAssociationLayer}
  \item the \sectionlink{geometrical layer}{GeometricalLayer} 
  \end{itemize}
  
  The {\em combinatorial layer\/} views a grid 
  simply as a \glossarylinkemph{lattice},
  or, more generally, a \glossarylinkemph{poset},
  given by the {\em incidence\/} relationship.
  The entities involved are of  a purely discrete nature;
  there is not notion of coordinates of vertices or the like.
  The notion of dimension used here is a combinatorial or {\em intrinsic\/}
  dimension.
  
  The {\em grid function\/} layer adds the possibility to associate data 
  with the discrete grid elements, 
  for example to put marks on vertices,
  or to store solution values on cells.
  Mathematically, it captures the notion of a function from
  the discrete set of grid elements of any fixed dimension 
  to entities of some given type, for example reals or integers.

  
  The {\em geometrical layer\/} provides an embedding into a concrete geometrical space of 
  an ({\em external\/}) dimension which is at least the intrinsic dimension. 
  For example, a grid of (intrinsic) dimension 2 could be embedded in 3-space,
  like the surface of a torus.
  It corresponds to the notion of a \glossarylinkemph{geometric realization}
  of a combinatorial complex.

  
  We will discuss these layers in an informal way 
  in the introductory sections on the
  \sectionlink{combinatorial}{CombinatorialLayer}, 
 \sectionlink{grid function}{DataAssociationLayer}
 and \sectionlink{geometrical}{GeometricalLayer} layers, respectively.

 A more formal description of the
 \glossarylink{concepts}{concept} involved starts in 
 \sectionlink{the concept section.}{Concepts}.
 Here C++ syntax is prescribed, in a way analogous to
 the STL documentation.
 We will refer to them with a trailing (C).
 This part is of cours crucial for actually implementing generic components;
 it is not so important for understanding the bib picture
 (because it contains inevitably many arbitrary decisions).

 In a short lookahead, here is the generic version of the 
 algorithm  \xlink{presented before}{GridIntro.html\#algo-surface-2}:

 \begin{example}
\conceptlinkfoot{grid_function}{TotalGridFunction}<\Conceptlinkfoot{Cell},double> surface(Grid);
for(\Conceptlinkfoot{CellIterator} c(Grid); !c.IsDone(); ++c) {
  surface[*c] = 0.0; 
  for(\Conceptlinkfoot{FacetOnCellIterator} fc(c); !fc.IsDone(); ++fc)
    surface[*c] += \conceptlinkfoot{Geometry}{VolumeGridGeometry}.volume(*fc);
}
 \end{example}

 In the \sectionlinkshort{third part}{Components} of the documentation,
 concrete \sectionlink{components}{Components}
 are described, sorted by functionality 
 (algorithms, grid ranges, iterators and so on).
 These components separate into two layers, 
 depending on whether they are  \emph{leaf components\/}
 (for example basic grid implementations),
 or \emph{generic extensions}, that is, 
 classes which provide additional functionality 
 for each grid component that is a \Glossarylink{model} of a certain \Glossarylink{concept}.
 In fact, most components belong to this second layer,
 which evidently greatly enhances reuse.
  
  \bigsep 

  In the present framework, a mathematical grid is not represented by a {\em single\/}
  entity (e.g. a class), 
  rather, {\em different\/} classes cooperate to this aim.
  This is already visible by the dissection into the three fundamental categories,
  namely combinatorics, data association and geometry.
  
  Moreover, each of these categories introduces many sub-categories 
  and refined concepts, 
  especially the combinatorial one.
  In order to understand how grids are represented by these concepts,
  it is important to understand their mutual relationship.

  For each concrete grid type, there is one class implementing
  the abstract \sectionlink{grid concept}{Grid}.
  {\small (In the STL parlance, it is \glossarylink{{\em models\/}}{model} of the concepts.)}
  To actually use this grid type, one has to access a number of
  different classes accounting for a representation 
  of this mental "The Grid" thing one may have in mind.
  Thus, there are seperate classes for 
  \link{grid elements}{element-intro} (vertices, edges, cells),
  \sectionlink{iterators}{GridSequenceIterator},
  and \sectionlink{grid functions}{DataAssociationLayer}.
  The concrete grid types act as a sort of {\em central entry point\/} 
  giving access to these {\sl associated types\/},
  which one can think of  as  forming a sort of `halo'
  around the central grid type.

  The number of supported types  will not be the same for all grids.
  The basic \conceptlink{Grid}{Grid} concepts requires any types at all
  --- everything useful comes in through refinements of this concept.
  A grid even does not need to define all possible element types ---
  there are grid types used only for I/O 
  (see \sectionlink{\type{IstreamComplex2DFmt}}{istream-complex2d-fmt})
  which do with a quite
  minimalistic set of associated types.
  Most profoundly, grids differ in the amount of
  \link{incidence iteration}[~(see section \Ref)~]{incidence-it-intro}
  they can support --- a trade-off between storage consumption and functionality.


  At first reading, 
  one can consider this type association as fixed.
  But in principle, it is possible to use different `halo' types 
  with one and the same grid type.
  This technique allows a fine control of grid functionality
  which is not attainable with a monolithic approach.

  \label{intro-grid-types}
  An important  technical question is now  
  how to access `halo types' in a consistent and uniform manner. 

  The way chosen here is to collect all non-generic grid-related types into
  a dedicated entity called {\tt grid\_types},
   associated to a concrete grid type.
  (With `non-generic', we mean types that have to be specialized for each
  concrete grid type.) 
  In practice, this is just a class template that has to be
  \glossarylink{specialized}{partial specialization} for concrete grids.

  This mechanism provides algorithms with a single entry point to 
  type information (and perhaps other) related to a grid parameter.
  It also offers a clean possibility to pass a different `halo',
  that is, a customized set of types which behave differently.
  For example, one could use special iterators for doing 
  limited algorithm animation in an automatic way.
  Currently, only few algorithms are actually parameterized by 
  grid types, an example is the \type{CGT} parameter of 
  \sectionlink{CalculateNeighborCells}{cell-neighbor-search}.
  
  We may contrast this  with the corresponding mechanisms for STL containers:
  These, too, are associated with some service types, namely the 
  iterator type(s), which are accessed by means of nested typedefs:
  \begin{example}
    Container C;
    typedef Container::iterator iter_type;
    iter_type i = C.begin();
  \end{example}
  In our approach, an analogous action would look like the following:
  \begin{example}
    MyGrid g;
    typedef grid_types<MyGrid> gt; // default set of related types
    gt::VertexIterator vi = g.FirstVertex();
  \end{example}
  A {\em quantitative\/} difference to the STL is, that in the grid category,
  many more types are necessary. 
  The total functionality making up a grid is also more broadly
  distributed among these types.

  An important {\em qualitative\/} difference is the parameterization of algorithms.
  STL algorithms are never parameterized by containers but always by iterators
  forming a logical range {\tt [begin, end)}. This captures well the only aspect of 
  containers the algorithms are interested in, namely that they represent linear 
  sequences.
  In contrast, algorithms for grids are parameterized by grids, and often also
  by grid functions or grid geometries, which correspond  in some sense
  to the STL containers. 
  This is due to the fact that grids in this case act as placeholders
  for the underlying mathematical structures on top of which the algorithm
  performs its tasks.

  The set of types used by these algorithms is rich, corresponding to
  the richer structures of the domain. Moreover, these types are not
  independent, by closely relate to each other.
  Therefore, the responsibility for bundling the right types together
  has to be concentrated somewhere.
  This is in essence what the {\tt grid\_types} mechanism does:
  It maps grid types to a set of related types.
  If one want to control the way  algorithms access grid-related types,
  one has to parameterize them with such mappings, 
  in addition to their regular parameters.
 

  


  \xname{CombinatorialLayer}
  \begin{Label}{CombinatorialLayer}
    \introsection{The combinatorial grid layer}
  \end{Label}
  Central to the notion of a grid are the combinatorial relationships
  among its entities of different dimension, namely, their incidence relation.

  The combinatorial grid layer consists of 
  (combinatorial) {\em Grids\/} \sectionlink{(C)}{Grid}
  as comprehensive entities,
  {\em grid elements\/} \sectionlink{(C)}{GridElement},
  like Vertex, Edge, or Cell, which are its building blocks,
  {\em element handles\/} \sectionlink{(C)}{GridElementHandle}
  (minimal representations or indices of elements),
  {\em sequences iterators\/}  \sectionlink{(C)}{GridSequenceIterator}
  that allow to iterate over all elements of a given type,
  and {\em incidence iterators\/} \sectionlink{(C)}{GridIncidenceIterator}
  allowing to iterate over all elements of some fixed dimension
  incident to a given element.

  \begin{center}
  \begin{figure}[h]
    \T\begin{minipage}{6cm}
        \T\includegraphics[width=5cm]{bilder/cellcx-2triang.eps} 
        \W\htmlimage[ALT="a grid consisting of 2 adjacent triangles (2 cells, 5 edges, 4 vertices)"]{\img{cellcx-2triang}}
        \caption{A simple grid \ldots}
      \T\end{minipage}  \T\hspace{1cm}
    \T\begin{minipage}{6cm}
        \T\includegraphics[width=5cm]{bilder/cellcx-2triang-lattice.eps} 
        \W\htmlimage[ALT="the lattice of this grid"]{\img{cellcx-2triang-lattice}}
        \caption{\ldots and its lattice}
      \T\end{minipage}
    \end{figure}
  \end{center}

  For example, in the images, the grid consists of the following elements:
  vertices $v_1, \ldots, v_4$, edges $e_1, \ldots e_5$ and cells $c_1, c_2$.
  Their incidence relation is given by the lattice, where two different elements
  $x, y$
  are incident, if there is a path from  $x$ down to $y$  (the path is not allowed to
  visit the same level more than once). Therefore, elements of the same dimension
  are never incident, only adjacent. For example, $c_1$ is incident to $v_3$,
  and adjacent to $c_2$.

  These concepts give rise to a lot of different concrete types which are associated with a
  concrete grid type.
  Because this association is so fundamental and important,
  we use the special  {\em grid-types\/} entity to represent it.
  This is a class template which is specialized for concrete grid types,
  and thus acts as a mapping from  concrete grid types to the associated types introduced below.
  \xname{element-intro}
  \begin{Label}{element-intro}
  \introsubsection{Elements}
  \end{Label}

  Basically, a grid consists of a collection of {\em vertices\/}, {\em edges\/}
  etc. which are termed {\em k-Elements\/} (other common
  names are {\em k-Faces\/} or {\em k-cells\/}). 
  \\
  The notion is captured by the following types, named according to their dimension and their
  co-dimension.

      
  \begin{tabular}{lcc} \\ \hline
    \bf  k-Element &  \bf  Dimension &  \bf  Codimension  \\ \hline
    {\tt Vertex} &   0 &   -  \\
    {\tt Edge} &   1 &   -  \\ 
    {\tt Face} &   2 &   -  \\
    {\tt Facet} &   - &   1  \\
    {\tt Cell} &   - &   0  \\ \hline \\
  \end{tabular}
  
  Other common names are {\em node\/} (for vertex) and {\em volume\/} (for cell).
  These are not used here.
        
  The above naming scheme allows for a dimension-independent
  formulation of many algorithms: e.g. fluxes are always defined on {\tt Facets}
  (see \link{example}[~on page \pageref{flux-example}~]{flux-example}).
      
  In general, some of these types can coincide: for a 2D-grid, the types {\tt Edge}
  and {\tt Facet} are often the same. 
  Also,  the type {\tt Face} cannot be defined for 1D-grids.

  \begin{Label}{handle-intro}
    \introsubsection{Element handles}
  \end{Label}
   An element  implementation must be a class, 
   because there are some member functions
   required \sectionlink{(C)}{GridElement}. 
   A corresponding object must contain a reference to the underlying grid,
   which is redundant when considering sets of elements belonging to the same grid.
   For a minimal representation of elements, often just a number (a {\em element handle\/})
   is required, which allows unambiguous reconstruction of an element belonging to a given grid.
   
   Element handles can be used to represent sets of elements in an economic way,
   or to implement mappings between different grids (grid morphisms).

   \begin{Label}{sequence-it-intro}
     \introsubsection{Sequence iterators}
   \end{Label}
   At a basic level, a grid can be seen as a container of its elements, that is,
   as a sequence of vertices, edges and so on. Correspondingly, a grid defines
   sequence iterators \sectionlink{(C)}{GridSequenceIterator}
   to capture this property.

   These iterators satisfy the STL Forward Iterator requirements.

   \begin{figure}[h]
     \begin{center}
       \T\includegraphics{bilder/cell-it.eps}
       \W\htmlimage[ALT="A cell iterator hopping from cell to cell"]{\img{cell-it}}
       \caption{A cell iterator}
     \end{center}
   \end{figure}

   {\bf Examples:} 
   A loop over all vertices of a grid {\tt G} would read:
   \begin{example}
       my_grid_type g;
       // ...
       int nv = 0;
       for(\link{VertexIterator}{GridVertexIterator.html} v = g.FirstVertex(); ! v.IsDone(); ++v)
         nv++;
       ASSERT( nv == \link{g.NumOfVertices()}{VertexGridRange.html} );
\end{example}
      To draw the line-graph of a grid, we would do:

\begin{example}
       my_geometry geom(g);
       // ...
       for(\link{EdgeIterator}{EdgeRange.html} e = G.FirstEdge(); ! e.IsDone(); ++e)
         draw(geom.\xlinkUNDEF{segment}{GeomSegment.html}(*e));
\end{example}

\begin{Label}{incidence-it-intro}
    \introsubsection{Incidence iterators}
    \end{Label}
    Grids are more than just a collection of vertices, edges and so
    on. We want to be able to navigate through the neighbourhood of an k-element.
    For this purpose, we need {\em incidence iterators\/}. 
    For example we might want
    to access all vertices incident to a cell.

   \begin{figure}[h]
     \begin{center}
       \T\includegraphics{bilder/cell-on-cell.eps}
       \W\htmlimage[ALT="A cell-on-cell iterator loops through the 3 adjacent triangles of a triangle"]{%
        \img{cell-on-cell}}
      \caption{A cell-on-cell iterator}
     \end{center}
   \end{figure}

    In order to iterate 
    over the incident elements of a {\tt Cell}, 
    we can implement (and use) the following iterators:

    \begin{tabular}{ll} 
      \\
      {\tt VertexOnCellIterator} &  {\tt Cell::FirstVertex()}  \\ 
      {\tt EdgeOnCellIterator} &   {\tt Cell::FirstEdge()} \\ 
      {\tt FacetOnCellIterator} &  {\tt Cell::FirstFacet()}  \\
      {\tt CellOnCellIterator} &  {\tt Cell::FirstCell()} \\ 
      \\
    \end{tabular}

    The corresponding iterators for vertices have analogous names.
    Strictly speaking {\tt CellOnCellIterator}  is not an 
    incidence-iterator, because two
    neighboring cells are not incident.
    A typical example might be to calculate flows over cell boundaries.
    In pseudo-code, this reads:
    \begin{example}
     for all cells c of the grid g
        initialize the flux to 0
        for all facets fc  of the cell c
          add the flux over fc to the flux of c
    \end{example}
     
   Using our concepts, this translates into the following:
 
    \begin{Label}{flux-example}
 
  \begin{example}
 a_grid_type g;
 typedef \xlinkUNDEF{grid\_types}{grid_types.html}<a_grid_type> gt;
 // ....
 \sectionlink{grid\_function}{DataAssociationLayer}<\sectionlink{Cell}{GridCell},flux\_type> fluxes(g);
 for(gt::\sectionlink{CellIterator}{GridVertexIterator} c = g.FirstCell(); ! c.IsDone(); ++c) \{
   fluxes[*c]= flux\_type(0.0);
   for(gt::\sectionlink{FacetOnCellIterator}{VertexOnCellIterator} f = (*c).FirstFacet(); ! f.IsDone(); ++f)
     fluxes[*c] += calculate\_flow(f);
   \}
\end{example}
\end{Label}

Here {\tt flux\_type} depends on the application and typically is a vector
of low dimension.


\xname{DataAssociationLayer}
\begin{Label}{DataAssociationLayer}
\introsection{Data associations (grid functions)}
\end{Label}

      In any serious application, we need to deal with values attached to k-elements:
      for example, state variables living on vertices in a typical FEM-program,
      costs on edges in network-flow algorithms or state variables on cells
      in a cell-centered Finite-Volume-code.
      There are two different aspects of this necessity:
      
      On the one hand, such mappings are often passed to algorithms.
      Then, algorithms should make only minimal assumptions on the nature 
      of the mapping, in particular, they should not commit to a particular representation
      of such data associations.

      On the other hand, algorithms themselves often need to attach values to
      element temporarily. In this case, they have to be able to create such data
      associations themselves, and perhaps to decide on some performance-critical
      features of the implementations.
      
      The underlying mathematical model of the first usage scenario 
      is that of a function $f$ from the set 
      of k-Elements of a Grid \T${\cal G}$ \W$G$ to values of an arbitrary type {\em T\/}:

      \begin{center}
        \begin{figure}[h]
          \W\rhtml{<TABLE> <TR>}
          \T\begin{minipage}{6cm}
            \T\includegraphics[width=5cm]{bilder/gf-map-cell}
            \W\rhtml{<TD>}
            \W\htmlimage[ALT="Grid function on cells"]{\img{gf-map-cell}}
            \caption{grid function on cells}
            \W\rhtml{</TD>}
            \T\end{minipage}  \T\hspace{1cm}
          \T\begin{minipage}{6cm}
            \T\includegraphics[width=5cm]{bilder/gf-map-edge}
            \W\rhtml{<TD>}
            \W\htmlimage[ALT="Grid function on edges"]{\img{gf-map-edge}}
            \caption{Grid function on edges}
            \W\rhtml{</TD>}
            \T\end{minipage}
          \W\rhtml{</TR> </TABLE>}
        \end{figure}
      \end{center}



      The grid function concept \sectionlink{(C)}{GridFunction},
      is a natural interface
      which is close to the  mathematical model.
      This interface can be implemented by many different representations,
      including calculation on-the-fly, and corresponds to the first
      requirement (data abstraction).
      
      Beyond this, the second scenario (temporary values / local grid functions)
      requires a  container component,
      which extends this interface with
      methods for creation of grid functions. 
      The corresponding concept 
      is {\em container grid function\/}  \sectionlink{(C)}{ContainerGridFunction}.

      There is a interesting and important choice left open now:
      Either, we can reserve memory for the value of each element in the grid,
      and considering each value not explicitly set on an element an error.
      Or, we could determine a default value, and consider each element whose
      valeu is not explicitely set having this default value.
      The first option leads to the {\em total\/} grid function concept
      \sectionlink{ (C) }{TotalGridFunction},
      storing a value for every element.
      
      The second option gives rise to  {\em partial\/}
      grid functions     \sectionlink{ (C) }{PartialGridFunction}
      which stores values  only on a part of the grid elements,
      assigning a default value to the others.
      
      Depending on the choice, there are slightly different semantics of the constructors,
      as well as same new operations in the case of  partial grid functions.
      
      Owing to the great importance of both total and partial grid function containers
      for the implementation of many algorithms, these entities should be 
      available under a uniform syntax, if they are defined for a concrete grid type.
      More precisely, the following class templates are then available: 
     {\small ({\tt E} is constrained to be an element type \sectionlink{(C)}{GridElement} 
      associated to this grid type, {\tt T} can be chosen arbitrary.)}
     
\begin{example}
template<class E, class T> 
class grid\_function; // total grid function 
template<class E, class T>
class partial\_grid\_function; // partial grid function
\end{example}

There are examples of grids types without such associated grid functions.
These are typically used for special restricted purposes only,
like grid adapters for input from a file, which offer a clean way 
to encapsulate specific file formats from the rest of the world.
An example is 
\sectionlink{{\tt IstreamComplex2DFmt}}{istream-complex2d-fmt}.


\begin{Label}{refine-example}
{\bf Example:}
We can use  a partial grid function to mark cells for refinement:
\begin{example}
partial\_grid\_function<Cell,bool> refine(G,false);
....
for(CellIterator c = G.FirstCell(); ! c.IsDone(); ++c)
 if(error(*c) > tolerance())
   refine[*c] = true;
\end{example}
\end{Label}


\xname{GeometricalLayer}
\begin{Label}{GeometricalLayer}
\introsection{The geometric layer}
\end{Label}

\needswork{This documentation is still incomplete!}

Until now, we have seen grids as purely combinatorial constructs. 
Adding geometric functionality in a {\em separate\/} layer
has the advantage to make a combinatorial grid usable in a 
broader context, 
because there may be many different geometric embeddings for
a grid. 
Also, there are a lot of algorithms that do not require any geometry at all.

Some of the major points of variation of the \emph{mathematical\/} aspects
of geometric embeddings are
\begin{itemize}
\item  embedding in higher-dimensional space (2D grid in 3D space)
\item embedding in curved geometry (e.g. on a sphere)
\item non-linear elements
\item time-dependent embeddings 
\end{itemize}

Furthermore, geometric embeddings can differ in \emph{computational\/} aspects:
\begin{itemize}
\item arithmetic: exact or floating-point?
\item storage or calculation of entities?
\item exact or approximate calculation (e.\ g.\ cell volume
  for nonlinear embeddings)
\end{itemize}

   \begin{figure}[h]
     \begin{center}
       \T\includegraphics{bilder/zwei-geometrien.eps}
       \W\htmlimage[ALT="A linear and a non-linear geometry for the same grid"]{\img{zwei-geometrien}}
      \caption{A linear and a non-linear geometry for the same grid}
     \end{center}
   \end{figure}


What kind of functionality is available in a class implementing a 
geometric embedding for a grid?
This question cannot be answered in general.
The most basic geometry concept is that of \conceptlink{Vertex Grid Geometry}{VertexGridGeometry},
which just allows access to vertex coordinates.
A more advanced concept is 
\conceptlink{Volume Grid Geometry}{VolumeGridGeometry},
which defines geometric counterparts for all combinatorial grid elements,
as shown in the table.

\begin{tabular}{ll}\\
  \hline
  \multicolumn{2}{c}{\bf types}  \\ 
  \hline
  {\tt typedef coord\_type} \conceptlink{(C)}{GeomCoord} & point in space  \\
  {\tt typedef segment\_type} \conceptlink{(C)}{GeomSegment} &  1D segment corr. to {\tt Edge} \\ 
  {\tt typedef polygon\_type} \conceptlink{(C)}{GeomPolygon}    &  2D polygon corr. to {\tt Face} \\ 
  {\tt typedef polyhedron\_type} \conceptlink{(C)}{GeomPolyhedron} &  3D polyhedron corr. to {\tt Cell} (in 3D Grids)\\
  \hline
  \multicolumn{2}{c}{\bf functionals from combinatorics to geometry} \\ 
  \hline
  \begin{tabular}{l}
    {\tt coord\_type  coord(Vertex)} \\
    {\tt segment\_type segment(Edge)} \\
    {\tt polygon\_type polygon(Face)} \\
    {\tt polyhedron\_type polyhedron(Cell)} 
  \end{tabular}
  &
  \begin{tabular}{p{5cm}} 
    mappings of combinatorial
    entities to geometric entities 
  \end{tabular} \\
  \hline
  \\
 \end{tabular}

There may be a lot more functionality available, depending on the
actual geometry. For example, if the combinatorial and the geometric
dimension are equal, one may define outward normal in
the centers of facets.



\T\begin{small}

\xchapter{Concepts}{Concepts}
\W\htmlmenu{2}
%% entities
% % elements
\section{Grid Entities}
\T\begin{small}
\xname{GridEntity}
\begin{Label}{GridEntity}
\conceptsection{Grid Entity Concept}
\end{Label}

\conceptsubsection{Description}
The {\em Grid Entity\/} concept is rather abstract in that it gives rise to rather
separate sub-concepts, notably \sectionlink{\concept{ Grid Element }}{GridElement},
\sectionlink{\concept{Grid Sequence Iterator}}{GridSequenceIterator},
and \conceptlink{Grid Incidence Iterator}{GridIncidenceIterator}.
Its purpose is to bundle some fundamental properties, much like the STL concepts
\Stllink{Assignable} and
\stllink{Equality Comparable}{EqualityComparable}
which it refines.

The fundamental property captured by {\em Grid Entity\/} is binding to a 
\sectionlink{\concept{Grid}}{Grid}
(the {\em anchor grid\/}), 
and global identity, 
in contrast to \sectionlink{\concept{Element Handles}}{GridElementHandle},
which are unique only within a single grid.

Conceptually, a grid is the owner of its entities; 
an entities lifetime cannot exceed that of
its anchor grid. 
Also, a grid entity needs  not necessarily to be stored permanently,
it may be constructed only temporarily, 
for example a
\concept{Grid Iterator}
or a  
\sectionlink{\concept{Grid Element}}{GridElement}
obtained by dereferencing a  Grid Iterator.

\conceptsubsection{Refinement of}
STL \Stllink{Assignable}
\\
STL \stllink{Equality Comparable}{EqualityComparable}


\conceptsubsection{Notation}
{\tt E} is a type which is a model of {\em Grid Entity\/}
\\
{\tt e, e1, e2} are objects of type {\tt E}

\conceptsubsection{Definitions}

      \label{anchor}
      An {\em anchor\/} of {\tt e} is a logically superior entity.
      It is either an object of a type which is a model of 
      \sectionlink{\concept{Grid}}{Grid}
      or an object of a type which is a model of 
      \sectionlink{\concept{Grid Element}}{GridElement},
      to which {\tt e} is incident. 
      This is defined more precisely depending on the
      subconcept involved. For example, for
      \sectionlink{\concept{ Grid Element}}{GridElement}
      the anchor is the underlying grid;
      for a \sectionlink{\concept{Grid Sequence Iterator}}{GridSequenceIterator}
      on a 
      \sectionlink{\concept{Grid Range}}{GridRange},
      it is the grid range, 
      and for a
      \sectionlink{\concept{Grid Incidence Iterator}}{GridIncidenceIterator},
      it is a grid element.
      So, the anchor of a 
      \sectionlink{\concept{Vertex}-On-Cell Iterator}{Vertex-On-CellIterator},
      is the cell over whose incident vertices the iteration is performed.
 
    \label{valid}
    
     A grid entity {\tt e} is {\em valid\/}, 
      if its anchor is valid (either refers to an existing grid, or is itself a valid Grid Entity),
      and its handle is a valid handle of that grid.


\conceptsubsection{Associated types}

\begin{tabular}{lll} 
  \\ \hline
  \bf  Name  & \bf  Expression  &\bf  Description   \\ \hline
  Grid type &  {\tt E::grid\_type} &
  type of the corresponding \sectionlinkshort{anchor grid}{Grid}   \\ 
  Anchor type &  {\tt E::anchor\_type} &
  type of the corresponding \footlink{anchor}{anchor}   \\ 
  handle type &  {\tt E::handle\_type} &
  type of the corresponding \sectionlinkshort{element handle}{GridElementHandle}  \\ \hline
  \\
\end{tabular}



\conceptsubsection{Valid Expressions}
\begin{tabular}{llll}
  \\ \hline
  \bf  Name  &\bf  Expression  &\bf  Type requirements  & \bf  return type  \\ \hline
  Anchor grid &  {\tt e.TheGrid()} & 
   ~ & 
  {\tt grid\_type const\&} 
  \\ 
  Anchor entity &  {\tt e.TheAnchor()} & 
  ~ & 
   {\tt anchor\_type const\&} 
   \\
   Handle &  {\tt e.handle()} & 
   ~ & 
   {\tt handle\_type}
   \\ \hline
   \\
 \end{tabular}
 
 \conceptsubsection{Expression semantics}
 
 %\begin{tabular}{|p{2cm}|l|p{4cm}|p{4cm}|p{4cm}} \hline
 \begin{tabularx}{15cm}{XlXXX} 
   \\ \hline 
   \bf  Name    &
   \bf  Expression &
   \bf  Precondition&
   \bf  Semantics &
   \bf  Postcondition
   \\ \hline
   Copy constructor &
   {\tt E e1(e2)} &
   ~    &
   ~    &
   {\tt e1 == e2} 
   \\ 
   Assignment &
   {\tt e1 = e2} &
   ~    &
   {\tt e1} is a copy of {\tt e2}       &
   {\tt e1 == e2}
   \\ 
   Anchor grid  &
   {\tt e.TheGrid()} &
   {\tt e} is \link{valid}[~(section \Ref)~]{valid} &
   get the grid to which {\tt e} is bound   & 
   ~  
   \\ 
   Equality comparison &
   {\tt e1 == e2} &
   {\tt e1.TheAnchor() == e2.TheAnchor()} \noteref{note-anchor} &
     true if {\tt e1} and {\tt e2}
     refer to the same element of  {\tt e1.TheGrid()} 
     {\tt e1 == e2} is equivalent to
     {\tt e1.handle() == e2.handle()} 
     and 
     {\tt e1.TheAnchor() == e2.TheAnchor()}\noteref{note-grid-comparison}
  & ~
   \\ \hline
   \\ 
 \end{tabularx}

 \conceptsubsection{Refinements}
 \sectionlink{\concept{Grid Element}}{GridElement}
 \\ 
 \sectionlink{\concept{Grid Sequence Iterator}}{GridSequenceIterator}
 \\ 
 \sectionlink{\concept{Grid Incidence Iterator}}{GridIncidenceIterator}
 \\ 

 \conceptsubsection{Notes}

 \begin{enumerate}
 \item 
   \notelabel{note-grid-comparison}
   This does not imply that for grids, there is a real equality comparison required.
   Rather, equality of grids generally means  {\em physical\/} identity of the corresponding
   objects, that is, their addresses.
   In contrast, equality for Grid Entities generally means {\em logical\/} rather than physical
   identity, because they might be created only temporarily.
   
   Put another way, for grid entities, {\tt e1 = e2} entails {\tt e1 == e2},
   which is not the case for grids.

 \item
   \notelabel{note-anchor}
   It can be discussed if this precondition is necessary. 
   Formally, two entities
   compare as different if not {\tt e1.TheAnchor() == e2.TheAnchor()}.
   However, requiring this equality allows to compare only the handles, 
   which is  faster.
 \end{enumerate}

 \conceptsubsection{See also}
 
 \sectionlink{\concept{Grid}}{Grid} ~
 \sectionlink{\concept{Element Handle}}{GridElementHandle}

      
  


\xname{GridElement}    
\begin{Label}{GridElement}
\conceptsection{Grid Element Concept}
\end{Label}

\conceptsubsection{Description}

A {\em Grid Element\/} is an entity, such as a 
\sectionlink{\concept{Grid Vertex}}{GridVertex},
that belongs to a {\em Grid\/}.
To each grid element, there is associated a unique grid (the {\em anchor\/} grid). 
Two elements may be compared for
equality, if they belong to the same grid.
    
Conceptually, a combinatorial grid consists of its elements of different dimension
(Vertices, Edges and so on), plus an incidence relation between them. 
This does not imply, however, that the element constituing a grid must be stored
permanently within the grid.

We name the element types of a grid consistently
according to the following table, where we
distinguish between names relating to element dimension and element
codimension\noteref{note-dim-codim}
    
\begin{tabular}{lcc}
  \T \\   \hline
  \bf  k-Element &  \bf  Dimension &  \bf  Codimension  \\ 
  \hline
  {\tt Vertex} &   0 &   ~  \\ 
  {\tt Edge} &   1 &   ~  \\ 
  {\tt Face}\noteref{note-face}  &   2 &   ~  \\ 
  {\tt Facet} &   ~ &   1  \\ 
  {\tt Cell} &   ~ &   0   
  \T \\   \hline  \\
\end{tabular}
    
This naming scheme allows for a dimension-independent
formulation of many algorithms: 
for \link{example}{flux-example} fluxes in finite volume algorithms
are always defined on Facets.
      
\conceptsubsection{Refinement of}
\sectionlink{\concept{Grid Entity}}{GridEntity}
    
The only refinement is, in fact, that the anchor type is equal to the grid type.

\conceptsubsection{Notation}
{\tt E} is a type which is a model of {\em Grid Element\/}
\\
{\tt e, e1, e2} are objects of type {\tt E}
\\
{\tt g} is an object of type {\tt E::grid\_type}

\conceptsubsection{Associated types}

\begin{tabularx}{12cm}{llX}
  \T \\ \hline
  \bf  Name  &\bf  Expression  &\bf  Description   \\ 
  \hline
  Grid type &  {\tt E::grid\_type} &
  the same as {\tt E::anchor\_type}
  (defined in \sectionlink{\concept{Grid Entity}}{GridEntity})  
  \T \\ \hline \\
\end{tabularx}

\conceptsubsection{Valid Expressions}
\begin{tabular}{llll}
  \T \\  \hline
  \bf  Name  &\bf  Expression  &\bf  Type requirements  & \bf  return type  \\ 
  \hline
  Anchor  Grid &  {\tt e.TheGrid()} &  ~ &  {\tt grid\_type const\&}  
  \T \\   \hline \\
\end{tabular}

\conceptsubsection{Expression semantics}

\T\begin{tabular}{p{2cm}p{3cm}lp{3cm}l}  \\ \hline
\W\begin{tabular}{lllll} 
  \bf  Name     &
  \bf  Expression &
  \bf  Precondition&
  \bf  Semantics &
  \bf  Postcondition
  \\ 
  \hline
  Anchor grid reference &
  {\tt grid\_type const\& g = e.TheGrid()}  &
  {\tt e} is valid &
  equivalent to  {\tt g = e.TheGrid()} &
  ~  
  \T \\ \hline \\
\end{tabular}

\W\conceptsubsection{Complexity guarantees}

\conceptsubsection{Refinements}
\sectionlink{\concept{Vertex}}{GridVertex}
\\
\sectionlink{\concept{Edge}}{GridEdge}
\\
\sectionlink{\concept{Face}}{GridFace}
\\
\sectionlink{\concept{Facet}}{GridFacet}
\\
\sectionlink{\concept{Cell}}{GridCell}
\\

\W\conceptsubsection{Models}


\conceptsubsection{Notes}
\begin{enumerate}
\item 
  \notelabel{note-dim-codim}
  some of these types can coincide: for a concrete 2D-grid, 
  the types {\tt Edge}
  and {\tt Facet} can be the same. 
  But it is also possible to define  them as distinct types.

\item     
  \notelabel{note-face}
  There cannot be a  type {\tt Face} defined for 1D-grids.

\end{enumerate}

\conceptsubsection{See also}

\sectionlink{\concept{Grid}}{Grid} ~
\sectionlink{\concept{Grid  Entity}}{GridEntity} ~
\sectionlink{\concept{Grid Element Handle}}{GridElementHandle} 

    
  


\xname{GridVertex}
\begin{Label}{GridVertex}
\conceptsection{Grid Vertex Concept}
\end{Label}

\conceptsubsection{Description}
A  {\em Grid Vertex\/} represents the mathematical concept of a vertex 
--- a 0-dimensional entity in a \sectionlink{\concept{ Grid }}{Grid}.

\conceptsubsection{Refinement of}
\sectionlink{\concept{ Grid Element}}{GridElement}
 
\conceptsubsection{Notation}
{\tt V} is a type which is a model of {\em grid vertex\/}
\\
{\tt v} is an object of type {\tt V}
\\
{\tt g} is an object of type {\tt V::grid\_type}
\\
{\tt h} is an object of type {\tt V::vertex\_handle}
\\
{\tt ci} is an object of type {\tt V::CellIterator}


\conceptsubsection{Associated types}

\noindent
{\em NOTE:\/} The types and expression involving 
\sectionlink{\concept{ Incidence Iterators}}{GridIncidenceIterator} are given below for the
case of \sectionlinkUNDEF{cell-on-vertex iteration}{CellOnVertexIterator}.
Analogous types and expressions can be defined for the other element types,
like edge, facet, or cell. 
The tables are to be understood in the following sense:
\\
{\em If\/} a vertex defines the incidence iterator over cells,
{\em then\/} the requirements under {\em Optional part\/} apply.
Analogous requirements take effect if `cell' is replaced by another element type.

\noindent
\begin{tabularx}{12cm}{llX} 
  \\
  \hline
  \bf  Name  &\bf  Expression  &\bf  Description   \\ \hline
  handle type & 
  {\tt V::vertex\_handle} &
  type of the corresponding \sectionlink{\concept{Vertex Handle}}{GridVertexHandle} 
  \\ 
  \hline
  \multicolumn{3}{c}{\bf \em Optional part (as example) }
  \\
  \hline
  cell-on-vertex iterator &
  {\tt V::CellIterator}&
  type of the corr. CellOnVertexIterator
  \\ \hline
\end{tabularx}
    
\conceptsubsection{Valid Expressions}

\noindent
\begin{tabularx}{14cm}{Xlll} 
%  \\ 
  \hline
  \bf  Name  &\bf  Expression  &\bf  Type requirements  & \bf  return type \\ 
  \hline
  handle & 
  {\tt v.handle()} &
  ~ &
  {\tt V::vertex\_handle} 
  \\
  \hline
  \multicolumn{3}{c}{\bf \em Optional part (as example) }
  \\
  \hline
  cell-on-vertex iteration start & 
  {\tt v.FirstCell()} &
  ~ &
  {\tt V::CellIterator} 
  \\
  number of incident cells & 
  {\tt v.NumOfCells()} &
  ~ &
  {\tt int} 
  \\
  \hline
  \\
\end{tabularx}
 
\T\begin{small}
\conceptsubsection{Expression semantics}
\begin{tabularx}{15cm}{XXXXX} \\
  \hline
  \bf  Name     &
  \bf  Expression &
  \bf  Precondition&
  \bf  Semantics &
  \bf  Postcondition
  \\ 
  \hline
  handle &
  {\tt h = v.handle();} &
  v is \footlink{valid}{valid} &
  shorthand for {\tt h = v.TheGrid(). handle(v)} &
  {\tt v == v.TheGrid(). vertex(h)}  
  \\ 
  \hline
  \multicolumn{3}{c}{\bf \em Optional part (as example) }
  \\
  \hline
  cell-on-vertex iteration start & 
  {\tt ci = v.FirstCell()} &
  {\tt v} is \link{valid}{valid}  &
  let {\tt ci} point to  the first cell incident to {\tt v}  & 
  {\tt ci.TheVertex() == ci.TheAnchor() == v} 
   and 
  {\tt ci.TheGrid() == v.TheGrid()}
  \\ 
  number of incident cells & 
  {\tt n =  v.NumOfCells()} &
  {\tt v} is \link{valid}{valid}  &
  n is the number of cells that are incident to {\tt v} &
  ~ 
  \\ 
  \hline
  \\
\end{tabularx}
\T\end{small}
    
\conceptsubsection{Complexity guarantees}
All operations are amortized constant time\noteref{note-vtx-amort}.

\W\conceptsubsection{Refinements}

\conceptsubsection{Models}

Vertex classes of 
\gralmodulelink{complex2d} (see \gralclasslink{Vertex2D}{complex2d}),
\gralmodulelink{complex3d},  
\gralmodulelink{cartesian2d},
\gralmodulelink{cartesian3d}, or
\gralmodulelink{triang2d}.

\conceptsubsection{Notes}
\begin{enumerate}
\item \notelabel{note-vtx-amort}
  Amortization is understood to involve calling the operations for all
  vertices of a grid.
\end{enumerate}

\conceptsubsection{See also}
\sectionlink{\concept{Grid}}{Grid} ~
\sectionlink{\concept{Grid Element Handle}}{GridElementHandle} ~
\sectionlink{\concept{Vertex Handle}}{GridVertexHandle}~
\sectionlink{\concept{Grid Element }}{GridElement} ~
\sectionlink{\concept{Grid Cell}}{GridCell} ~
\sectionlink{\concept{Sequence Iterator}}{GridSequenceIterator} ~
\sectionlink{\concept{Incidence Iterator}}{GridIncidenceIterator} ~

  


\xname{GridEdge}
\begin{Label}{GridEdge}
\conceptsection{Grid Edge Concept}
\end{Label}

\conceptsubsection{Description}
A  {\em Grid Edge\/} represents the mathematical concept of an edge 
--- a 1-dimensional entity in a \sectionlink{\concept{Grid}}{Grid}.


\conceptsubsection{Refinement of}
\sectionlink{\concept{ Grid Element}}{GridElement}
 
\conceptsubsection{Notation}
{\tt E} is a type which is a model of \concept{Grid Edge}
\\
{\tt e} is an object of type {\tt E}
\\
{\tt V} is  a shorthand for the {\tt E::Vertex} type
\\
{\tt v} is an object of type {\tt V}
\\
{\tt g} is an object of type {\tt E::grid\_type}
\\
{\tt h} is an object of type {\tt E::edge\_handle}

\conceptsubsection{Associated types}

{\em NOTE:\/} The types and expression involving 
\sectionlink{\concept{ Incidence Iterators}}{GridIncidenceIterator} are given below for the
case of \sectionlink{cell-on-edge iteration}{Cell-On-EdgeIterator}.
Analogous types and expressions can be defined for the other element types,
like edge, facet, or cell. 
The tables are to be understood in the following sense:
\\
{\em If\/} a edge defines the incidence iterator over cells,
{\em then\/} the requirements under {\em Optional part\/} apply.
Analogous requirements take effect if `cell' is replaced by another element type.

\noindent
\begin{tabularx}{12cm}{llX} 
  \\
  \hline
  \bf  Name  &\bf  Expression  &\bf  Description   \\ \hline
  handle type & 
  {\tt E::edge\_handle} &
  type of the corresponding \sectionlink{\concept{Edge Handle}}{GridVertexHandle} 
  \\ 
  vertex type &
  \code{E::Vertex} &
  Vertex type of the edge, 
  short for \code{E::grid\_type::Vertex}.
  \\ 
  \hline
  \multicolumn{3}{c}{\bf \em Optional part (as example) }
  \\
  \hline
  cell-on-edge iterator &
  {\tt E::CellIterator}&
  type of the corr. CellOnEdgeIterator
  \\ 
  \hline
  \\
\end{tabularx}
    
\conceptsubsection{Valid Expressions}

\begin{tabularx}{12cm}{Xlll} \\ 
  \hline
  \bf  Name  &\bf  Expression  &\bf  Type requirements  & \bf  return type \\ 
  \hline
  handle & 
  {\tt e.handle()} &
  ~ &
  {\tt E::edge\_handle} 
  \\
  first vertex &
  \code{v = e.V1()} &
  ~ &
  \type{Vertex}
  \\
  second vertex &
  \code{v = e.V2()} &
  ~ &
  \type{Vertex}
  \\
  \hline
  \multicolumn{3}{c}{\bf \em Optional part (as example) }
  \\
  \hline
  cell-on-edge iteration start & 
  {\tt e.FirstCell()} &
  ~ &
  {\tt E::CellIterator} 
  \\
  number of incident cells & 
  {\tt E.NumOfCells()} &
  ~ &
  {\tt int} 
  \\
  \hline
  \\
\end{tabularx}
 
\T\begin{small}
\conceptsubsection{Expression semantics}
\begin{tabularx}{15cm}{XXXXX} \\
  \hline
  \bf  Name     &
  \bf  Expression &
  \bf  Precondition&
  \bf  Semantics &
  \bf  Postcondition
  \\ 
  \hline
  handle &
  {\tt h = e.handle();} &
  e is \footlink{valid}{valid} &
  shorthand for {\tt h = e.TheGrid(). handle(e)} &
  {\tt e == e.TheGrid(). edge(h)}  
  \\ 
  \hline
  \multicolumn{3}{c}{\bf \em Optional part (as example) }
  \\
  \hline
  cell-on-edge iteration start & 
  {\tt ci = e.FirstCell()} &
  {\tt e} is \link{valid}{valid}  &
  let {\tt ci} point to  the first cell incident to {\tt e}  & 
  {\tt ci.TheEdge() == ci.TheAnchor() == e} 
   and 
  {\tt ci.TheGrid() == e.TheGrid()}
  \\ 
  number of incident cells & 
  {\tt n =  e.NumOfCells()} &
  {\tt e} is \link{valid}{valid}  &
  n is the number of cells that are incident to {\tt e} &
  ~ 
  \\ 
  \hline
  \\
\end{tabularx}
\T\end{small}
    
\conceptsubsection{Complexity guarantees}
All operations are amortized constant time\noteref{note-edge-amort}.

\conceptsubsection{Refinements}

\conceptsubsection{Models}
\sectionlink{{\tt Complex2D::Edge}}{Complex2DEdge} 
defined in
\gralfilelink{edge2d}{h}{complex2d}

\conceptsubsection{Notes}
\begin{enumerate}
\item \notelabel{note-edge-amort}
  Amortization is understood to involve calling the operations for all
  edges of a grid.
\end{enumerate}

\conceptsubsection{See also}
\sectionlink{\concept{Grid}}{Grid} ~
\sectionlink{\concept{Grid Element Handle}}{GridElementHandle} ~
\sectionlink{\concept{Edge Handle}}{GridVertexHandle}~
\sectionlink{\concept{Grid Element }}{GridElement} ~
\sectionlink{\concept{Grid Cell}}{GridCell} ~
\sectionlink{\concept{Sequence Iterator}}{GridSequenceIterator} ~
\sectionlink{\concept{Incidence Iterator}}{GridIncidenceIterator} ~

  


\xname{GridFacet}
\begin{Label}{GridFacet}
\conceptsection{Grid Facet Concept}
\end{Label}

\conceptsubsection{Description}
A  {\em Grid Facet\/} represents the mathematical concept of a facet 
--- a 1-codimensional entity ($d-1$ dimensional) 
in a ($d$-dimensional) \sectionlink{\concept{Grid}}{Grid}.

\conceptsubsection{Refinement of}
\sectionlink{\concept{ Grid Element}}{GridElement}
 
\conceptsubsection{Notation}
{\tt F} is a type which is a model of \concept{Grid Facet}.
\\
{\tt f} is an object of type {\tt F}
\\
{\tt C} is  a shorthand for the {\tt F::Cell} type
\\
{\tt c} is an object of type {\tt C}
\\
{\tt g} is an object of type {\tt F::grid\_type}
\\
{\tt h} is an object of type {\tt F::facet\_handle}
\\
{\tt vf} is an object of type {\tt F::VertexIterator}

\conceptsubsection{Associated types}

{\em NOTE:\/} The types and expression involving 
\sectionlink{\concept{ Incidence Iterators}}{GridIncidenceIterator} are given below for the
case of \sectionlink{vertex-on-facet iteration}{Vertex-On-CellIterator}.
Analogous types and expressions can be defined for the other element types.

The tables are to be understood in the following sense:
\\
{\em If\/} a facet defines the incidence iterator over vertices,
{\em then\/} the requirements under {\em Optional part\/} apply.
Analogous requirements take effect if `vertex' is replaced by another element type.

\noindent
\begin{tabularx}{12cm}{llX} \hline
  \bf  Name  &\bf  Expression  &\bf  Description   \\ \hline
  handle type & 
  {\tt F::facet\_handle} &
  type of the corresponding \sectionlink{\concept{Facet Handle}}{GridVertexHandle} 
  \\ 
  cell type &
  \code{F::Cell} &
  Cell type of the facet, 
  short for \code{F::grid\_type::Cell}.
  \\ 
  \hline
  \multicolumn{3}{c}{\bf \em Optional part (as example) }
  \\
  \hline
  vertex-on-facet iterator &
  {\tt F::VertexIterator}&
  type of the corr. VertexOnFacetIterator
  \\ 
  \hline
\end{tabularx}
    
\conceptsubsection{Valid Expressions}

\noindent
\begin{tabularx}{14cm}{RRRR} \\ 
  \hline
  \bf  Name  &\bf  Expression  &\bf  Type requirements  & \bf  return type \\ 
  \hline
  handle & 
  {\tt f.handle()} &
  ~ &
  {\tt F::facet\_handle} 
  \\
  first cell &
  \code{c = f.C1()} &
  ~ &
  \type{Cell}
  \\
  second cell &
  \code{c = f.C2()} &
  ~ &
  \type{Cell}
  \\
  \hline
  \multicolumn{4}{c}{\bf \em Optional part (as example) }  
  \\
  \hline
  vertex-on-facet iteration start & 
  {\tt f.FirstVertex()} &
  ~ &
  {\tt F::VertexIterator} 
  \\
  number of incident vertices & 
  {\tt F.NumOfVertices()} &
  ~ &
  {\tt int} 
  \\
  \hline
  \\
\end{tabularx}
 
\T\begin{small}
\conceptsubsection{Expression semantics}
\begin{tabularx}{15cm}{XXXXX} \\
  \hline
  \bf  Name     &
  \bf  Expression &
  \bf  Precondition&
  \bf  Semantics &
  \bf  Postcondition
  \\ 
  \hline
  handle &
  {\tt h = f.handle();} &
  f is \footlink{valid}{valid} &
  shorthand for {\tt h = f.TheGrid(). handle(f)} &
  {\tt f == f.TheGrid(). facet(h)}  
  \\ 
  \hline
  \multicolumn{5}{c}{\bf \em Optional part (as example) }
  \\
  \hline
  Vertex-on-facet iteration start & 
  {\tt vf = f.FirstVertex()} &
  {\tt f} is \link{valid}{valid}  &
  let {\tt fi} point to  the first vertex incident to {\tt f}  & 
  {\tt vf.TheEdge() == vf.TheAnchor() == f} 
   and 
  {\tt vf.TheGrid() == f.TheGrid()}
  \\ 
  number of incident vertices & 
  {\tt n =  f.NumOfVertices()} &
  {\tt f} is \link{valid}{valid}  &
  n is the number of vertices that are incident to {\tt f} &
  ~ 
  \\ 
  \hline
  \\
\end{tabularx}
\T\end{small}
    
\conceptsubsection{Complexity guarantees}
All operations are amortized constant time\noteref{note-facet-amort}.

\conceptsubsection{Refinements}

\conceptsubsection{Models}
\sectionlink{\type{Complex2D::Facet}}{Complex2DEdge} 
defined in
\gralfilelink{edge2d}{h}{complex2d}
(identical to \type{Complex2D::Facet})
\conceptsubsection{Notes}
\begin{enumerate}
\item \notelabel{note-facet-amort}
  Amortization is understood to involve calling the operations for all
  facets of a grid.
\end{enumerate}

\conceptsubsection{See also}
\sectionlink{\concept{Grid}}{Grid} ~
\sectionlink{\concept{Grid Element Handle}}{GridElementHandle} ~
\sectionlink{\concept{Facet Handle}}{GridVertexHandle}~
\sectionlink{\concept{Grid Element }}{GridElement} ~
\sectionlink{\concept{Grid Cell}}{GridCell} ~
\sectionlink{\concept{Sequence Iterator}}{GridSequenceIterator} ~
\sectionlink{\concept{Incidence Iterator}}{GridIncidenceIterator} ~

  


 
\xname{GridCell}
\begin{Label}{GridCell}
\conceptsection{Grid Cell Concept}
\end{Label}

\conceptsubsection{Description} 
The {\em Grid Cell\/} concept  corresponds to the mathematical concept of a d-dimensional entity
of a d-dimensional \sectionlink{\concept{Grid}}{Grid}

\conceptsubsection{Refinement of}
\sectionlink{\concept{Grid Element}}{GridElement}

\conceptsubsection{Notation}
{\tt C} is a type which is a model of \concept{Grid Cell}
\\
{\tt c} is an object of type {\tt C}
\\
{\tt g} is an object of type {\tt C::grid\_type}
\\
{\tt h} is an object of type {\tt C::cell\_handle}
\\

\conceptsubsection{Associated types}

\noindent
\begin{tabular}{lll} 
  \T \hline
  \bf  Name  &\bf  Expression  &\bf  Description  \\ \hline
  handle type & 
  {\tt C::cell\_handle} &
  type of the corresponding \sectionlink{\concept{cell handle}}{GridVertexHandle} 
 \T \\ \hline
\end{tabular}


\conceptsubsection{Valid Expressions}

\noindent
\begin{tabular}{llll} 
  \T \hline
  \bf  Name  &\bf  Expression  &\bf  Type requirements  & \bf  return type  \\ \hline
  handle & 
  {\tt c.handle()} &
  ~ &
  {\tt C::cell\_handle} 
  \T \\ \hline
\end{tabular}

\conceptsubsection{Expression semantics}

\noindent
\begin{tabularx}{14cm}{RRRRR}
  \T \hline
  \bf  Name     &
  \bf  Expression &
  \bf  Precondition&
  \bf  Semantics &
  \bf  Postcondition
  \\ \hline
  handle &
  {\tt h = c.handle();} &
  {\tt c} is \link{valid}{valid} &
  shorthand for 
  \par  {\tt h = c.TheGrid(). handle(c)} &
  {\tt c == c.TheGrid().cell(h)}  
 \T \\ \hline
\end{tabularx}

\conceptsubsection{Refinements}

\par
\noindent
\sectionlinkUNDEF{\concept{Vertex-Range Cell}}{Vertex-RangeCell}
\\
\sectionlinkUNDEF{\concept{Edge-Range Cell}}{Edge-RangeCell}
\\
\sectionlinkUNDEF{\concept{Neighbor-Range Cell}}{Neighbor-RangeCell}
\\
\sectionlinkUNDEF{\concept{Full Grid Cell}}{FullGridCell}
\\

\conceptsubsection{Models}
Cell classes of 
\gralmodulelink{complex2d} (see \gralclasslink{Cell2D}{complex2d}),
\gralmodulelink{complex3d},  
\gralmodulelink{cartesian2d},
\gralmodulelink{cartesian3d}, or
\gralmodulelink{triang2d}.

\conceptsubsection{See also}
\sectionlink{\concept{Grid Vertex}}{GridVertex} ~
\sectionlink{\concept{Grid Element}}{GridElement} ~
\sectionlink{\concept{Vertex-On-Cell Iterator}}{Vertex-On-CellIterator} ~
\sectionlink{\concept{Grid Range}}{GridRange} ~
\sectionlink{\concept{Vertex Grid Range}}{VertexGridRange} ~

  


\xname{GridElementHandle} 
\begin{Label}{GridElementHandle} 
 \conceptsection{Grid Element Handle Concept}
\end{Label}

\conceptsubsection{Description}
A {\em Grid Element Handle\/} is a minimal representation of  a
\sectionlink{\concept{Grid Elements}}{GridElement}. 
They are unique only within a single 
\sectionlink{\concept{Grid}}{Grid}, 
which allows to map back and forth between handles and their
corresponding elements.
 

\conceptsubsection{Refinement of}
STL \xlink{Assignable}{http://www.sgi.com/Technology/STL/Assignable.html}
\\
STL \xlink{Equality Comparable}{http://www.sgi.com/Technology/STL/EqualityComparable.html}



\W\conceptsubsection{Notation}

\W\conceptsubsection{Valid Expressions}

\W\conceptsubsection{Expression semantics}

\conceptsubsection{Refinements}
\conceptlink{Grid Vertex Handle}{GridVertexHandle}
\\
\conceptlink{Grid Edge   Handle}{GridVertexHandle}
\\
\conceptlink{Grid Facet  Handle}{GridVertexHandle}
\\
\conceptlink{Grid Cell   Handle}{GridVertexHandle}
\\

\W\conceptsubsection{Models}

\conceptsubsection{Note}
\begin{enumerate}
\item \notelabel{note-handles-are-builtin}
  Typical models of \concept{Element Handles} are basic built-in types, 
  such as integral types or pointers.
  As such, it is not required that the types of handles corresponding to different element types
  (such as vertices and cells) be themselves different.
\item
  Often, handles have to fullfil additional requirements. 
  For example, implementations of \conceptlink{Container Grid Functions}{ContainerGridFunction}
  often exploit special properties of handles, such as being consecutively ordered integral types,
  or being a hashable type (that is, the \stltypelink{hash}{hash} template has been specialized
  for them).
\end{enumerate}

\conceptsubsection{See also}
\sectionlink{\concept{Grid}}{Grid} ~
\sectionlink{\concept{ Grid Entity}}{GridEntity} ~
\sectionlink{\concept{ Grid Element}}{GridElement} ~



\xname{GridVertexHandle}
\begin{Label}{GridVertexHandle}
\conceptsection{Vertex (Cell, Edge, ...) Handle Concept}
\end{Label}


\conceptsubsection{Description}
A {\em Grid Vertex Handle\/} is a placeholder for a 
\sectionlink{\concept{Vertex}}{GridVertex}
with minimal space requirements.

The same description applies in principle to cell handles, edge handles and so on.
Therefore, this documentation is understood to stand {\em sui generis\/} for these, too.

\conceptsubsection{Refinement of}
\sectionlink{\concept{Grid Element Handle}}{GridElementHandle}

\W\conceptsubsection{Notation}
{\tt G} is a type which is a model of \sectionlink{\concept{Grid}}{Grid}
\\
{\tt g} is an object of type  {\tt G}
\\
\type{H} is a shorthand for \type{G::vertex\_handle}, which is a model 
of \concept{Grid Vertex Handle}
\\
{\tt h1, h2} are objects of type \type{H}

\W\conceptsubsection{Associated Types}

\W\conceptsubsection{Valid Expressions}

\conceptsubsection{Expression semantics}
\begin{tabularx}{15cm}{Rllp{5cm}R} \\ 
  \hline 
  \bf  Name     &
  \bf  Expression &
  \bf  Precondition&
  \bf  Semantics &
  \bf  Postcondition
  \\ 
  \hline
  Equality comparison & 
  {\tt h1 == h2} &
  ~ &
  equivalent to  {\tt g.vertex(h1) == g.vertex(h2)}
  \par (if  {\tt h1, h2}
  are valid handles of {\tt g}) &
  ~  
  \\ 
  \hline
  \\
\end{tabularx}


\conceptsubsection{Models}
{\tt Complex2D::vertex\_handle} is equal to {\tt int}.
\\
{\tt Complex2D::cell\_handle} is equal to {\tt int}.
\\
A generic version for facet handles 
(for non-stored facets)
is {\tt facet\_handle<Complex2D::cell\_handle>},
defined in \gralcodelink{facet-handle.h}{base}.

\conceptsubsection{See also}
\sectionlink{\concept{ Grid }}{Grid} ~
\sectionlink{\concept{ Grid Entity}}{GridEntity} ~
\sectionlink{\concept{ Grid Element}}{GridElement} ~
\sectionlink{\concept{ Grid Element Handle}}{GridElementHandle} ~
\sectionlink{\concept{ Grid Vertex}}{GridVertex} ~

  



 % iterators
\section{Grid Iterators}
\xname{GridSequenceIterator}
\begin{Label}{GridSequenceIterator}
\conceptsection{Grid Sequence iterator Concept}
\end{Label}

\conceptsubsection{Description}
A {\em Grid Sequence Iterator\/}
lets a \sectionlink{\concept{Grid Range}}{GridRange}
be seen as a sequence of the iterators element type.

\conceptsubsection{Refinement of}
STL \stllink{Forward Iterator}{ForwardIterator}
\noteref{note-forward}
\\
\sectionlink{\concept{Grid Entity}}{GridEntity}


\conceptsubsection{Notation}
{\tt I} is a model of sequence iterator
\\ 
{\tt i, j} are  objects of type  {\tt I}

\conceptsubsection{Associated types}
\begin{tabularx}{12cm}{llR} 
  \T \\ \hline
  \bf  Name  &\bf  Expression  &\bf  Description  \\ 
  \hline
  grid type  & 
  {\tt I::grid\_type} &
  type of the  underlying grid, 
  model of  \sectionlink{\concept{Grid}}{Grid}
  \\ 
  anchor type  & 
  {\tt I::anchor\_type} &
  type of the underlying grid range,
   model of \sectionlink{\concept{Grid Range}}{GridRange}
  \\ 
  element type  & 
  {\tt I::element\_type} &
  type of the underlying element, 
  model of \sectionlink{\concept{Grid Element}}{GridElement}
  \\ 
  value type  &
  {\tt I::value\_type} &
  synomym for {\tt I::element\_type}   
  \T \\ \hline  \\
\end{tabularx}
 

\conceptsubsection{Valid Expressions}
\begin{tabular}{llll} 
  \T \\ \hline
  \bf  Name  &\bf  Expression  &\bf  Type requirements  & \bf  return type  \\ 
  \hline
  prefix increment  &
  {\tt ++it;} &
  ~ &
  {\tt I\&} 
  \\ 
  dereference  &
  {\tt *it;} & 
  ~ &
  {\tt I::element\_type} 
  \\
  equality comparison  &
  {\tt i == j} & 
  ~ &
  {\tt bool} 
  \\ 
  validity check  &
  {\tt i.IsDone();} & 
  ~ &
  {\tt bool} 
  \T \\ \hline  \\
\end{tabular}

   
\conceptsubsection{Expression semantics}

\begin{tabularx}{15cm}{RlRRR} 
  \T \\ \hline
  \bf  Name     &
  \bf  Expression &
  \bf  Precondition&
  \bf  Semantics &
  \bf  Postcondition
  \\ 
  \hline
  prefix increment  &
  {\tt ++i;} &
  {\tt ! i.IsDone()} &
  move iterator forward  & 
  {\tt i.IsDone()} or {\tt *i} is a valid grid element 
  \\ 
  dereference  &
  {\tt e = *it;} & 
  {\tt ! it.IsDone()} &  
  access the element {\tt it} points to &
  {\tt E == (*it);} 
  \\ 
  equality comparison  &
  {\tt i == j} & 
  {\tt \&(i.TheGrid()) == \&(j.TheGrid())} &  
  true if i and j reference the same element:  {\tt *i == *j}  &
  ~ 
  \\ 
  validity check  &
  {\tt i.IsDone();} & 
  ~ &  
  true iff {\tt i} is past-the-end. &
  ~ 
  \T \\  \hline  \\
\end{tabularx}

\W\conceptsubsection{Complexity guarantees}

\conceptsubsection{Refinements}
\sectionlink{\concept{Grid Vertex Iterator}}{GridVertexIterator}
\\
\sectionlink{\concept{Grid Edge Iterator}}{GridVertexIterator}
\\
\sectionlink{\concept{Grid Face Iterator}}{GridVertexIterator}
\\
\sectionlink{\concept{Grid Facet Iterator}}{GridVertexIterator}
\\
\sectionlink{\concept{Grid CellIterator}}{GridVertexIterator}
    
\W\conceptsubsection{Models}

\conceptsubsection{Notes}
\begin{enumerate}
\item 
  \notelabel{note-forward}
  For the existing models, this is currently only partially implemented: postfix increment is missing
  (it is inefficient because of the temporary object involved), and
  \xlink{{\tt iterator\_traits<>}}{\STLPATH{iterator_traits}.html} 
  is not (yet) specialized.
  However, the main property of Forward Iterator, namely support of multiple passes,
  is satisfied.
\end{enumerate}

\conceptsubsection{See also}
\sectionlink{\concept{Grid}}{Grid} ~
\sectionlink{\concept{Grid Element}}{GridElement} ~
\sectionlink{\concept{Grid Incidence Iterator}}{GridIncidenceIterator} 

  


 \xname{GridIncidenceIterator}
\begin{Label}{GridIncidenceIterator}
 \conceptsection{Grid Incidence Iterator Concept}
\end{Label}

 \conceptsubsection{Description}
 A {\em Grid Incidence Iterator\/} allows to access all elements of a given type incident
 to an given element (the {\em anchor\/}), 
 for example  all vertices incident to a given cell,
 or all edges incident to a given vertex. 
 
 For each combination of 
 \sectionlink{\concept{element types}}{GridElement}, 
 there is a subconcept of incidence iterator,
 for example,  the 
 \sectionlink{\concept{Vertex}-On-Cell Iterator}{Vertex-On-CellIterator}.

 \conceptsubsection{Refinement of}
 STL \stllink{Forward Iterator}{ForwardIterator}
 \\
 \sectionlink{\concept{Grid Entity}}{GridEntity}
 
 \conceptsubsection{Notation}
 {\tt I} is a model of incidence iterator
 \\ 
 {\tt i, j} are  objects of type  {\tt I}
 
 \conceptsubsection{Associated types}
 \begin{tabularx}{12cm}{llR}\\
   \hline
   \bf  Name  &\bf  Expression  &\bf  Description   \\ 
   \hline
   grid type  &  
   {\tt I::grid\_type} &
   type of the iterators underlying grid (range)   \\ 
   anchor type  & 
   {\tt I::anchor\_type} &
   type of the iterators anchor element  \\ 
   element type  &
   {\tt I::element\_type} &
   type of the \sectionlink{\concept{Element}}{GridElement} that the iterator references   \\ 
   value type  & 
   {\tt I::value\_type} &
   synomym for {\tt I::element\_type} \\ 
   \hline
   \\
 \end{tabularx}
 

 \conceptsubsection{Valid Expressions}
 \begin{tabular}{llll} \\
   \hline
   \bf  Name  &\bf  Expression  &\bf  Type requirements  & \bf  return type  \\
   \hline
   prefix increment  &
   {\tt ++it;} &
   ~ &
   {\tt I\&} 
   \\
   dereference  &
   {\tt *it;} & 
   ~ &
   {\tt I::element\_type} 
   \\
   equality comparison  &
   {\tt i == j} & 
   \ &
   {\tt bool} 
   \\
   validity check  &
   {\tt i.IsDone();}\noteref{note-isdone} & 
   ~ &
   {\tt bool} \\
   \hline
   \\
 \end{tabular}

   
 \conceptsubsection{Expression semantics}

\begin{tabularx}{15cm}{RlRRR} \\
  \hline      
  \bf  Name       &
  \bf  Expression &
  \bf  Precondition&
  \bf  Semantics &
  \bf  Postcondition
  \\ 
  \hline
  constructor  &
  {\tt i(a);} &
  {\tt e} is \link{valid}[~(see sec.\ \Ref)~]{valid} &
  set i to  {\tt e}'s first incident element of type {\tt I::element\_type}  & 
  {\tt i.TheAnchor() == a} 
  \\ 
  prefix increment  &
  {\tt ++i;} &
  {\tt ! i.IsDone()} &
  move iterator forward  & 
  {\tt i.IsDone()} or {\tt *i} is a \link{valid}{valid}
  grid element 
  \\ 
  dereference  &
  {\tt e = *it;} & 
  {\tt ! it.IsDone()} &  
  access the element {\tt it} points to &
  {\tt E == (*it);}
  \\ 
  equality comparison  &
  {\tt i == j} & 
  {\tt \&(i.TheGrid()) == \&(j.TheGrid())} &  
  true if i and j reference the same element:  {\tt *i == *j}  &
  ~
  \\ 
  validity check  &
  {\tt i.IsDone();} & 
  ~ &  
  true iff {\tt i} is past-the-end. &
  ~ 
  \\ 
  \hline
  \\
\end{tabularx}

\W\conceptsubsection{Complexity guarantees}

\conceptsubsection{Refinements}
\sectionlink{\concept{Vertex}-On-Cell Iterator}{Vertex-On-CellIterator}
\\
\sectionlink{\concept{Edge}-On-Cell Iterator}{Edge-On-CellIterator}
\\
    
\W\conceptsubsection{Models}

\conceptsubsection{Notes}
\begin{enumerate}
\item 
\notelabel{note-isdone}
For circular sequences, such as the vertices around a cell, there is no natural,
predefined past-the-end value. Therefore it is more natural to let the iterator
itself decide when it is invalid, instead of the standard comparison with a
past-the-end iterator (which is defined as well, in order to be able to use
STL algorithms).
\end{enumerate}

\conceptsubsection{See also}
\sectionlink{\concept{Grid}}{Grid} ~
\sectionlink{\concept{Grid Element}}{GridElement} ~
\sectionlink{\concept{Grid Sequence Iterator}}{GridSequenceIterator} ~


  


\xname{GridVertexIterator}
\begin{Label}{GridVertexIterator}
 \conceptsection{Grid Vertex (Edge, Facet, Cell ...) Iterator Concept}
\end{Label}


\conceptsubsection{Description}
A \concept{Grid Vertex Iterator} allows iteration over the vertices
of a \conceptlink{Vertex Grid Range}{VertexGridRange}.

Analogous concepts exist for edges, facets and cells; 
these are not listed seperately.

\conceptsubsection{Refinement of}
\conceptlink{Grid Sequence Iterator}{GridSequenceIterator}

\conceptsubsection{Notation}
\conceptsubsection{Definitions}
\conceptsubsection{Associated types}
\begin{tabularx}{12cm}{llR} \\
  \hline
  \bf  Name  &\bf  Expression  &\bf  Description  \\ 
  \hline
  \\ 
  vertex type  & 
  {\tt I::Vertex} &
   type of the underlying vertex,
   \par   model of \conceptlink{Grid Vertex}{GridVertex}
   \par identical to \code{V::element\_type}
  \\ 
  \hline
  \\
\end{tabularx}

\conceptsubsection{Valid Expressions}
\conceptsubsection{Expression semantics}
\conceptsubsection{Invariants}
\conceptsubsection{Refinements}

\conceptsubsection{Models}
\sectionlink{Complex2D::VertexIterator}{Complex2DVertexIterator},
defined in 
\gralcodelink{vertex2d-it.h}{complex2d}

\conceptsubsection{Notes}
\conceptsubsection{See also}
\conceptlink{Grid Sequence Iterator}{GridSequenceIterator}
~
\conceptlink{Vertex Grid Range}{VertexGridRange}


 \xname{Vertex-On-CellIterator}
\begin{Label}{Vertex-On-CellIterator}
\conceptsection{Vertex-On-Cell (-Facet, ...) Iterator Concept}
\end{Label}

\conceptsubsection{Description}
A {\em Vertex-On-Cell Iterator\/} refines the concept of 
\sectionlink{\concept{Grid Incidence Iterator }}{GridIncidenceIterator} :
It allows to access all vertices incident to a given cell.
Thus, it can also be seen as a 
\sectionlink{\concept{Grid Sequence Iterator}}{GridSequenceIterator} 
over the 
\sectionlink{\concept{ Vertex Grid Range}}{VertexGridRange} of that cell.

Similar iterators can be defined for other anchor elements, most notably facets.
For these, just replace {\em cell\/} with {\em facet}.
\conceptsubsection{Refinement of}
\sectionlink{\concept{ Incidence Iterator}}{GridIncidenceIterator}
\\
%\sectionlink{\concept{Sequence Iterator}}{GridSequenceIterator}

\conceptsubsection{Notation}
{\tt V} is a model of Vertex-On-Cell Iterator
\\ 
{\tt v} is an  object of type  {\tt V}
\\
{\tt c} is an  object of type  {\tt V::Cell}

\conceptsubsection{Associated types}
The types {\tt V::element\_type} and {\tt V::anchor\_type}
can now be named more  specifically. The names from the incidence iterator concept
remain valid.

\noindent
\begin{tabular}{llp{7cm}} 
  \\
  \hline
  \bf  Name &\bf  Expression  &\bf  Description 
  \\ 
  \hline
  Vertex type  &
  {\tt V::Vertex} &
  synomym to {\tt V::element\_type} 
  \\ 
  Cell type  &
  {\tt V::Cell} &
  the cell type {\tt V} operates upon,
  \par model of \sectionlink{\concept{ Grid Cell }}{GridCell}
  \\ 
  Anchor type  & 
  {\tt V::anchor\_type} &
  synonym to  {\tt V::Cell},
  \par model of \sectionlink{\concept{Vertex Grid Range}}{VertexGridRange}  
  \\ 
  \hline
  \\
\end{tabular}
    
\conceptsubsection{Valid Expressions}

\begin{tabular}{llll} \\
  \hline
  \bf  Name  &\bf  Expression  &\bf  Type requirements  & \bf  return type  \\ 
  \hline
  anchor cell  &
  {\tt v.TheCell();} &
  ~ &
  {\tt V::Cell const\&} 
  \\ 
  \hline
  \\
\end{tabular}


\conceptsubsection{Expression semantics}

\begin{tabularx}{14cm}{RRRRR} \\
  \hline     
  \bf  Name     &
  \bf  Expression &
  \bf  Precondition&
  \bf  Semantics &
  \bf  Postcondition
  \\
  \hline
  anchor cell  &
  {\tt Cell C = v.TheCell();} &
  {\tt v is not singular} &
  get the anchor of {\tt v} & 
  {\tt c} is a valid cell; 
  \par
  {\tt c == v.TheCell()}
  \\ 
  \hline
  \\
\end{tabularx}

\W\conceptsubsection{Refinements}

\conceptsubsection{Models}
\sectionlinkUNDEF{{\tt Complex2D::VertexOnCellIterator}}{Complex2DVertexOnCellIterator}, 
defined in  
\xlink{{\tt vertex-on-cell2d-it.h}}{\NMWRROOT/include/Grids/Complex2D/vertex-on-cell2d-it.h } 

\W\conceptsubsection{Notes}


\conceptsubsection{See also}
\sectionlink{\concept{ Grid }}{Grid} ~
\sectionlink{\concept{ Grid Range }}{GridRange} ~
\\
\sectionlink{\concept{ Grid Element }}{GridElement} ~
\sectionlink{\concept{ Grid Vertex }}{GridVertex} ~
\sectionlink{\concept{ Grid Cell}}{GridCell} ~
\\
\sectionlink{\concept{Grid Sequence Iterator }}{GridSequenceIterator} ~
\sectionlink{\concept{Grid Incidence Iterator }}{GridIncidenceIterator} ~
  



% ranges
\section{Grids and Grid Ranges}
\xname{GridRange}
\begin{Label}{GridRange}  
  \conceptsection{Grid Range Concept}
\end{Label}

    \conceptsubsection{Description}
    A {\em  Grid Range} is a part of a grid, its {\em  base grid}.
    The underlying mathematical concept
    is that of (a subset of) a (finite) 
    \xlink{CW-complex}{http://www.treasure-troves.com/math/CW-Complex.html}.
    Some well-known specializations of this concept are
    \xlink{triangulations}{http://www.treasure-troves.com/math/Triangulation.html}
    boundary complexes of convex 
    \xlink{polytopes}{http://www.treasure-troves.com/math/Polytope.html} 
    and regular Cartesian grids.

    A Grid Range behaves in most circumstances like a 
%    \xlink{Grid}[~(section \ref{Grid})~]{Grid.html}. 
    \sectionlink{\concept{Grid}}{Grid}
    The main difference is that a Grid Range has reference semantics with respect to
    its underlying base grid, that is, the incidence relationship is determined by the 
    base grid.
    This influences the behaviour of
    \sectionlink{\concept{Incidence Iterators}}{GridIncidenceIterator}
    associated with a Grid Range,
    which may  visit 
    \sectionlink{\concept{grid elements}}{GridElement}
    that are contained in the base grid,
    but not  in the grid range    
    \noteref{note-incidence}
%    \link*{\arabic{\value{notecounter}}}[\Ref]{note-incidence}).
    
    A Grid is a special case of a Grid 
    Range \noteref{note-reference} %\link*{\ref{note-reference}}[~(see note \Ref)~]{note-reference}.
    \W\\
    {\em  NOTE:} A grid range as such does offer almost no functionality at all.
    Any useful model will be a specialization of one or more 
    element ranges 
    \noteref{note-element-types}
    %\link*{\ref{note-element-types}}[~(see note \Ref)~]{note-element-types}
    like     
    \sectionlink{\concept{Grid Vertex Range}}{VertexGridRange}.

    \conceptsubsection{Refinement of}
     STL \Stllink{Assignable}


    \conceptsubsection{Notation }
    {\tt  R} is a type which is a model of grid range
    \\
    {\tt  r} is an object of type {\tt  R}
    \\
    {\tt G} is {\tt R::grid\_type}

    \conceptsubsection{Associated types} %\noteref{note-gridtypes}

    \begin{tabular}{lll} \\
      \hline
      \bf  Name  &\bf  Expression  &\bf  Description   \\ 
      \hline
      base grid &
      {\tt  R::grid\_type}  &
      type of the  ranges' base grid \noteref{note-gridtypes}
      \\ 
      \hline
      \\
    \end{tabular}
 

    \conceptsubsection{Valid Expressions }
   \begin{tabular}{llll} \\
     \hline
       \bf  Name  &\bf  Expression  &\bf  Type requirements  & \bf  return type  \\ 
       \hline
       base grid reference  &
       {\tt  r.TheGrid()}  &
       & 
       {\tt  grid\_type const\&}  \\ 
       \hline
       \\
    \end{tabular}

   
    \conceptsubsection{Expression semantics }
    \begin{tabularx}{14cm}{RRRRR} \\
      \hline
      \bf  Name       &
      \bf  Expression &
      \bf  Precondition&
      \bf   Semantics &
      \bf   Postcondition
      \\
      \hline
      base grid   reference  &
      {\tt G const\&  g = r.TheGrid()}  &
      {\tt r } has not been default constructed &
      get the grid {\tt  r} references  &
      {\tt  \&g == \&(g.TheGrid())} {\tt == \&(r.TheGrid())}  \\ 
    \hline
    \\
  \end{tabularx}

  \W\conceptsubsection{Complexity guarantees }

    \conceptsubsection{Refinements}
    \sectionlink{\concept{Vertex Grid Range}}{VertexGridRange} ~
    \W\\
    \sectionlink{\concept{Edge Grid Range}}{VertexGridRange} ~
    \W\\
    \sectionlink{\concept{Facet Grid Range}}{VertexGridRange} ~
    \W\\
    \sectionlink{\concept{Cell Grid Range}}{VertexGridRange} ~
    \W\\
    
    \conceptsubsection{Models }
    \gralclasslink{enumerated\_subrange}{base} ~
    \W\\
    \gralclasslink{Complex2D}{complex2d} ~
    \W\\
    \gralclasslink{IstreamComplex2DFmt}{base}

    \conceptsubsection{Notes}
    \begin{enumerate}
    \item       \notelabel{note-reference}
      If {\tt  R} is a model of
      \sectionlinkshort{\concept{Grid}}{Grid},
      then {\tt  R::grid\_type} is identical to {\tt  R}.
      An object {\tt  r} of type {\tt R} 
      then references itself via {\tt  r.TheGrid()}, 
      that is, it has value semantics.
          
      \item   \notelabel{note-gridtypes}
        Technically, these types are bundled in a struct {\tt  grid\_types<R>}
      which is used by the algorithms to access these types. This opens up the possibility
      to parameterize algorithms by such a {\em  traits class} like {\tt  grid\_types<R>},
      thereby introducing different iterator and element types, for example counting iterators
      or debug iterators producing graphical output.
    
      In this case, it would be more precise to say that one 
      ``associates types with {\tt  R}'', 
      instead of speaking of ``types associated with {\tt  R}''.

      
      \item \notelabel{note-incidence}
        This may seem as an odd behaviour. However, this faithfully reflects what many
      locally operating grid-based algorithms are intuitively expected to do when
      given a proper subrange of a grid: The range restricts the region where some
      work is to be done, but on each element, the algorithm accesses also some
      meighboring elements via incidence iterators, which may or may not belong to the
      range. The most striking example for this occurs if grids are distributed with
      some overlap: On each part, the algorithm works only on the locally owned range, 
      but it accesses also elements in the overlap which are copied from other parts.

     
      \item 
        \notelabel{note-element-types}
        It is not necessary to require all possible element types to be defined.
      Useful examples are Input Grids, 
      such as   \gralclasslink{IstreamComplex2DFmt}{base}
      which are used just to read a grid from
      a specific file format.
    \end{enumerate}

    \conceptsubsection{See also }
    \sectionlink{\concept{Grid}}{Grid} ~
    \sectionlink{\concept{Grid Element}}{GridElement} ~
    \sectionlink{\concept{Sequence Iterator}}{GridSequenceIterator} ~
    \sectionlink{\concept{Incidence Iterator}}{GridIncidenceIterator} 



\xname{Grid}
\begin{Label}{Grid}
\conceptsection{Grid Concept}
\end{Label}

\conceptsubsection{Description }
The  mathematical concept underlying {\em  Grid }    
is that of (a subset of) a (finite)     
\xlink{CW-complex}[\cite{weingram}]{http://www.treasure-troves.com/math/CW-Complex.html}. 
Some well-known specialization of this concept are    
triangulations, boundary complexes of convex  
\xlink{polytopes}{http://www.treasure-troves.com/math/Polytope.html} 
and Cartesian grids.     

\conceptsubsection{Refinement of }
\sectionlink{\concept{Grid Range}}{GridRange}  

The main difference to grid ranges is that grids stand for their own ---
there is no underlying base grid. 
This means that all grid entities produced by calls to member functions of a grid      
{\tt  g} refer to {\tt  g} with their grid anchor references.    

Virtually all algorithms can do with grid ranges, they do not require grids.    

\conceptsubsection{Notation }
{\tt  G} is a type which is a model of grid      

\conceptsubsection{Associated types}
    
\begin{tabular}{ccc} 
  \T \\   \hline
  {\bf  Name  } & {\bf  Expression  } & {\bf  Description  } \\ 
  \hline 
  base grid  & {\tt  G::grid\_type}  & identical to {\tt  G }  
  \T \\   \hline  \\
\end{tabular}


\conceptsubsection{Refinements}
\conceptlink{Grid-With-Boundary}{Grid-With-Boundary}
\conceptsubsection{Models}
\gralclasslink{Triang2D}{triang2d}
\\
\gralclasslink{Complex2D}{complex2d}

\conceptsubsection{Notes}
 \begin{enumerate}
   \item Technically, these types are bundled in a struct 
     {\tt  grid\_types<G>}      
     which is used by the algorithms to access these types. 
     This opens up the possibility  to parameterize algorithms by such a {\em  traits class } 
     like {\tt  grid\_types<G> },      
     thereby introducing different iterator and element types, 
     for example counting iterators or debug iterators producing graphical output.    
\end{enumerate}

\conceptsubsection{See also }
\sectionlink{\concept{Grid  Range}}{GridRange} ~
\sectionlink{\concept{Grid Element}}{GridElement} ~ 
\sectionlink{\concept{Grid Sequence Iterator}}{GridSequenceIterator} 

    \xname{Cell-VertexInputGridRange}
\begin{Label}{Cell-VertexInputGridRange}
\conceptsection{Cell-Vertex Input Grid Range}
\end{Label}

\conceptsubsection{Description}
A {\em Cell-Vertex Input Grid Range\/} is a Grid Range which can be used to construct
another grid. More precisely, its representation is cell-based:
Supported is mostly iteration over cells and over vertices incident to cells.

\conceptsubsection{Refinement of}
\sectionlink{\concept{ Cell Grid Range }}{CellRange}
\\
\sectionlink{\concept{ Vertex Grid Range }}{VertexRange}

\conceptsubsection{Associated types}\noteref{note-gridtypes}

\begin{tabularx}{12cm}{llR} \\
  \hline
  \bf  Name  &\bf  Expression  &\bf  Description   \\
  \hline
  cell type &
  {\tt R::Cell} &
  model of \sectionlink{\concept{ Grid Cell }}{GridCell} 
  defining 
  \par {\tt R::Cell::VertexIterator}
  \par (a model of \sectionlink{\concept{VertexOnCellIterator}}{Vertex-On-CellIterator}) 
  \\ 
  \hline
  \\
\end{tabularx}

\begin{ifhtml}

\conceptsubsection{Valid Expressions}

None, besides those defined in 
\sectionlink{\concept{ Cell   Grid Range }}{CellGridRange}
and
\sectionlink{\concept{ Vertex Grid Range }}{VertexGridRange}

\conceptsubsection{Expression semantics}
None, besides those defined in 
\sectionlink{\concept{ Cell   Grid Range }}{CellGridRange}
and 
\sectionlink{\concept{ Vertex Grid Range }}{VertexGridRange}

\end{ifhtml}

\conceptsubsection{Models}
\sectionlink{{\tt enumerated\_grid\_range}}{EnumeratedGridRange}
\\
\sectionlink{{\tt Complex2D}}{Complex2D}
\\
\gralclasslink{{\tt IstreamComplex2DFmt}}{base}
    
\W\conceptsubsection{Notes}

\conceptsubsection{See also}
\sectionlink{\concept{ Grid Range }}{GridRange} ~ 
\sectionlink{\concept{ Cell   Grid Range }}{CellGridRange} ~
\sectionlink{\concept{ Vertex Grid Range }}{VertexGridRange} ~
\sectionlink{\concept{ VertexOnCellIterator }}{Vertex-On-CellIterator}
  


\xname{VertexGridRange}
\begin{Label}{VertexGridRange}
\conceptsection{Vertex (Edge, Cell, ...) Grid Range Concept}
\end{Label}

\conceptsubsection{Description}
A {\em Vertex Grid Range\/} is a sequence of objects of type {\tt V}
which is a model of 
\sectionlink{\concept{Grid Vertex}}{GridVertex}.

Virtually the same definitions can be made for the other element types,
replacing {\tt Vertex} with {\tt Edge, Cell} etc.
Therefore, Vertex Grid Range is chosen to stand {\em sui generis\/} for 
Edge Grid  Range, Cell  Grid Range and so on.

The Vertex Grid Range concept is seldom useful as such;
its primary use is to
define further refinements of the basic 
\sectionlink{\concept{Grid Range}}{GridRange} concept: 
Useful concrete grid types generally
will be models of (at least) two different range concepts,
such as Vertex Grid Range and Grid Cell Sequence.

\conceptsubsection{Refinement of}
\sectionlink{\concept{ Grid Range }}{GridRange}

\conceptsubsection{Notation}
{\tt R} is a type which is a model of Vertex Range 
\\
{\tt r} is an object of type {\tt R::grid\_type}
\\
{\tt v} is an object of type  {\tt R::Vertex}
\\
{\tt vi} is an object of type  {\tt R::VertexIterator}

\conceptsubsection{Associated types}
\begin{tabularx}{14cm}{llR} \\ 
  \hline
  \bf  Name  &\bf  Expression  &\bf  Description  \\ \hline
  Vertex & 
  {\tt R::Vertex} &
  type of vertex (model of \sectionlink{\concept{Grid Vertex}}{GridVertex})
  \\ 
  \hline
       
  handle  & 
  {\tt R::vertex\_handle} &
  \sectionlink{\concept{ handle}}{GridElementHandle} type 
  corresponding to {\tt R::Vertex} 
  \par
   model of \sectionlink{\concept{ Grid Vertex Handle }}{GridVertexHandle}
  \par
   ({\tt R::Vertex::handle\_type} is identical to  {\tt R::vertex\_handle}) 
   \\ 
   Vertex iterator  & 
   {\tt R::VertexIterator} &
   type of the  \sectionlink{\concept{Sequence Iterator}}{GridSequenceIterator}
   corresponding to {\tt R::Vertex}
   \par
   model of \sectionlink{\concept{ Grid Vertex Iterator }}{GridVertexIterator}
   \par
    ({\tt R::Vertex} is identical to  {\tt R::VertexIterator::Vertex}) 
   \\ 
   \hline
   \\
 \end{tabularx}

 \conceptsubsection{Valid Expressions}
 \begin{tabularx}{14cm}{RlRR} \\
   \hline
   \bf  Name  &\bf  Expression  &\bf  Type requirements  & \bf  return type  \\ 
   \hline
   start of sequence &
   {\tt vi = r.FirstVertex()} & 
   ~ & 
   {\tt VertexIterator}  
   \\
   end of sequence & 
   {\tt vi = r.EndVertex()} & 
   ~ & 
   {\tt VertexIterator} 
   \\
   size of sequence\noteref{note-size-of-sequence}  & 
   {\tt r.NumOfVertices()} & 
   ~ &
   {\tt size\_type} 
   \\
   handle-to-element conversion &
   {\tt v = r.vertex(h)} & 
   ~ &
   {\tt Vertex} 
   \\
   element-to-handle  conversion   &
   {\tt h = r.handle(v)} & 
   ~ &
   {\tt vertex\_handle} 
   \\
   \hline
   \\
 \end{tabularx}

 \conceptsubsection{Expression semantics}

\T\begin{small}
 \begin{tabularx}{16cm}{RRRRR} \\
   \hline
   \bf  Name    &
   \bf  Expression &
   \bf  Precondition&
   \bf   Semantics &
   \bf   Postcondition
   \\ 
   \hline
   start of  sequence &
   {\tt v = r.FirstVertex()} &
   ~    &
   return iterator pointing to the first vertex of {\tt s} &
   {\tt v} is \footlink{valid}{valid}, or  {\tt v == r.EndVertex()}  
   \\
   end of sequence &
   {\tt v = r.EndVertex()} &
   ~    &
   ~    &
   {\tt v} is past-the-end.
   \\
   size of sequence & 
   {\tt n = r.NumOfVertices()} & 
   ~ &
   get the number of vertices of {\tt s} &
   {\tt n == \Stllink{distance}(r.FirstVertex(), r.EndVertex())} 
   \\
   handle-to-element &
   v = r.vertex(h) &
   {\tt h} is a handle of {\tt r.TheGrid()} &
   Construct the vertex corresponding to the handle {\tt h}  & 
   {\tt r.handle(v) == h} 
   \\
   element-to-handle &
   h = r.handle(v) &
   {\tt v} is a vertex of {\tt r.TheGrid()} &
   Construct the vertex corresponding to the handle {\tt h}  & 
   {\tt r.element(h) == v} 
   \\
   \hline
   \\
 \end{tabularx}
\T\end{small}

 \conceptsubsection{Complexity guarantees}
 The {\tt r.NumOfVertices()} operation has complexity at most O(V), where
 V is  the number of vertices of {\tt r} \noteref{note-complexity}
 
 \conceptsubsection{Models}
 Virtually every concrete grid  or grid range:
 \\
 \sectionlink{{\tt enumerated\_grid\_range}}{EnumeratedGridRange}
 \\
 \sectionlink{\concept{Complex}2D}{Complex2D}
 
 \conceptsubsection{Notes}
 \begin{enumerate}
 \item \notelabel{note-complexity}
   The reason why a general Vertex Range cannot guarantee $O(1)$ complexity for
   this operation is the following: 
   Such a sequence  might arise from a grid range
   given by a simple enumeration
   of its cells 
   (or their handles,
   see for example   \sectionlink{{\tt enumerated\_grid\_range}}{EnumeratedGridRange}).
   The elements of lower dimension belonging to {\tt r} are then implicitely defined
   by the closure (in a topological sense) of the set of cells in the underlying grid.
   The determination of, for example, the set of vertices can be done in expected time
   O(V) (using \sectionlink{\concept{partial grid functions}}{PartialGridFunction}).
   Requiring to do this determination at time of construction of the range would 
   impose the cost of doing so also to clients not interested in this functionality.
\item \notelabel{note-size-of-sequence} This functionality can always be implemented
   in terms of the element iterators, in which case it is $O(n)$. In some cases,
   there is no better implementation possible, for example, if edges or facets
   are implicitely ordered by there vertex sets. Therefore, there is no guarantee
   for a better complexity here.
   In these cases, it is not always simple to provide an \tt{EndElement()}
   past-the-end iterator, without compromising efficiency of iterator comparison.
 \end{enumerate}
  


\xname{Grid-With-Boundary}
\begin{Label}{Grid-With-Boundary}
 \conceptsection{Grid-With-Boundary Concept}
\end{Label}

\conceptsubsection{Description}
A \concept{Grid-With-Boundary}  allows to query 
whether a facet is a \Glossarylink{boundary facet}.
Moreover, it defines a special cell handle for an `outside cell', 
and allows to query whether a cell is inside the grid.

\conceptsubsection{Refinement of}
\conceptlink{Grid}{Grid}
\\
\conceptlink{Cell Grid Range}{VertexGridRange}
\\
\conceptlink{Facet Grid Range}{VertexGridRange}
\\
The cell type of a  \concept{Grid-With-Boundary} is a model
of \conceptlink{Grid Cell}{GridCell} and of 
\conceptlink{Facet Grid Range}{VertexGridRange}.


\conceptsubsection{Notation}
\type{G} is a type which is a model of \concept{Grid-With-Boundary}\\
\variable{g} is an object of type \type{G}\\
\variable{c} is an object of type \type{G::Cell} \\
\variable{h} is an object of type \type{G::cell\_handle}\\
\variable{f} is an object of type \type{G::Facet}\\
\variable{fc} is an object of type \type{G::FacetOnCellIterator}

\conceptsubsection{Definitions}

\conceptsubsection{Associated types}
No new ones, but the requirements on \code{G::Cell}
are strengthened.

\begin{tabular}{ccc} \\ 
  \hline
  {\bf  Name  } & {\bf  Expression  } & {\bf  Description  } \\ 
  \hline
  cell type  & \code{G::Cell} &
  the cell element type of \code{G}
  \par model of \conceptlink{Grid Cell}{GridCell}
  and of \conceptlink{Facet Grid Range}{VertexGridRange} 
  \\
  \hline
  \\
\end{tabular}

\conceptsubsection{Valid Expressions}

\noindent
\begin{tabular}{llll} \\
  \hline
  \bf  Name  &\bf  Expression  &\bf  Type requirements  & \bf  return type  \\ 
  \hline
  inside test &
  \code{g.IsInside(c)} &
   ~ & 
  \type{bool} \\
  inside test &
  \code{g.IsInside(h)} &
  ~  & 
  \type{bool} \\
  boundary test &
  \code{g.IsOnBoundary(fc)} &
  ~ & 
  \type{bool} \\
  boundary test&
  \code{g.IsOnBoundary(f)} &
  ~  & 
  \type{bool} \\
  outer cell handle &
  \code{g.outer\_cell\_handle()} &
  ~ &
  \type{G::cell\_handle}
  \\
  \hline
  \\
\end{tabular}

\conceptsubsection{Expression semantics}

\noindent
\begin{tabularx}{14cm}{RRRRR} \\
  \hline
  \bf  Name       &
  \bf  Expression &
  \bf  Precondition&
  \bf   Semantics &
  \bf   Postcondition
  \\
  \hline
  inside test &
  \code{g.IsInside(c);} &
  \variable{c} is \footlink{bound}{bound}   to \variable{g}
  \par
   \variable{c} is  \footlink{valid}{valid} 
   or \code{c.handle() == g.outer\_cell\_handle()} &
   Test if \variable{c} is inside the grid &
   true if \variable{c} is  \link{valid}{valid} for \variable{g}.
   \par
   false if  \code{c.handle() == g.outer\_cell\_handle()}
   \\
   inside test&
  \code{g.IsInside(h);} &
   \variable{h} is  \footlink{valid}{valid} 
   or \code{h == g.outer\_cell\_handle()} &
   Test if \variable{h} is handle of a cell inside the grid &
   true if \variable{h} is  \link{valid}{valid} for \variable{g}.
   \par
   false if  \code{h == g.outer\_cell\_handle()} 
   \\
   boundary test &
   \code{g.IsOnBoundary(f)} &
   \variable{f} is \link{valid}{valid} &
   Returns \code{true} if \variable{f}
   is a \Glossarylink{boundary facet} of \variable{g} &
   ~ 
   \\
   boundary test &
   \code{g.IsOnBoundary(fc)} &
   \variable{fc} is \link{valid}{valid} &
   equivalent to \code{g.IsOnBoundary(*fc)}.& 
   ~ 
   \\
   outer cell handle &
  \code{h = g.outer\_cell\_handle();} &
   ~ &
  return a handle denoting an `outside' cell &
  \code{g.IsInside(h)} is false 
  \\
  \hline
\end{tabularx}

\conceptsubsection{Invariants}

\noindent
\begin{tabularx}{14cm}{RR}
  \\
  \hline
  & \code{g.IsInside(g.outer\_cell\_handle()) == false}\\
  & \code{g.IsInside(c) == g.IsInside(c.handle())} \\
  & \code{g.IsOnBoundary(fc) == ! g.IsInside(fc.OtherCell())}\noteref{note-gwb-othercell} \\
  & \code{g.IsOnBoundary(f) == (!g.IsInside(f.TheCell()) || !g.IsInside(f.OtherCell()))} \\
  & \code{g.IsInside()} evaluates to \code{true} for all
  cells in the range [\code{g.FirstCell()},\code{g.EndCell()}).\\
  \hline
\end{tabularx}

\conceptsubsection{Refinements}

\conceptsubsection{Models}
\sectionlink{\type{Complex2D}}{Complex2D}   

\conceptsubsection{Notes}
\begin{enumerate}
\item \notelabel{note-gwb-outer-non-unique}
It does however not follow that if 
\code{! g.IsInside(h)}, then
{\code h == g.outer\_cell\_handle()}, 
because it is possible that there are many different handles
denoting an outer cell. For example, when we consider Cartesian grids,
cells with an logical index $(i,j)$ outside the grid range would 
satisfy \code{! g.IsInside(c)}
but do not necessarily possess a handle equal to
{\code g.outer\_cell\_handle()}.

\item  \notelabel{note-gwb-othercell}
It is possible to create a cell with the `outer' cell handle:
\begin{example}
 Cell c = g.cell(g.outer_cell_handle())
\end{example}
is possible, as is 
\begin{example}
Cell c = fc.OtherCell()
\end{example} 
when \code{g.IsOnBoundary(fc)} is true.
However, the only operation guaranteed to work with \variable{c} is
the query \code{g.IsInside(c)} which evaluates to \code{false}.
\end{enumerate}
\conceptsubsection{See also}




\T\end{small}

% grid functions
\section{Grid Functions}
\xname{GridElementFunction}
\begin{Label}{GridElementFunction}
  \conceptsection{Grid Element Function Concept}
\end{Label}

\conceptsubsection{Description}
The {\em Grid Element Function\/} concept 
models the mathematical concept of a mapping from
grid elements of some fixed type (vertex, edge, cell)
which is a model of \sectionlink{\concept{Grid Element}}{GridElement},
to values of  some type {\tt T}.

\conceptsubsection{Refinement of}
STL \stllink{Adaptable Unary Function}{AdaptableUnaryFunction}

\conceptsubsection{Notation}
{\tt F} is a type which is a model of  Grid Element Function 
\\
{\tt f} is an object of type  {\tt F}
\\
{\tt e, e1, e2} are objects of  {\tt F::element\_type}
\\
{\tt t} is an object of  {\tt F::value\_type}

\conceptsubsection{Associated types}

\noindent
\begin{tabularx}{14cm}{llR}  
  \hline
  \bf  Name  &\bf  Expression  &\bf  Description   \\ 
  \hline
  element type  & 
  {\tt F::element\_type} &
  type of the underlying element, 
  \par model of \sectionlink{\concept{ Grid Element}}{GridElement} 
  \par synonym for {\tt F::argument\_type}
  (from \stllink{Adaptable Unary Function}{AdaptableUnaryFunction})
  \\
  value type  &
  {\tt F::value\_type} &
  synomym for {\tt F::result\_type}   
  \par (from \stllink{Adaptable Unary Function}{AdaptableUnaryFunction})
  \\ 
  \hline
  \\
\end{tabularx}
    
\conceptsubsection{Valid Expressions}
No new ones, besides those from 
\stllink{Adaptable Unary Function}{AdaptableUnaryFunction}

\noindent
\begin{tabular}{llll} \\
  \hline
  \bf  Name  &\bf  Expression  &\bf  Type requirements  & \bf  return type  \\ 
  \hline
  function evaluation  &
  {\tt t = f(e);} &
  ~      &
  {\tt value\_type} 
  \\ 
  \hline
  \\
\end{tabular}

\conceptsubsection{Expression semantics}
The semantics of function evaluation are more restrictive than those for
\stllink{Adaptable Unary Function}{AdaptableUnaryFunction}
see the notes below\noteref{note-evaluation}.

\noindent
\begin{tabularx}{14cm}{llRRR} \\
  \hline
  \bf  Name       &
  \bf  Expression &
  \bf  Precondition&
  \bf  Semantics &
  \bf  Postcondition
  \\ 
  \hline
  evaluation  &
  {\tt t = f(e)} &
  {\tt e} is in the domain of {\tt f} &
  evaluate {\tt f} at the argument {\tt e} & 
  {\tt t} is equal to {\tt f(e)}\noteref{note-evaluation}
  \\ 
  \hline
  \\
\end{tabularx}

\conceptsubsection{Invariants}
\begin{tabular}{ll} 
  Argument identity &
  if {\tt e1 == e2} then {\tt f(e1)} is equal to {\tt f(e2)} \noteref{note-equal-comp}
  \\ 
\end{tabular}

\conceptsubsection{Refinements}
\sectionlink{\concept{ Grid Function }}{GridFunction}

\conceptsubsection{Models}
\sectionlink{{\tt cell\_nb\_degree<GRID>}}{cell_nb_degree}
defined in
\gralfilelink{grid-functors}{h}{base}.

\conceptsubsection{Notes}
\begin{enumerate}
\item \notelabel{note-evaluation}
  The important difference to STL function objects
  is that the latter are {\em not\/} guaranteed to deliver the same result 
  for subsequent evaluations on the same argument.
\item \notelabel{note-equal-comp}    
  The type {\tt F::value\_type} is not required to be 
  STL \stllink{Equality Comparable}{EqualityComparable}
  If it is, then {\tt e1 == e2} implies {\tt f(e1) == f(e2)}.
\end{enumerate}
    
\conceptsubsection{See also}
\sectionlink{\concept{ Grid Function }}{GridFunction} ~

  


\xname{GridFunction}
\begin{Label}{GridFunction}
\conceptsection{Grid Function Concept}
\end{Label}

\conceptsubsection{Description}

The  {\em Grid Function\/} concept refines the 
\sectionlink{\concept{ Grid Element Function}}{GridElementFunction} concept
in that it bind the function
to a particular grid. 
Evaluating  a grid function without a valid grid set or
with an element whose grid is different from that of the grid function
is considered an error.

By binding to a particular grid, it is possible to treat both domain
and range of a grid function as  sequences with associated iterators.

\conceptsubsection{Refinement of}
\sectionlink{\concept{Grid Element Function}}{GridElementFunction}

\conceptsubsection{Notation}
{\tt F} is a type which is a model of  Grid  Function 
\\
{\tt f} is an object of type  {\tt F}
\\
{\tt i} is an object of  {\tt F::const\_iterator}
\\
{\tt e} is an object of  {\tt F::element\_type}
\\
{\tt n} is an object of  {\tt F::size\_type}
\\
{\tt G} is shorthand for  {\tt F::grid\_type}

\conceptsubsection{Definitions}
\label{bound}
A grid function {\tt f} is {\em bound\/} to a grid {\tt g},
if {\tt \&g == \&f.TheGrid()}. Else  {\tt f} is {\em unbound\/}.

\label{range}    
The {\em range\/} of a grid function {\tt f} is the set of all elements of type
{\tt F::element\_type} in {\tt f.TheGrid()}.

\label{domain}
The {\em domain\/} of a grid function is the set of all values of the form {\tt f(e)} 
where {\tt e} is in the range of {\tt f}.

\conceptsubsection{Associated types}

\begin{tabularx}{14cm}{llR}
  \hline
  \bf  Name  &\bf  Expression  &\bf  Description   \\ 
  \hline
  Grid type & 
  {\tt F::grid\_type} &
  type of the corresponding associated grid,
  \par model of \sectionlink{\concept{ Grid}}{Grid} 
  \\ 
  value iterator &
  {\tt F::const\_iterator} &
  iterator over the values of {\tt f}
  \par model of STL
  \xlink{Forward iterator}{http://www.sgi.com/Technology/STL/ForwardIterator.html}
  \\ 
  size type & 
  {\tt F::size\_type} &
  integral type capable of representing the possible sizes
  of the value sequence.
  \\ 
  \hline
  \\
\end{tabularx}

\conceptsubsection{Valid Expressions}
\begin{tabularx}{14cm}{RRll} 
  \hline
  \bf  Name  &\bf  Expression  &\bf  Type requirements  & \bf  return type  \\
  \hline
  Grid reference  &
  {\tt f.TheGrid();} &
  ~ &
  {\tt F::grid\_type const\&} 
  \\ 
  start of value sequence  &
  {\tt f.begin();} &
  ~ &
  {\tt F::const\_iterator} 
  \\ 
  end of value sequence  &
  {\tt f.end();} &
  ~ &
  {\tt F::const\_iterator} 
  \\ 
  size of value sequence  &
  {\tt f.size();} &
  ~ &
  {\tt F::size\_type} 
  \\ 
  \hline
  \\
\end{tabularx}

\T\begin{small}
\conceptsubsection{Expression semantics}
\W\begin{tabularx}{15cm}{RRRRR} 
\T\begin{tabularx}{15cm}{>{\raggedright\arraybackslash}p{1.5cm}RRRR}
  \hline
  \bf  Name     &
  \bf  Expression &
  \bf  Precondition&
  \bf  Semantics &
  \bf  Postcondition
  \\ 
  \hline
  Grid reference  &
  {\tt G\& g = f.TheGrid();} &
  {\tt f} is bound to a grid  &
  get reference to the underlying grid & 
  {\tt \&(g.TheGrid()) == \&(f.TheGrid())} 
  \\ 
  start of value sequence  &
  {\tt i = f.begin();} &
  {\tt f} is bound to a grid  &
  return iterator to start of value sequence &
  {\tt \xlink{distance}{\STLURL/distance.html} (i,f.end()) == f.size()}
  \\ 
  end of value sequence  &
  {\tt i = f.end();} &
  {\tt f} is bound to a grid  &
  return iterator to past-the-end of value sequence &
  {\tt i} is past-the-end, {\tt i == f.end()}
  \\ 
  size of value sequence  &
  {\tt n = f.size();} &
  {\tt f} is bound to a grid  &
  return size of value sequence &
  {\tt f.size() == 
    \xlink{distance}{\STLURL/distance.html} (f.begin(),f.end())}
  \\ 
  \hline
  \\
\end{tabularx}
\T\end{small}

\conceptsubsection{Refinements}
\sectionlink{\concept{ Mutable Grid Function }}{MutableGridFunction}

\conceptsubsection{Models}
\sectionlink{{\tt cell2handle\_map<G>}}{cell2handle-map} 
defined in
\xlink{{\tt grid-functors.h}}{\NMWRINC/Grids/grid-functors.h} 
(\noteref{note-functors})
\\
\sectionlink{{\tt vertex2coord\_map<Geom>}}{vertex2coord-map}
defined in
\xlink{{\tt geometry-functors.h}}{\NMWRINC/Grids/geometry-functors.h} 
(\noteref{note-functors})

\conceptsubsection{Notes}
\begin{enumerate}
\item \notelabel{note-functors}
       Currently, iteration and size is not supported in these examples.
\end{enumerate}

\conceptsubsection{See also}
\sectionlink{\concept{ Grid Element Function }}{GridElementFunction} ~
  


\xname{MutableGridFunction}
\begin{Label}{MutableGridFunction}
\conceptsection{Mutable Grid Function Concept}
\end{Label}

\conceptsubsection{Description}

The  {\em Mutable Grid Function\/} concept refines the 
\sectionlink{\concept{ Grid Function}}{GridFunction} concept.
It allows to change function values, that is,
to store values on elements.


\conceptsubsection{Refinement of}

\sectionlink{\concept{ Grid Function}}{GridFunction}
 
\conceptsubsection{Notation}
{\tt F} is a type which is a model of  Mutable  Grid  Function 
\\
{\tt f} is an object of type  {\tt F}
\\
{\tt i} is an object of  {\tt F::iterator}
\\
{\tt t} is an object of  {\tt F::value\_type}
\\

\conceptsubsection{Associated types}
\begin{tabularx}{14cm}{RlR} 
  \T \hline
  \bf  Name  &\bf  Expression  &\bf  Description   \\ 
  \hline
  mutable value iterator &
  {\tt F::iterator} &
  iterator over the values of {\tt f}
  \par model of STL \stllink{Forward Iterator}{ForwardIterator}
  \T \\   \hline  \\
\end{tabularx}

\conceptsubsection{Valid Expressions}
\begin{tabular}{llll} 
  \T \hline
  \bf  Name  &\bf  Expression  &\bf  Type requirements  & \bf  return type 
  \\ 
  \hline
  start of value sequence  &
  {\tt f.begin();} &
  ~ &
  {\tt F::iterator} 
  \\
  end of value sequence  &
  {\tt f.end();} &
  ~ &
  {\tt F::iterator} 
  \\
  write access &
  {\tt f[e];} &
  ~ &
  {\tt F::value\_type\&} 
  \T \\  \hline  \\
\end{tabular}

\conceptsubsection{Expression semantics}
\begin{tabularx}{15cm}{RRRRR} 
  \T \hline
  \bf  Name     &
  \bf  Expression &
  \bf  Precondition&
  \bf  Semantics &
  \bf  Postcondition
  \\
  \hline
  start of value sequence  &
  {\tt i = f.begin();} &
  {\tt f} is bound to a grid  &
  return iterator to start of value sequence &
  {\tt  \Stllink{distance}(i,f.end()) == f.size()}
  \\ 
  end of value sequence  &
  {\tt i = f.end();} &
  {\tt f} is bound to a grid  &
  return iterator to past-the-end of value sequence &
  {\tt i} is past-the-end, {\tt i == f.end()}
  \\ 
  write access &
  {\tt f[e] = t;} &
  {\tt f} is bound to a grid
  \par {\tt e.TheGrid() == f.TheGrid()}
  &
  assign the value {\tt t} to {\tt f(e)} &
  {\tt f(e)} is equal to {\tt t}
  \T \\   \hline  \\
\end{tabularx}

\conceptsubsection{Refinements}
\sectionlink{\concept{ Container Grid Function}}{ContainerGridFunction}

\W\conceptsubsection{Models}

\W\conceptsubsection{Notes}
    

\conceptsubsection{See also}
\sectionlink{\concept{ Grid Element Function }}{GridElementFunction} ~
\sectionlink{\concept{ Grid  Function }}{GridFunction} ~

  


\xname{ContainerGridFunction}
\begin{Label}{ContainerGridFunction}
\conceptsection{Container Grid Function Concept}
\end{Label}

\conceptsubsection{Description}

The  {\em Container Grid Function\/} concept refines the 
\sectionlink{\concept{ Mutable Grid Function}}{MutableGridFunction} concept.
A Container Grid Function can be created and filled with values by a client,
much like a ordinary container. This of particular importance for 
algorithms needing temporary storage, such as boolean flags on grid elements.

\conceptsubsection{Refinement of}
\sectionlink{\concept{ Mutable Grid Function}}{MutableGridFunction}
\\
STL \xlink{Assignable}{http://www.sgi.com/Technology/STL/Assignable.html}

\conceptsubsection{Notation}
{\tt F} is a type which is a model of  Container Grid  Function 
\\
{\tt f} is an object of type  {\tt F}
\\
{\tt G} is shorthand for  {\tt F::grid\_type}
\\
{\tt g} is an object of type  {\tt G}.

\conceptsubsection{Associated types}
None, exept those defined in
\sectionlink{\concept{ Mutable Grid Function}}{MutableGridFunction}

\conceptsubsection{Valid Expressions}
\begin{tabular}{llll} 
  \hline
  \bf  Name  &\bf  Expression  &\bf  Type requirements  & \bf  return type  \\ \hline
  Default construction & 
  {\tt F f();} &
  ~ &
  ~ 
  \\ 
  Construction from grid & 
  {\tt F f(g);} &
  ~ &
  ~ 
  \\ 
  Construction and initialization & 
  {\tt F f(g,t);} &
  ~ &
  ~ 
  \\ 
  Binding to grid &
  {\tt f.set\_grid(g);} &
  ~ &
  ~ 
  \\ 
  \hline
\end{tabular}

\conceptsubsection{Expression semantics}
\begin{tabularx}{15cm}{RRRRR} 
  \hline    
  \bf  Name     &
  \bf  Expression &
  \bf  Precondition&
  \bf  Semantics &
  \bf  Postcondition
  \\ 
  \hline
  Default construction & 
  {\tt F f();} &
  ~ &
  default construct {\tt f} &
  {\tt f} is unbound
  \par write access is an error
  \par read  access is an error
  ~
  \\ 
  construction from grid & 
  {\tt F f(g);} &
  ~ &
  construct and bind {\tt f}  to {\tt g} &
  {\tt f} is bound to {\tt g} 
  \par write access is allowed 
  \par read access is undefined
  \\ 
  construction and initialization & 
  {\tt F f(g,t);} &
  ~ &
  construct and bind {\tt f}  to {\tt g}, 
  initialize all values to {\tt t} &
  {\tt f} is bound to {\tt g} 
  \par write access is allowed 
  \par {\tt f(e)} is equal to {\tt t} for all elements {\tt e}
  in the range of {\tt f}.
  \\ 
  Binding to grid &
  {\tt f.set\_grid(g);} &
  f is unbound &
  bind {\tt f}  to {\tt g} &
  {\tt f} is bound to {\tt g} 
  \par write access is allowed, 
  \par read access is undefined
  \\ 
  \hline
  \\
\end{tabularx}

\conceptsubsection{Complexity Guarantees}
Default construction takes constant time.
\\
Construction from grid and construction with initalization both
take time at most O({\tt f.size()}), that is, the number of 
elements of type {\tt F::element\_type} of {\tt g}.
 

\conceptsubsection{Refinements}
\sectionlink{\concept{ Total Grid Function}}{TotalGridFunction} ~
\W\\
\sectionlink{\concept{ Partial Grid Function}}{PartialGridFunction}

\W\conceptsubsection{Models}
    

\W\conceptsubsection{Notes}

 
\conceptsubsection{See also}
\sectionlink{\concept{ Grid Element Function }}{GridElementFunction} ~
\sectionlink{\concept{ Grid  Function }}{GridFunction} ~
\sectionlink{\concept{ Mutable Grid  Function }}{MutableGridFunction} ~

  


  
\xname{TotalGridFunction}
\begin{Label}{TotalGridFunction}
  \conceptsection{Total Grid Function Concept}
\end{Label}

\conceptsubsection{Description}

The  {\em Total Grid Function\/} concept refines the 
\sectionlink{\concept{ Container Grid Function}}{ContainerGridFunction} concept.
A total grid function reserves storage to hold a value for each element in
its range.

\conceptsubsection{Refinement of}
\sectionlink{\concept{ Container Grid Function}}{ContainerGridFunction}

\conceptsubsection{Notation}
{\tt F} is a type which is a model of Total  Grid  Function 
\\
{\tt f} is an object of type  {\tt F}
\\
{\tt G} is shorthand for  {\tt F::grid\_type}
\\
{\tt g} is an object of type  {\tt G}.

\begin{ifhtml}
\conceptsubsection{Associated types}
None, exept those defined in
\sectionlink{\concept{ Container Grid Function}}{ContainerGridFunction}

\conceptsubsection{Valid Expressions}
None, exept those defined in
\sectionlink{\concept{ Container Grid Function}}{ContainerGridFunction}
\end{ifhtml}

\conceptsubsection{Expression semantics}
\begin{tabularx}{15cm}{RRRRR}
  \hline    
  \bf  Name     &
  \bf  Expression &
  \bf  Precondition&
  \bf  Semantics &
  \bf  Postcondition
  \\ 
  \hline
  construction from grid & 
  {\tt F f(g);} &
  ~ &
  construct and bind {\tt f}  to {\tt g},
  allocate memory for {\tt f.size()} values. &
  {\tt f} is \link{bound}[\footnote{see \Ref}]{bound} to {\tt g} 
  \par write access is allowed 
  \par read access is undefined
  \par {\tt f.size()} is equal to the cardinality of the 
  \link{range}[\footnote{see \Ref}]{range}  of {\tt f}
  \\ 
  construction and initialization & 
  {\tt F f(g,t);} &
  ~ &
  construct and bind {\tt f}  to {\tt g}, 
  \par allocate memory for {\tt f.size()} values,
  \par initialize all values to {\tt t} 
  &
  {\tt f} is bound to {\tt g} 
  \par write access is allowed 
  \par {\tt f(e)} is equal to {\tt t} for all elements {\tt e}
  in the range of {\tt f}.
  \par {\tt f.size()} is equal to the cardinality of the 
  \link{range}{range} of {\tt f}
  \\ 
  Binding to grid &
  {\tt f.set\_grid(g);} &
  f is unbound &
  bind {\tt f}  to {\tt g},
  \par allocate memory for {\tt f.size()} values. &
  {\tt f} is bound to {\tt g} 
  \par write access is allowed, 
  \par read access is undefined
  \par {\tt f.size()} is equal to the cardinality of the 
  \link{range}{range} of {\tt f}
  \\ 
  \hline
  \\
\end{tabularx}

\conceptsubsection{Complexity Guarantees}
Default construction takes constant time.
\\
Construction from grid and construction with initalization both
take time at  O({\tt f.size()}), that is, the number of 
elements of type {\tt F::element\_type} of {\tt g}.



\conceptsubsection{Models}
\sectionlink{{\tt grid\_function\_vector<E,T>}}{grid-function-vector}
defined in
\gralcodelink{grid-function-vector.h}{base}

Total grid functions for the
\sectionlink{{\tt Complex2D}}{Complex2D}
concrete grid, defined in
\gralcodelink{grid-functions.h}{complex2d}

For {\tt E = \sectionlink{\concept{Complex}2D::Vertex}{Complex2D::Vertex}} 
and {\tt E = Complex2D::Cell},
the total grid functions are derived from
\sectionlink{{\tt grid\_function\_vector<E,T>}}{grid-function-vector};
and for  {\tt E = Complex2D::Edge}, 
the total grid function is derived from 
\sectionlink{{\tt grid\_function\_hash<E,T>}}{grid-function-hash}.
The reason  is taht edges are not stored in the 
\sectionlink{{\tt Complex2D}}{Complex2D}
data structure, and hence there is no consecutive index available for type  
{\tt Complex2D::Edge}.


\W\conceptsubsection{Notes}
    

\conceptsubsection{See also}
\sectionlink{\concept{ Grid Element Function }}{GridElementFunction} ~
\sectionlink{\concept{ Grid  Function }}{GridFunction} ~
\sectionlink{\concept{ Mutable Grid  Function }}{MutableGridFunction} ~
\sectionlink{\concept{ Container Grid  Function }}{ContainerGridFunction} ~
\sectionlink{\concept{ Partial Grid Function }}{PartialGridFunction} ~

  


\xname{PartialGridFunction}
\begin{Label}{PartialGridFunction}
\conceptsection{Partial Grid Function Concept}
\end{Label}

\conceptsubsection{Description}

The  {\em Partial Grid Function\/} concept refines the 
\sectionlink{\concept{ Container Grid Function}}{ContainerGridFunction} concept.
A partial grid function reserves storage only for those values which are
write-accessed explicitly.
This is in contrast to 
\sectionlink{\concept{ Total Grid Function }}{TotalGridFunction} 
which allocates storage for each value.

Partial grid functions are of particular importance for locally operating 
algorithms with sublinear runtime and memory requirements.

\conceptsubsection{Refinement of}
\sectionlink{\concept{ Container Grid Function}}{ContainerGridFunction}

\conceptsubsection{Notation}
{\tt F} is a type which is a model of  Partial Grid  Function 
\\
{\tt f} is an object of type  {\tt F}
\\
{\tt G} is shorthand for  {\tt F::grid\_type}
\\
{\tt g} is an object of type  {\tt G}.

\begin{ifhtml}
\conceptsubsection{Associated types}
None, exept those defined in
\sectionlink{\concept{ Container Grid Function}}{ContainerGridFunction}
\end{ifhtml}

\conceptsubsection{Valid Expressions}
\begin{tabular}{llll} 
  \hline
  \bf  Name  &\bf  Expression  &\bf  Type requirements  & \bf  return type  \\ 
  \hline
  Definition test  &
  {\tt f.defined(e);} &
  ~ &
  {\tt bool} 
  \\ 
  Definition undo &
  {\tt f.undefine(e);} &
  ~ &
  ~
  \\ 
  Default setting &
  {\tt f.set\_default(t);} &
  ~ &
  ~ 
  \\ 
  Default access &
  {\tt f.default();} &
  ~ &
  F::value\_type 
  \\ 
  \hline
  \\
\end{tabular}


\conceptsubsection{Expression semantics}
\begin{tabularx}{15cm}{RRRRR}
  \hline      
  \bf  Name    &
  \bf  Expression &
  \bf  Precondition&
  \bf  Semantics &
  \bf  Postcondition
  \\ 
  \hline
    construction from grid & 
    {\tt F f(g);} &
    ~ &
    construct and bind {\tt f}  to {\tt g} &
    {\tt f} is bound to {\tt g} 
    \par write access is allowed 
    \par read access is undefined
    \par {\tt f.size()} is equal to the cardinality of the 
    \link{range}[\footnote{see \Ref}]{range} of {\tt f}
  \\ 
    construction and initialization & 
    {\tt F f(g,t);} &
    ~ &
    construct and bind {\tt f}  to {\tt g}, 
    \par initialize the default value to {\tt t} 
    &
    {\tt f} is bound to {\tt g} 
    \par write access is allowed 
    \par {\tt f(e)} is equal to {\tt t} for all elements {\tt e}
    in the range of {\tt f}.
    \par {\tt f.size()} is equal to the cardinality of the 
    \link{range}{range} of {\tt f}
  \\ 
    Binding to grid &
    {\tt f.set\_grid(g);} &
    f is unbound &
    bind {\tt f}  to {\tt g} &
    {\tt f} is bound to {\tt g} 
    \par write access is allowed, 
    \par read access is undefined
    \par {\tt f.size()} is equal to the cardinality of the 
    \link{range}{range} of {\tt f}
   \\
    Definition test &
    {\tt if(f.defined(e))} &
    ~ &
    true if {\tt f[e]} has been explicitely set.&
    ~ 
   \\
    Definition undo &
    {\tt f.undefine(e)} &
    {\tt f.defined(e)} &
    remove {\tt e} from the set of explicitely defined values &
    {\tt f(e)} is equal to {\tt f.default()} 
    \par {\tt f.defined(e) == false}
   \\
    Default setting &
    {\tt f.set\_default(t)} &
    ~ &
    set default value to {\tt t} &
    {\tt ! f.defined(e)} implies{\tt f(e)} is equal to {\tt t}
    \par {\tt f.default()} is equal to {\tt t}
   \\
    Default access &
    {\tt t = f.default()} &
    default has been set by constructor {\tt f(g,t)} 
    or by  {\tt f.set\_default(t)} &
    set {\tt t} to  default value of {\tt f} &
    {\tt t} is equal to {\tt f.default()} 
   \\
    write access &
    {\tt t = f[e];} &
    {\tt f} is bound to a grid
    \par {\tt e.TheGrid() == f.TheGrid()} &
    create storage for a value\noteref{note-create-storage}
    if  {\tt ! f.defined(e)};
    \par assign  {\tt f(e)} to  {\tt t}. &
    {\tt f(e)} is equal to {\tt t}
   \\
   \hline
   \\
 \end{tabularx}

 \conceptsubsection{Invariants}
 \begin{tabular}{ll} 
   \hline
   Default value &
   If not  {\tt f.defined(e)} then  {\tt f(e)} is equal to {\tt f.default()}  
   \\ 
   \hline
 \end{tabular}

 \conceptsubsection{Complexity Guarantees}
 Default construction takes constant time.
 \\
 Construction from grid and construction with initalization both
 take constant time.
 \\
 The {\tt set\_default(t)} operation takes constant time.
 \\
 The access operations {\tt f(e), f[e]} and the test  {\tt f.defined(e)}
 take at most logarithmic time, or amortized constant time, or expected constant time.
 \\
 The memory requirements are at most proportional to the total number of different
 elements {\tt e} for which {\tt f[e]} has ever been called.
 
 \conceptsubsection{Models}
 \sectionlink{{\tt partial\_grid\_function<E,T>}}{partial-grid-function-hash}
 --- a generic implementation of partial grid functions by hash tables,
 defined in
 \gralcodelink{partial-grid-function-hash.h}{base}.

 \conceptsubsection{Notes}

 \begin{enumerate}
 \item \notelabel{note-create-storage}
   This means that storage is also added if the access is meant read-only!
   Therefore, one should use the syntax {\tt t = f(e)} in this case.
 \end{enumerate}
 
 \conceptsubsection{See also}
 \sectionlink{\concept{ Grid Element Function }}{GridElementFunction} ~
 \sectionlink{\concept{ Grid  Function }}{GridFunction} ~
 \sectionlink{\concept{ Mutable Grid  Function }}{MutableGridFunction} ~
 \sectionlink{\concept{ Container Grid  Function }}{ContainerGridFunction} ~
 \sectionlink{\concept{ Total Grid Function }}{TotalGridFunction} ~
  



% grid geometries
\section{Grid Geometries}
\xname{VertexGridGeometry}
\begin{Label}{VertexGridGeometry}
 \conceptsection{Vertex Grid Geometry Concept} 
\end{Label}

\conceptsubsection{Description}
The \concept{Vertex Grid Geometry} concept is the weakest
\sectionlinkshort{grid geometry}{GeometricalLayer} concept.
It simply provides a mapping from grid vertices to points in some space.

The \concept{Mutable Vertex Grid Geometry} concept allows in addition
the assignment of vertex coordinates.

\conceptsubsection{Refinement of}

\conceptsubsection{Notation}
\type{Geo} is a type which is a model of \concept{Vertex Grid Geometry} \\
{\tt g} is an object of type \type{Geo}\\
{\tt v, v1, v2} are objects of a type \type{V} which is a model of \conceptlink{Grid Vertex}{GridVertex}.\\
{\tt q} is an object of {\tt Geo::coord\_type}

\W\conceptsubsection{Definitions}

\conceptsubsection{Associated types}

\noindent
\begin{tabularx}{14cm}{llR} 
  \\ \hline
  \bf  Name  & \bf  Expression  &\bf  Description   \\
  \hline
  grid type &
  {\tt Geo::grid\_type} &
  underlying grid type, model of \conceptlink{Vertex Grid Range}{VertexGridRange}.
  \\
   point type &
   {\tt Geo::coord\_type}&
   the geometric point type, 
   representation of the elements of the topological 
   space where the geometry lives
   \par model
   of STL \Stllink{Assignable}.
   \\
  \hline
\end{tabularx}

\conceptsubsection{Valid Expressions}

\noindent
\begin{tabular}{llll}
  \\ \hline
  \bf  Name  &\bf  Expression  &\bf  Type requirements  & \bf  return type  \\
  \hline
   get vertex coordinates & q = g.coord(v)  &  & \type{coord\_type const\&} \\
   set vertex coordinates & g.coord(v) = q & \type{Geo} is mutable & \type{coord\_type \&} \\
  \hline
\end{tabular}

\conceptsubsection{Expression semantics}

\noindent 
 \begin{tabularx}{15cm}{RlRRR} 
   \\ 
   \hline 
   \bf  Name    &
   \bf  Expression &
   \bf  Precondition&
   \bf  Semantics &
   \bf  Postcondition
   \\ 
   \hline
    coordinate read access & 
    {\tt q = g.coord(v)} &
    {\tt v} is \footlink{valid}{valid} &
    if {\tt v1 == v2} then {\tt g.coord(v1)} is the same as\noteref{note-gvg-comparison}
    {\tt g.coord(v2)} &
    \\
    coordinate write access &
    {\tt g.coord(v) = q} &
    {\tt v} is \link{valid}{valid} &
    set the coordinate of {\tt v} to {\tt q} &
    {\tt g.coord(v)} is equal to {\tt q}
    \\
   \hline
\end{tabularx}

\W\conceptsubsection{Invariants}

\conceptsubsection{Refinements}
\conceptlink{Volume Grid Geometry}{VolumeGridGeometry}

\conceptsubsection{Models}
\sectionlink{\type{IstreamComplex2DFmt}}{istream-complex2d-fmt}

\conceptsubsection{Notes}
\begin{enumerate}
\item \notelabel{note-gvg-comparison} 
In general, the type \type{Geo::coord\_type} will not be a model
of STL \stllink{Equality Comparable}{EqualityComparable},
because this is normally not useful for floating point values.
\end{enumerate}

\conceptsubsection{See also}



\xname{VolumeGridGeometry}
\begin{Label}{VolumeGridGeometry}
 \conceptsection{Volume Grid Geometry Concept}
\end{Label}

\conceptsubsection{Description}
 The \concept{Volume Grid Geometry} assigns a geometric entity
to every combinatorial element of a grid. 
Moreover, the it offers methods to calculate the 
volumes of these geometric entities, 
which means volume in the combinatorial dimension of the 
entities, for example, length of edges or area of faces.

\conceptsubsection{Refinement of}
\conceptlink{Vertex Grid  Geometry}{VertexGridGeometry}

\conceptsubsection{Notation}
\type{Geo} is a type which is a model of \concept{Vertex Grid Geometry} \\
{\tt g} is an object of type \type{Geo}\\
{\tt x} is an object of type \type{Geo::real\_type} \\
{\tt q} is an object of {\tt Geo::coord\_type}\\
{\tt s} is an object of type \type{Geo::segment\_type} \\
{\tt p} is an object of type \type{Geo::polygon\_type}\\
{\tt h} is an object of type \type{Geo::polyhedron\_type}\\
{\tt e} is an objects of type \type{Geo::grid\_type::Edge},
 which is a model of \conceptlink{Grid Edge}{GridEdge}.\\
{\tt f}  is an objects of type \type{Geo::grid\_type::Face},
which is a model of \conceptlink{Grid Face}{GridFace}.\\
{\tt c}  is an objects of type \type{Geo::grid\_type::Cell},
which is a model of \conceptlink{Grid Cell}{GridCell}.\\

\W\conceptsubsection{Definitions}

\conceptsubsection{Associated types}
\begin{tabularx}{14cm}{llR} 
  \T \\ \hline
  \bf  Name  & \bf  Expression  &\bf  Description   \\
  \hline
   point type &
   {\tt Geo::coord\_type}&
   the geometric point type, 
   representation of the elements of the topological 
   space where the geometry lives
   \par model
   of STL \Stllink{Assignable}.
   \\
   real number type & 
   \type{real\_type} &
   representation of a real number,
   model of   STL \Stllink{Assignable} 
   \\
   segment type &
   \type{segment\_type} & 
   geometric type corressponding to edges,
   \par model of   STL \Stllink{Assignable} 
   \\
   polygon type &
   \type{polygon\_type} &
   geometric type corresponding to faces 
   (if the grid type has combinatorial dimension $\geq 2$.)
   \par model of   STL \Stllink{Assignable} 
   \\
   polyhedron type &
   \type{polyhedron\_type} &
   geometric type corresponding to cells 
   (if the grid type has combinatorial dimension $3$.)
   \par model of   STL \Stllink{Assignable} 
   \T \\   \hline
 \end{tabularx}

\conceptsubsection{Valid Expressions}
 
\begin{tabularx}{14cm}{RRRR}
  \T \\ \hline
  \bf  Name  &\bf  Expression  &\bf  Type requirements  & \bf  return type  
  \\
  \hline
   $1$-dimensional geometric entity &
   \pcode{s = g.segment(e)} &
   & \type{Geo::segment\_type}
   \\
   $2$-dimensional geometric entity &
   \pcode{p = g.polygon(f)} &
   & \type{Geo::polygon\_type}
   \\
   $3$-dimensional geometric entity &
   \pcode{h = g.polyhedron(c)} &
   & \type{Geo::polyhedron\_type}
   \\
   $1$-dimensional volume &
   \pcode{x = g.volume(e);} &
    & real\_type
    \\
   $1$-dimensional volume &
   \pcode{x = g.length(e);} &
    & real\_type
    \\
   $2$-dimensional volume &
   \pcode{x = g.volume(f);} &
    & real\_type
    \\
   $2$-dimensional volume &
   \pcode{x = g.area(f);} &
    & real\_type
    \\
   $3$-dimensional volume &
   \pcode{x = g.volume(c);} &
    & \type{real\_type}
    \T \\  \hline
\end{tabularx}

\conceptsubsection{Expression semantics}

The general requirement here is that the mapping to geometric entities
is faithful with respect to the combinatorial structure,
 that is, the poset defined by the inclusion relation of the geometric elements
is identical or a refinement of the poset of the grid:
The relative boundary of a geometric image of a grid element is identical of
the union of the geometric images of its lower-dimensional incidents.
Note, however, that self-intersecting immersions are not excluded.

 \begin{tabularx}{15cm}{RRRRR} 
   \T \\    \hline 
   \bf  Name    &
   \bf  Expression &
   \bf  Precondition&
   \bf  Semantics &
   \bf  Postcondition
   \\ 
   \hline
   $1$-dimensional geometric entity &
   \pcode{s = g.segment(e)} &
   \pcode{e} is \footlink{valid}{valid} &
   get the segment corresponding to \pcode{e} &
    \pcode{s == g.segment(e)} 
    \par \pcode{start(s) == g.coord(e.V1())}
    \par \pcode{end(s) == g.coord(e.V2())}
    \\
    $2$-dimensional geometric entity &
    \pcode{p = g.polygon(f)} &
    \pcode{f} is \link{valid}{valid} &
    get the polygon corresponding to \pcode{f} &
    $\texorhtml{\partial}{\mbox{boundary of}}
      \mbox{\tt p} = \bigcup_{\mbox{\tt e} \prec \mbox{\tt f}} \mbox{\tt g.segment(e)}$
    \T \\      \hline
\end{tabularx}

\conceptsubsection{Invariants}
\conceptsubsection{Refinements}

\conceptsubsection{Models}
\conceptsubsection{Notes}
\conceptsubsection{See also}




\xchapter{Components}{Components}
\W\htmlmenu{2}
\setcounter{htmlautomenu}{1}
%\renewcommand{\RelativeSection}[1]{\subsubsection{#1}}
%\renewcommand{\RelativeSubsection}[1]{\par\medskip\noindent{\bf #1}\par\noindent}
%\renewcommand{\RelativeSubsection}[1]{\paragraph*{#1} \nix \smallskip \par\noindent }

\section{Grids and grid ranges}
%  \W\htmlmenu{1}

 % `native' grids
\xname{Complex2D}
\begin{Label}{Complex2D}
\typesection{{\tt Complex2D} Grid Type}
\end{Label}

\needswork{This documentation is still incomplete!}

\typesubsection{Description}
The type {\tt Complex2D} is a model of \sectionlink{\concept{Grid}}{Grid}.
It allows representation of  general regular $2$-dimensional finite CW-complexes.
That is, the cells are combinatorially equivalent to  simple polygons.

\typesubsection{Example}

\begin{example}
Complex2D G;         // empty grid
\sectionlinkfoot{RegGrid2D}{RegGrid2D} R(10,10);  // 10x10 Cartesian grid
\sectionlinkfoot{ConstructGrid0}{ConstructGrid}(G,R); // copy R to G;
assert(G.NumOfVertices() == R.NumOfVertices());
assert(G.NumOfCells()    == R.NumOfCells());
typedef grid\_types<Complex2D> gt; // 'namespace' for Complex2D-related types 
for(gt::CellIterator c(G), ! c.IsDone(); ++c)
  cout << "Cell " << c.handle() << "  "
       << "has "  << (*c).NumOfVertices() << " vertices\\n";
\end{example}

\typesubsection{Definition}

Defined in \gralfilelink{complex2d}{h}{complex2d}.
  
\typesubsection{Model of}
\sectionlink{\concept{Grid}}{Grid}
\\
\sectionlink{\concept{Vertex Grid Range}}{VertexGridRange}
\\
\sectionlink{\concept{Edge Grid Range}}{VertexGridRange}
\\
\sectionlink{\concept{Facet Grid Range}}{VertexGridRange}
\\
\sectionlink{\concept{Cell Grid Range}}{VertexGridRange}

\typesubsection{Members}
\begin{tabularx}{15cm}{XXX} \hline
  \bf  Member   &
  \bf  Where defined &
  \bf  Description 
  \\ \hline
  \multicolumn{3}{c}{{\bf \em Types}}  \\ \hline
  \multicolumn{3}{c}{{\em handle types\/}}  \\ \hline
  \type{vertex\_handle} &
  \sectionlinkshort{\concept{Vertex Grid Range}}{VertexGridRange} &
  \sectionlinkshort{ handle }{GridElementHandle}   type
  corr. to \type{Vertex} 
  \\
  \type{edge\_handle} &
  \sectionlinkshort{\concept{Edge Grid Range}}{VertexGridRange} &
  \sectionlinkshort{handle}{GridElementHandle}   type
  corr.\ to \type{Edge}
  \\
  \type{facet\_handle} &
  \sectionlinkshort{\concept{Facet Grid Range}}{VertexGridRange} &
  \sectionlinkshort{ handle }{GridElementHandle}   type
  corr.\ to \type{Facet}
  \\
  \type{cell\_handle} &
  \sectionlinkshort{\concept{Cell Grid Range}}{VertexGridRange} &
  \sectionlinkshort{handle}{GridElementHandle}   type
  corr.\ to \type{Cell}
  \\ \hline
  \multicolumn{3}{c}{{\em element types\/}} \\ \hline
  \sectionlinkUNDEF{\type{Vertex}}{Complex2DVertex} &
  \sectionlinkshort{\concept{Vertex Grid Range}}{VertexGridRange}&
  The \sectionlinkshort{\concept{Vertex}}{GridVertex} element type 
  \\
  \sectionlinkUNDEF{\type{Edge}}{Complex2DEdge} &
  \sectionlinkshort{\concept{Edge Grid Range}}{VertexGridRange}&
  The \sectionlinkshort{\concept{Edge}}{GridEdge} element type 
  \\
  \sectionlinkUNDEF{\type{Facet}}{Complex2DFacet} &
  \sectionlinkshort{\concept{Facet Grid Range}}{VertexGridRange}&
  The \sectionlinkshort{\concept{Facet}}{GridFacet} element type 
  \\
  \sectionlinkUNDEF{\type{Cell}}{Complex2DCell} &
  \sectionlinkshort{\concept{Cell Grid Range}}{VertexGridRange} &
  The \sectionlinkshort{\concept{Cell}}{GridCell} element type 
  \\ \hline
  \multicolumn{3}{c}{{\em sequence iterator types\/}} \\ \hline
  \type{VertexIterator} &
  \sectionlinkshort{\concept{Vertex Grid Range}}{VertexGridRange} &
  The  \sectionlinkshort{\concept{ Sequence Iterator}}{GridSequenceIterator}   
  type for \type{Vertex}
  \\
  \type{EdgeIterator} &
  \sectionlinkshort{\concept{Edge Grid Range}}{VertexGridRange}&
  The  \sectionlinkshort{\concept{ Sequence Iterator}}{GridSequenceIterator}   
  type for \type{Edge}
  \\
  \type{FacetIterator} &
  \sectionlinkshort{\concept{Facet Grid Range}}{VertexGridRange}&
  The  \sectionlinkshort{\concept{ Sequence Iterator}}{GridSequenceIterator}   
  type for \type{Facet}
  \\
  \type{CellIterator} &
  \sectionlinkshort{\concept{Cell Grid Range}}{VertexGridRange}&
  The  \sectionlinkshort{\concept{ Sequence Iterator}}{GridSequenceIterator}   
  type for \type{Cell}
  \\ 
  \hline
  \multicolumn{3}{c}{{\bf \em Functions\/}} \\ 
  \hline
  \multicolumn{3}{c}{{\em sequence iteration\/}}  \\ 
  \hline
  \type{VertexIterator} \par \type{FirstVertex()} &
  \sectionlinkshort{\concept{Vertex Grid Range}}{VertexGridRange} &
  Iterator pointing to the first vertex 
  \\ 
  \multicolumn{3}{c}{ ---  {\em same for\/} \type{Edge}, \type{Facet}, \type{Cell} {\em types\/} --- } \\ \hline
  \multicolumn{3}{c}{{\em sequence sizes\/}}  \\ \hline
  \type{int NumOfVertices()} &
  \sectionlinkshort{\concept{Vertex Grid Range}}{VertexGridRange} &
  number of vertices 
  \\
  \multicolumn{3}{c}{ --- {\em same for\/} \type{Edge}, \type{Facet}, \type{Cell} {\em types\/} --- } \\ \hline
\end{tabularx}

\typesubsection{See also}
\gralclasslink{Triang2D}{triang2d} ~
\gralclasslink{RegGrid2D}{cartesian2d}

    
  


\xname{RegGrid2D}
\begin{Label}{RegGrid2D}
\typesection{{\tt RegGrid2D} Grid Type}
\end{Label}

\needswork{This documentation is still incomplete!}

\typesubsection{Description}
The type {\tt RegGrid2D} is a model of \sectionlink{\concept{Grid}}{Grid}.
It allows representation of Cartesian grids, that is, regular $m \times n$
tensor product grids.

\typesubsection{Example}

\begin{example}
#include "Grids/Reg2D/cartesian-grid2d.h"

\sectionlinkfoot{RegGrid2D}{RegGrid2D} R(10,10);  // 10x10 Cartesian grid
assert(R.NumOfVertices() == 10*10);
assert(R.NumOfCells()    == 9*9);
typedef grid\_types<RegGrid2D> gt; // 'namespace' for RegGrid2D-related types 
for(gt::CellIterator c(R), ! c.IsDone(); ++c)
  cout << "Cell " << c.handle() << "  "
       << "has "  << (*c).NumOfVertices() << " vertices\\n";
\end{example}

\typesubsection{Definition}

Defined in \gralcodelink{cartesian-grid2d.h}{cartesian2d}
\texorhtml{}{(see also \xlink{cartesian2d}{\GRALINC{cartesiand2d}{index}})}.
  
\typesubsection{Model of}
\sectionlink{\concept{Grid-With-Boundary}}{Grid-With-Boundary}
\\
\sectionlink{\concept{Vertex Grid Range}}{VertexGridRange}
\\
\sectionlink{\concept{Edge Grid Range}}{VertexGridRange}
\\
\sectionlink{\concept{Facet Grid Range}}{VertexGridRange}
\\
\sectionlink{\concept{Cell Grid Range}}{VertexGridRange}

\typesubsection{Members}
\begin{tabularx}{15cm}{XXX} \hline
  \bf  Member   &
  \bf  Where defined &
  \bf  Description 
  \\ \hline
  \multicolumn{3}{c}{{\bf \em Types}}  \\ \hline
  \multicolumn{3}{c}{{\em handle types\/}}  \\ \hline
  \type{vertex\_handle} &
  \sectionlinkshort{\concept{Vertex Grid Range}}{VertexGridRange} &
  \sectionlinkshort{ handle }{GridElementHandle}   type
  corr. to \type{Vertex} 
  \\
  \type{edge\_handle} &
  \sectionlinkshort{\concept{Edge Grid Range}}{VertexGridRange} &
  \sectionlinkshort{handle}{GridElementHandle}   type
  corr.\ to \type{Edge}
  \\
  \type{facet\_handle} &
  \sectionlinkshort{\concept{Facet Grid Range}}{VertexGridRange} &
  \sectionlinkshort{ handle }{GridElementHandle}   type
  corr.\ to \type{Facet}
  \\
  \type{cell\_handle} &
  \sectionlinkshort{\concept{Cell Grid Range}}{VertexGridRange} &
  \sectionlinkshort{handle}{GridElementHandle}   type
  corr.\ to \type{Cell}
  \\ \hline
  \multicolumn{3}{c}{{\em element types\/}} \\ \hline
  \sectionlinkshort{\type{Vertex}}{RegGrid2DVertex} &
  \sectionlinkshort{\concept{Vertex Grid Range}}{VertexGridRange}&
  The \sectionlinkshort{\concept{Vertex}}{GridVertex} element type 
  \\
  \sectionlinkshort{\type{Edge}}{RegGrid2DEdge} &
  \sectionlinkshort{\concept{Edge Grid Range}}{VertexGridRange}&
  The \sectionlinkshort{\concept{Edge}}{GridEdge} element type 
  \\
  \sectionlinkshort{\type{Facet}}{RegGrid2DFacet} &
  \sectionlinkshort{\concept{Facet Grid Range}}{VertexGridRange}&
  The \sectionlinkshort{\concept{Facet}}{GridFacet} element type 
  \\
  \sectionlinkshort{\type{Cell}}{RegGrid2DCell} &
  \sectionlinkshort{\concept{Cell Grid Range}}{VertexGridRange} &
  The \sectionlinkshort{\concept{Cell}}{GridCell} element type 
  \\ \hline
  \multicolumn{3}{c}{{\em sequence iterator types\/}} \\ \hline
  \type{VertexIterator} &
  \sectionlinkshort{\concept{Vertex Grid Range}}{VertexGridRange} &
  The  \sectionlinkshort{\concept{ Sequence Iterator}}{GridSequenceIterator}   
  type for \type{Vertex}
  \\
  \type{EdgeIterator} &
  \sectionlinkshort{\concept{Edge Grid Range}}{VertexGridRange}&
  The  \sectionlinkshort{\concept{ Sequence Iterator}}{GridSequenceIterator}   
  type for \type{Edge}
  \\
  \type{FacetIterator} &
  \sectionlinkshort{\concept{Facet Grid Range}}{VertexGridRange}&
  The  \sectionlinkshort{\concept{ Sequence Iterator}}{GridSequenceIterator}   
  type for \type{Facet}
  \\
  \type{CellIterator} &
  \sectionlinkshort{\concept{Cell Grid Range}}{VertexGridRange}&
  The  \sectionlinkshort{\concept{ Sequence Iterator}}{GridSequenceIterator}   
  type for \type{Cell}
  \\ 
  \hline
  \multicolumn{3}{c}{{\bf \em Functions\/}} \\ 
  \hline
  \multicolumn{3}{c}{{\em constructors\/}}  \\ 
  \hline
  RegGrid2D(int m, int n) &
  ~ &
  construct cartesian grid with $n$ vertices in $x$ direction
  and $m$ vertices in $y$ direction. 
  \\
  \hline
  \\
  \multicolumn{3}{c}{{\em sequence iteration\/}}  
  \\ 
  \hline
  \type{VertexIterator} \par \type{FirstVertex()} &
  \sectionlinkshort{\concept{Vertex Grid Range}}{VertexGridRange} &
  Iterator pointing to the first vertex 
  \\ 
  \multicolumn{3}{c}{ ---  {\em same for\/} \type{Edge}, \type{Facet}, \type{Cell} {\em types\/} --- } \\ \hline
  \multicolumn{3}{c}{{\em sequence sizes\/}}  \\ \hline
  \type{int NumOfVertices()} &
  \sectionlinkshort{\concept{Vertex Grid Range}}{VertexGridRange} &
  number of vertices 
  \\
  \multicolumn{3}{c}{ --- {\em same for\/} \type{Edge}, \type{Facet}, \type{Cell} {\em types\/} --- } \\ \hline
\end{tabularx}

\typesubsection{See also}
\sectionlinkUNDEF{\concept{ }\type{Triang2D} }{Triang2D} ~
\sectionlink{\concept{ }\type{Complex2D} }{Complex2D} ~
    
  


 %%\input{triang2d}
 %% subranges/ subgrids
 \xname{EnumeratedGridRange}
\begin{Label}{EnumeratedGridRange}
\datasection{enumerated\_subrange}
\end{Label}

\datasubsection{Declaration}
\begin{example}
template<class G>
class enumerated\_subrange;
\end{example}

\datasubsection{Description}
The  class template \type{enumerated\_subrange} implements
a sub-range of a grid.
Cells and vertices must be explicitly joined 
to the range\noteref{note-vertex-cell-consistency}.
Iteration over facets is provided.

\datasubsection{Model of}
\sectionlink{\concept{Vertex Grid Range}}{VertexRange}
\\
\sectionlink{\concept{Edge Grid Range}}{EdgeRange}\noteref{note-edgerange-in-2d-only}
\\
\sectionlink{\concept{Facet Grid Range}}{FacetRange}
\\
\sectionlink{\concept{Cell Grid Range}}{CellRange}

\datasubsection{Definition}
Defined in \gralfilelink{enumerated-subrange}{h}{base}
\datasubsection{Template parameters}

\begin{tabular}{lll} \hline
  \bf Parameter & \bf Description & \bf Default \\
  \hline
  {\tt G}  & the base grid  & ~ \\
  \hline
\end{tabular}

\datasubsection{Type requirements}
{\tt G} must be a model of \sectionlink{\concept{Grid}}{Grid}.

\datasubsection{Public base classes}
None.
\datasubsection{Members}

\datasubsection{New members}

\begin{tabularx}{11cm}{lR}
  \hline
  \bf Member & \bf Description \\
  \hline
   \multicolumn{2}{c}{\bf\em Types} \\
   \hline
   {\tt vertex\_range\_ref } 
   & reference to vertex range, 
   \par defined as {\gralclasslink{vertex\_range\_ref}{base}<G,R>} \\
   {\tt cell\_range\_ref } 
   & reference to cell range, 
   \par defined as {\tt \gralclasslink{cell\_range\_ref}{base}<G,R>} \\
   \hline
   \multicolumn{2}{c}{\bf\em Functions} \\
   \hline
  {\tt append\_vertex(vertex\_handle v)} & add a new vertex to {\tt r} \\
  {\tt append\_cell(cell\_handle c)} & add a new cell to {\tt r} \\
  {\tt vertex\_range\_ref vertices()}  & reference to vertex range \\
  {\tt cell\_range\_ref cells()}  & reference to cell range \\
  \hline
\end{tabularx}

\datasubsection{Example}
\begin{example}
  a\_grid\_type g;
  ...
  enumerated\_subrange<a\_grid\_type>  r(g); // empty range
  // fill with cells
  for(gt::CellIterator c(g); ! c.IsDone(); ++c)
    if(predicate(*c)) // some predicate on cells
      r.append_cell(c.handle());
  // determine vertex set of r.cells()
  \gralfunctionlink{ConstructSubrangeFromCells}{base}(r,r.cells());
  
\end{example}
\datasubsection{Notes}
\begin{enumerate}
\item \notelabel{note-vertex-cell-consistency}
  The availability of these operations means that
  instances of  \type{enumerated\_subrange}
  do not ensure that their 
  vertex range is the vertex set of the 
  their cell range.
  It is the responsibility of the client to ensure this 
  {\em if\/} it is needed.
  If this equality is not satisfied, it simply means that the range
  does not represent a \Glossarylink{dimension-homogeneous} grid.

  Note, however, that there are components that can help in ensuring
  this property in a subrange, for example the 
  \gralfunctionlink{ConstructSubrangeFromCells}{base}  algorithm.


\item \notelabel{note-edgerange-in-2d-only}
  The  \sectionlinkshort{\concept{Edge Grid Range}}{EdgeRange} concept
  is supported in 2D only (when edges and facets coincide).
  The current implementation simply assumes that the template parameter
  {\tt G} is two-dimensional; for future extension, 
  a compile-time switch on the dimension should be performed here.
\item \notelabel{note-element-types-of-subrange}
  The element types and incidence iterator types of
   \type{enumerated\_subrange} are forwarded from the template
   parameter \type{G}. There is currently no mechanism implemented
   to determine at compile time which types are provided by \type{G},
   therefore, some default assumption has to be made, excluding 
   some simpler grid types from the set of possible parameters.
   This restriction is artificial and should be removed.
\end{enumerate}

\datasubsection{See also}

\gralclasslink{enumerated\_element\_range}{base} ~
\gralclasslink{enumerated\_element\_range\_ref}{base} ~
\sectionlinkshort{\concept{Grid Range}}{GridRange}
 %%\input{enumerated-vertex-range}
 %%\input{enumerated-grid-range-ref}
 %%\input{enumerated-vertex-range-ref}
 %%\input{remove-cells-grid}
 %%\input{boundary}
 %%\input{neumann-triang2d}

 %% input grids %% 
 \xname{istream-complex2d-fmt}
\begin{Label}{istream-complex2d-fmt}
\datasection{IstreamComplex2DFmt}
\end{Label}

\datasubsection{Declaration}
class \gralclasslink{IstreamComplex2DFmt}{base};

\datasubsection{Description}
The class \type{IstreamComplex2DFmt} is an adapter to the grid interface
for a grid given by a sequential stream, for example a file,
in the complex2d format.
By using this class, it is possible to read an arbitrary grid from
this format, without knowing its details.

\datasubsection{Model of}
\conceptlink{Cell-Vertex Input Grid Range}{Cell-VertexInputGridRange}
\\
\conceptlink{Vertex Grid Geometry}{VertexGridGeometry}

\datasubsection{Definition}
Defined in \gralcodelink{istream-grid-complex2d-fmt.h}{base}
\datasubsection{Public base classes}
\datasubsection{Members}
\datasubsection{New members}
\begin{tabularx}{14cm}{lR}
  \hline
  \bf Member & \bf Description \\
  \hline
  \pcode{IstreamComplex2DFmt(istream\&, int offset = 0)} &
     Constructor, offset gives the lowest vertex number used
    (usually 0 or 1).
    \\
    \pcode{coord\_type const\& coord(Vertex const\& v)}
    & access to vertex coordinates.  
    \\
    \hline
\end{tabularx}

\datasubsection{Example}
\begin{example}
a_grid_type Agrid;
a_geom_type Ageom; // model of \conceptlinkfoot{Mutable Grid Vertex Geometry}{VertexGridGeometry}
ifstream in("grid.complex2d");
IstreamComplex2DFmt SrcG(in); // Grid & geom combined

// (Agrid, Ageom) = (SrcG, SrcG)
\conceptlinkfoot{Construct}{ConstructGrid}(Agrid, Ageom, SrcG, SrcG);
\end{example}
\datasubsection{Known uses}
\datasubsection{Notes}
\datasubsection{See also}


 %%\input{ostream-complex2d-fmt}
 %%\input{array-complex2d-fmt}


 \section{Grid elements}
 \xname{vtuple}
\begin{Label}{vtuple}
\datasection{vtuple<GRID>}
\end{Label}

\datasubsection{Declaration}
\begin{example}
template<class GRID>
class vtuple;

template<class FaceOnCellIt>
vtuple<typename FacetOnCellIt::grid_type>
get_vertices(FacetOnCellIt const& f);

template<class GRID>
hash<vtuple<GRID> >;
\end{example}

\datasubsection{Description}
The class template \type{vtuple} serves to hold 
a sufficient part of the vertex set of a \sectionlink{Grid Facet}{GridFacet}
in order to indentify it unambigously within a given grid.
\datasubsection{Model of}
STL \Stllink{Assignable}
\\
STL \Stllink{EqualityComparable}
\\
STL \Stllink{LessThanComparable}
\\
\type{hash<vtuple<GRID> >} is a model of
STL \Stllink{HashFunction}.

\datasubsection{Definition}
Defined in
\gralcodelink{vtuple2d.h}{base}

\datasubsection{Template parameters}
\begin{tabular}{lll} \hline
  \bf Parameter & \bf Description & \bf Default \\
  \hline
  \type{GRID}  & the base grid  & ~ \\
  \type{FacetOnCellIt}  & a facet-on-cell iterator for \type{GRID}  & ~ \\
  \hline
\end{tabular}

\datasubsection{Type requirements}
\type{GRID} is a model of \conceptlink{Grid}{Grid}
and of \conceptlink{Vertex Grid Range}{VertexGridRange}
(it must \glossarylinkfoot{publish}{publishing a type}
 the type \type{vertex\_handle}). \\
\type{FacetOnCellIt} is a model of 
\conceptlink{FacetOnCell Iterator}{Vertex-On-CellIterator}
and of 
\conceptlink{Vertex Grid Range}{VertexGridRange}.

\datasubsection{Public base classes}
None.
\datasubsection{Members}
\datasubsection{New members}
\datasubsection{Example}

\datasubsection{Known uses}
Used in \sectionlink{Cell-neighbor-search}{cell-neighbor-search}.
\datasubsection{Notes}
\begin{enumerate}
\item \notelabel{note-vtuple-only2d}
The type vtuple has only been implemented for two-dimensional grids.
For general use, a compile-time switch on the combinatorial 
dimension of the grid has to be implemented.
\end{enumerate}

\datasubsection{See also}


%\input{facet-handle}

 \section{Grid iterators}
 \xname{GenericFacetIterator}
\begin{Label}{GenericFacetIterator}
\itersection{FacetIterator<>}
\end{Label}

\itersubsection{Declaration}
\begin{example}
template<class FacetOnCellIt>
class FacetIterator;
\end{example}
\itersubsection{Description}

%\noindent
The class template \type{FacetIterator<>} implements a 
\conceptlink{Grid Facet Iterator}{GridVertexIterator}
based on cell iteration and local facet iteration.
The iteration proceeds by iterating over each facet of each cell;
in order to ensure visiting each facet only once, 
there is established an arbitrary order between the two
possible occurences of an interior facet, and the
greater one is silently skipped over.

In order to evaluate the comparison, an order must be 
defined on cells, and neighbor cells must be accessible\noteref{note-fi-neighbors}.

\itersubsection{Definition}
Defined in \nmwrcodelink{facet-iterator.h}{Grids/Iterators}

\itersubsection{Template parameters}

\par
\begin{tabular}{lll} \hline
  \bf Parameter & \bf Description & \bf Default \\
  \hline
  \type{FacetOnCellIt}  & a local facet-on-cell iterator & ~ \\
  \hline
\end{tabular}

\itersubsection{Model of}
\conceptlink{Facet Iterator}{GridVertexIterator}

\itersubsection{Type requirements}
\type{FacetOnCellIt} is a model of
\conceptlink{FacetOnCell Iterator}{Vertex-On-CellIterator}.
The grid type of \type{FacetOnCellIt} must be a model of
\conceptlink{Cell Grid Range}{VertexGridRange}
and of \Conceptlink{Grid-With-Boundary};
the associated cell type must be a model
of STL \Stllink{LessThanComparable}.

\itersubsection{Members}
\itersubsection{Complexity}
Each facet is visited at most $2$ times.
Memory consumption is constant --- one
\type{CellOnCellIterator} and one \type{FacetOnCellIterator}.
\itersubsection{Example}
\begin{example}
a_grid_type g;
a_geometry_type geom(g); // a \conceptlinkfoot{grid geometry}{GeometricalLayer} for g

typedef grid_types<a_grid_type> gt;
\sectionlinkfoot{grid_function}{ContainerGridFunction}<gt::\conceptlinkfoot{Facet}{GridFacet}, double> volume(g);

typedef FacetIterator<gt::\conceptlinkfoot{FacetOnCellIterator}{Vertex-On-CellIterator}> FacetIt;
for(FacetIt f(g); ! f.IsDone(); ++f)
  volume[*f] = \conceptlinkfoot{geom.volume}{VolumeGridGeometry}(*f);
\end{example}
\itersubsection{Uses}
\itersubsection{Used by}

\itersubsection{Notes}
\begin{enumerate}
\item \notelabel{note-fi-neighbor}
If information about cell neighbors is not available, 
one has to mark visited facets explicitely,
which is done by \sectionlinkUNDEF{\type{facet\_set\_of\_cells\_iterator}}{facet-set-of-cells}.
In a future version, there could be a compile-time switch on the
availability of cell neighbor information, 
selecting the appropriate version of \type{FacetIterator}
automatically.
\end{enumerate}
\itersubsection{See also}
\sectionlink{\type{facet\_handle}}{facet-handle}
~
\sectionlinkUNDEF{\type{facet\_set\_of\_cells\_iterator}}{facet-set-of-cells}
 %%\input{facet-set-of-cells-iterator}
 %%\input{boundary-iterator}

\section{Grid functions}
\xname{grid-function-vector}
\begin{Label}{grid-function-vector}
\datasection{grid\_function\_vector<E,T> }
\end{Label}

\datasubsection{Declaration}
\begin{example}
template<class E, class T>
class grid_function_vector;
\end{example}
\datasubsection{Description}
The class template \type{grid\_function\_function} 
is an implementation 
of the \conceptlink{Total Grid Function}{TotalGridFunction} concept.
It works for any element type whose handles are 
consecutively ordered integers \noteref{note-gfv-consecutive}.
\datasubsection{Model of}
\conceptlink{Total Grid Function}{TotalGridFunction}
\datasubsection{Definition}
Defined in \nmwrcodelink{grid-function-vector.h}{Grids}

\datasubsection{Template parameters}
\begin{tabular}{lll} \hline
  \bf Parameter & \bf Description & \bf Default \\
  \hline
  \type{E}  & the element type  & ~ \\
  \type{T}  & the value  type  & ~ \\
  \hline
\end{tabular}

\datasubsection{Type requirements}
\type{E} must be a model of \conceptlink{Grid Element}{GridElement}.\\
\type{T} must be a model of STL \Stllink{Assignable}.

\datasubsection{Public base classes}
\type{grid\_function\_hash\_base<E,T>}, defined in
\nmwrcodelink{grid-function-hash.h}{Grids}.

\datasubsection{Members}
\datasubsection{New members}

\datasubsection{Example}
\begin{example}
 a_grid_type g;
 
 typedef grid_types<a_grid_type> gt;
 partial_grid_function<gt::Vertex, int> color(g,black);
 mark_white_vertices(g,colors);
\end{example}
\datasubsection{Known uses}
Used for implementation of total grid functions for most grids,
for example \Sectionlinkshort{RegGrid2D}  and
total grid function for the element types \type{Vertex} and \type{Cell}
of \Sectionlinkshort{Complex2D}.

These implemenentations rely on the technique of
\Glossarylink{partial specialization}.

\datasubsection{Notes}
\begin{enumerate}
\item \notelabel{note-gfv-consecutive}
Currently, it is not possible to obtain this information at compile
time.
\end{enumerate}
\datasubsection{See also}
\Sectionlinkshort{partial-grid-function-hash}


\xname{grid-function-hash}
\begin{Label}{grid-function-hash}
\datasection{grid\_function\_hash<E,T> }
\end{Label}

\datasubsection{Declaration}
\begin{example}
template<class E, class T>
class grid_function_hash;
\end{example}
\datasubsection{Description}
The class template \type{grid\_function\_function} 
is an implementation 
of the \conceptlink{Total Grid Function}{TotalGridFunction} concept.
It works for any element type for which the
\Stltypelink{hash} template has been specialized.

\datasubsection{Model of}
\conceptlink{Total Grid Function}{TotalGridFunction}
\datasubsection{Definition}
Defined in \nmwrcodelink{grid-function-hash.h}{Grids}

\datasubsection{Template parameters}
\begin{tabular}{lll} \hline
  \bf Parameter & \bf Description & \bf Default \\
  \hline
  \type{E}  & the element type  & ~ \\
  \type{T}  & the value  type  & ~ \\
  \hline
\end{tabular}

\datasubsection{Type requirements}
\type{E} must be a model of \conceptlink{Grid Element}{GridElement}.\\
The \stltypelink{hash}{hash} template must be specialized for \type{E}. \\
\type{T} must be a model of STL \Stllink{Assignable}.

\datasubsection{Public base classes}
\type{grid\_function\_hash\_base<E,T>}, defined in
\nmwrcodelink{grid-function-hash.h}{Grids}.

\datasubsection{Members}
\datasubsection{New members}

\datasubsection{Example}
\begin{example}
 a_grid_type g;
 
 typedef grid_types<a_grid_type> gt;
 partial_grid_function<gt::Vertex, int> color(g,black);
 mark_white_vertices(g,colors);
\end{example}
\datasubsection{Known uses}
Used for implementation of total grid functions for most grids,
for example \Sectionlinkshort{RegGrid2D}  and
total grid function for the element types \type{Vertex} and \type{Cell}
of \Sectionlinkshort{Complex2D}.

These implemenentations rely on the technique of
\Glossarylink{partial specialization}.

\datasubsection{Notes}
\datasubsection{See also}
\Sectionlinkshort{partial-grid-function-hash}


\xname{partial-grid-function-hash}
\begin{Label}{partial-grid-function-hash}
\datasection{partial\_grid\_function\_<E,T> }
\end{Label}

\datasubsection{Declaration}
\begin{example}
template<class E, class T>
class partial_grid_function;
\end{example}
\datasubsection{Description}
The class template \type{partial\_grid\_function} is an implementation 
of the \conceptlink{Partial Grid Function}{PartialGridFunction} concept.
It works for any element type, provided there 
is a specialization of the \stltypelink{hash}{hash} template for 
this type defined.
\datasubsection{Model of}
\conceptlink{Partial Grid Function}{PartialGridFunction}
\datasubsection{Definition}
Defined in \nmwrcodelink{partial-grid-function-hash.h}{Grids}

\datasubsection{Template parameters}
\begin{tabular}{lll} \hline
  \bf Parameter & \bf Description & \bf Default \\
  \hline
  \type{E}  & the element type  & ~ \\
  \type{T}  & the value  type  & ~ \\
  \hline
\end{tabular}

\datasubsection{Type requirements}
\type{E} must be a model of \conceptlink{Grid Element}{GridElement}.\\
The \stltypelink{hash}{hash} template must be specialized for \type{E}. \\
\type{T} must be a model of STL \Stllink{Assignable}.

\datasubsection{Public base classes}
\type{grid\_function\_hash\_base<E,T>}, defined in
\nmwrcodelink{grid-function-hash.h}{Grids}.

\datasubsection{Members}
\datasubsection{New members}

\datasubsection{Example}
\begin{example}
 a_grid_type g;
 
 typedef grid_types<a\_grid\_type> gt;
 grid_function_vector<gt::Edge, double> length(g);
 for(gt::EdgeIterator e(g); ! e.IsDone(); ++e)
   length[*e] = distance(geom.coord((*e).V1()),
                         geom.coord((*e).V2()));
\end{example}
\datasubsection{Known uses}
Used for implementation of total grid functions for 
\type{Complex2D::Edge}.

\datasubsection{Notes}

\datasubsection{See also}
\Sectionlink{grid-function-vector}
~
\Sectionlink{grid-function-hash}


\section{Grid geometries}
%\input{linear-geometry}
%\input{transformed-geometry}
\section{Functors}

\htmldepth{5}
\section{Grid algorithms}

%%\subsection{Combinatorial algorithms}

\noindent
{\bf Combinatorial algorithms}
\xname{cell-neighbor-search}
\begin{Label}{cell-neighbor-search}
\algosection{Cell-neighbor-search}
\end{Label}

\algosubsection{Declaration}
\begin{example}
[1] template<class NBF, class CELLRANGE>
    void CalculateNeighborCells
          (NBF             &  Nb,   // out
           CELLRANGE  const&  C);   // in

[2] template<class NBF, class CELLRANGE, class FACETMAP>
    void CalculateNeighborCells
          (NBF             &  Nb,   // out
           CELLRANGE  const&  C,    // in
           FACETMAP        &  F);   // inout

[3] template<class NBF, class CELLRANGE, class FACETMAP, class CGT>
    void CalculateNeighborCells
         (NBF             &  Nb,    // out
          CELLRANGE  const&  C,     // in
          FACETMAP        &  F,     // inout
          CGT        const&);       // in (only type used) 
\end{example}


\algosubsection{Description}
\function{CalculateNeighborRange} takes a range of cells \variable{C} 
and determines
the neighbor relation on its cells.
(Two cells are neighbors if they share a facet.)
The output is stored in \variable{nb}.

\function{CalculateNeighborRange} (versions [2], [3]) 
can be used incrementally
in the variable \variable{C}:
If $C = C_1 \cup \ldots \cup C_n$ is a partition of a given cell set,
then calling \function{CalculateNeighborRange} with the cell set $C$ is
equivalent to the successive calls with cell sets $C_1, \ldots, C_n$.
(The remaining arguments stay identical.)

\algosubsection{Definition}
\function{CalculateNeighborRange} is defined in 
\gralcodelink{cell-neighbor-search.h}{base}

\algosubsection{Requirements on types}
\begin{itemize}
\item \templateparam{CELLRANGE}, \templateparam{CGT}:\\
  \templateparam{CELLRANGE} is a model of 
  \sectionlink{\concept{CellGridRange}}{VertexGridRange}. 

  The following types must be 
  \glossarylink{published}{publishing a type}
  by 
  the type \templateparam{CGT} (version [3])
  or 
  the template \type{grid\_types<CELLSET>} (versions [1], [2]):
 
  \begin{itemize}
  \item \type{Cell}: a model of 
    \sectionlink{\concept{Grid Cell}}{GridCell} 
    and
    \sectionlink{\concept{Facet Grid Range}}{VertexGridRange}
  \item \type{FacetOnCellIterator}: a model of 
    \sectionlink{\concept{Facet-On-Cell Iterator}}{Vertex-On-CellIterator}.
  \item \type{Facet}: a model of
    \sectionlink{\concept{Grid Facet}}{GridFacet} and of 
    \sectionlink{\concept{Vertex Grid Range}}{VertexGridRange}
  \item \type{Vertex}: a model of \sectionlink{\concept{Grid Vertex}}{GridVertex}.
  \end{itemize}

  In the following, we use
  \templateparam{CGT} also for \footlink{\type{grid\_types}}{intro-grid-types}\type{<CELLSET>} 
  in versions [1], [2].
\item \templateparam{NBF} \\ 
  is a model of 
  \sectionlinkUNDEF{\concept{Mutable Mapping}}{MutableMapping}
  from \type{CGT::FacetOnCellIterator} to \type{CGT::cell\_handle}.
\item \templateparam{FACETMAP} \\
  is a model of 
  \sectionlinkUNDEF{\concept{Mutable Mapping}}{MutableMapping}
  from 
  \sectionlink{\type{vtuple}}{vtuple}\type{<CGT::grid\_type>}
  to \type{CGT::FacetOnCellIterator}.
  In addition, the following expressions must be defined:
  (Here {\tt i} is of type \type{FACETMAP::iterator}, {\tt f}
  is of type \type{CGT::FacetOnCellIterator}, and
  {\tt F} is of type \type{FACETMAP}. )
  \begin{itemize}
  \item type \type{iterator}
  \item {\tt i = F.end();}
  \item {\tt i = F.find(f);}
  \item {\tt F.erase(i);}
  \end{itemize}
  This is for example fullfilled if \templateparam{FACETMAP}
  is a model of STL \stllink{Pair Associative Container}{PairAssociativeContainer}.
\end{itemize}


\algosubsection{Notation}
$F$ is the set of facets of \variable{C}.
\\
$I$  the set of  \glossarylink{interior facets}{interior facet}
of  \variable{C}.
\\$B$ the \glossarylink{boundary facets}{boundary facet} of  \variable{C}
(such that $I \cup B = F$).
\\
$D$ is the domain of the map \variable{F} before the call.
\\
$D^\prime$ is the domain of the map \variable{F} after the call.

\algosubsection{Preconditions}
The grid underlying \variable{C} is a 
subgrid of a \glossarylink{manifold-with-boundary complex}{manifold-with-boundary}.
\\
For versions [2] and [3],
the domain of the map 
\variable{F} may already contain facets,
but these must be  boundary facets:
\[
   D \cap I = \emptyset
\]

\algosubsection{Postconditions}
All pairs of cells sharing an interior facets $\in I$ have been found,
as well as all cells having a Facet such that the corresponding 
vertex set has been in $D$.

After the call, 
\variable{F} 
contains exactly the boundary facets,
except those that have been in it before (versions [2], [3]).
In version [1], the value of \variable{Nb} is set to an invalid cell handle
for boundary facets (that is, facets which have been visited by only
facet-on-cell iterator).


\[
  D^\prime = (D \setminus B) \cup (B \setminus D)
\]
It follows that  $D^\prime \cap I = \emptyset$.

\algosubsection{Complexity}
Expected time $O(F)$.

\algosubsection{Example}
\begin{example}
#include "Grids/Algorithms/cell-neighbor-search.h"

\xlinkUNDEF{Triang2D}{Triang2D} t;
typedef grid_types<Triang2D> gt;
...
\xlink{hash_map}{\STLPATH{hash_map.html}}<gt::FacetOnCellIterator> , gt::cell_handle> Nbs;
CalculateNeighborCells(nbf,T);

for(gt::CellIterator c(t); ! c.IsDone(); ++c) {
  for(gt::FacetOnCellIterator fc(c); ! fc.IsDone(); ++fc)
    out << nbf[fc] << ' ';
  out << '\\\\n';
}
\end{example}
The mapping from \code{gt::FacetOnCellIterator} to \code{gt::cell\_handle}
can of course be done much more efficiently than with a hash table, see
the example in  \xlink{{\tt main-triang2d.C}}{\GRALROOT/test/Grids/triang2d/main-triang2d.C}.

\algosubsection{Known Uses}

Used in \xlink{test}{\GRALROOT/test/Grids/triang2d/main-triang2d.C}
of \sectionlinkUNDEF{\type{Triang2D}}{Triang2D}.

\algosubsection{Notes}

\algosubsection{See also}


\xname{count-boundary-components}
\begin{Label}{count-boundary-components}
\algosection{Count boundary components}
\end{Label}

\algosubsection{Declaration}

\begin{example}
template<class GRID>
int count\_boundary\_components(GRID const& G);
\end{example}

\algosubsection{Description}
Counts the number of connected components of the boundary of {\tt G}.
\algosubsection{Definition}
Declared in \gralfilelink{find-boundary-components}{h}{base}\\
Defined  in \gralfilelink{find-boundary-components}{C}{base}\\

\algosubsection{Type requirements}
\type{GRID} is a model of 
\Conceptlink{Grid} and of
\conceptlink{Facet Grid Range}{VertexGridRange}.\\
There is a \conceptlink{Partial Grid Function}{PartialGridFunction}
container on facets
associated to \type{GRID} \noteref{note-cbc-partial-gf}.
\algosubsection{Preconditions}
\Var{G} represents a \Glossarylink{manifold-with-boundary} grid.
\algosubsection{Postconditions}
The value returned is the number of connected components of \Var{G}'s boundary.
\algosubsection{Complexity}
Expected time $O(F)$, where $F$ is the number of facets of \Var{G}.
\algosubsection{Example}

\algosubsection{Uses}
 \sectionlink{find-boundary-component-germs}{find-boundary-component-germs}.

\algosubsection{Used in}

\algosubsection{Notes}
\begin{enumerate}
\item \label{note-cbc-partial-gf}
This is not a real constraint, because there is a 
\sectionlink{generic implementation}{partial-grid-function-hash}
of partial grid functions.
The only thing needed to use it is to define (specialize)
the \xlink{\type{hash<>}}{\STLURL/hash.html}
template on \type{Grid::Facet}.
\end{enumerate}
\algosubsection{See also}


\xname{ConstructGrid}
\begin{Label}{ConstructGrid}
\algosection{ConstructGrid}
\end{Label}

\needswork{This documentation is still incomplete!}

\algosubsection{Declaration}
This is actually a family of related functions:
\begin{example}
[1]
template<class G1, class G2>
inline void 
ConstructGrid0(G1            & destG, 
               G2       const& srcG);

[2]
template<class G1, class G2, class VertexMap, class CellMap>
extern void 
ConstructGrid0VC(G1            & destG, 
                 G2       const& srcG,
                 VertexMap     & V21,
                 CellMap       & C21); 

[3]
template<class G1, class Geom1, class G2, class Geom2>
extern void 
ConstructGrid(G1         & destG,
              Geom1 const& destGeom,
              G2    const& srcG, 
              Geom2 const& srcGeom);

[4]
template<class G1, class Geom1, 
         class G2, class Geom2, class VertexMap>
extern void 
ConstructGridV(G1            & destG, 
               Geom1         & destGeom,
               G2       const& srcG,
               Geom2    const& srcGeom,
               VertexMap     & V21); 

[5]
template<class G1, class Geom1, 
         class G2, class Geom2, class VertexMap, class CellMap>
extern void 
ConstructGridVC(G1            & destG, 
                Geom1         & destGeom,
                G2       const& srcG,
                Geom2    const& srcGeom,
                VertexMap     & V21,
                CellMap       & C21);
\end{example}

\algosubsection{Description}
The \code{ConstructGridXXX} family of template functions
make an \Glossarylink{associative copy} 
of a source grid \variable{Gsrc} to a destination grid \variable{Gdest}.
The  \code{ConstructGrid0XXX} functions ([1], [2])
do only a copy of combinatoric grids,
the other copy also geometry information.

The templates above are actually not really fully generic ---
they must be specialized to the type of the first parameter, \type{G1}.
The more general grids the \type{G1} grid type can represent,
the more useful these functions are.
For very specialized grids, like Cartesian ones, 
they do not make much sense.

This algorithm is in some sense the analogue to the STL
\stllink{\type{copy}}{copy} algorithm for sequences,
only that it cannot be fully generic on the destination argument.
\algosubsection{Definition}
Declared in \gralfilelink{construct-grid}{h}{base}.
\\
Specialized in 
\gralfilelink{complex2d-construct}{C}{complex2d}
for \sectionlink{\type{Complex2D}}{Complex2D}
and
\gralfilelink{construct}{C}{triang2d}

\algosubsection{Template parameters}

\algosubsection{Type requirements}
\type{Geom1} is a model of \conceptlink{Mutable Vertex Grid Geometry}{VertexGridGeometry}
\type{G2} is a model of \conceptlink{Cell-Vertex Input Grid Range}{Cell-VertexInputGridRange}
\\ 
\type{Geom2} is a model of \conceptlink{Vertex Grid Geometry}{VertexGridGeometry}
\\
\type{VertexMap} is a model of \sectionlinkUNDEF{Mutable Mapping}{MutableMapping}
from \type{G2::vertex\_handle} to \type{G1::vertex\_handle}
\\
\type{CellMap} is a model of \sectionlinkUNDEF{Mutable Mapping}{MutableMapping}
from \type{G2::cell\_handle} to \type{G1::cell\_handle}

\algosubsection{Preconditions}
\variable{srcGeom} is \glossarylink{bound}{bound to a grid} to 
\variable{srcG}.

\algosubsection{Postconditions}
\variable{destG} contains a copy of \variable{srcG},
and ([3]-[5]) \variable{destGeom} contains a copy of 
\variable{srcGeom},
such that the mappings 
\variable{V21} and \variable{C21} induce a 
\Glossarylink{grid morphism}
from \variable{srcG} to \variable{destG}.

\algosubsection{Complexity}
Linear in the size of \variable{srcG}.

\algosubsection{Example}
\begin{example}
\sectionlinkfoot{RegGrid2D}{RegGrid2D} R(10,10);  // 10x10 Cartesian grid
\sectionlinkUNDEF{mapped_geometry_reg2d<Linear2D>}{MappedGeometryReg2D}  geomR(R, Linear2D::identity); // 
\sectionlink{Complex2D}{Complex2D} G;         // empty grid
\sectionlinkUNDEF{stored_geometry_complex2d}{StoredGeometryComplex2D} geomG(G);
ConstructGrid(G,geomG, R,geomR); // copy R to G;
assert(G.NumOfVertices() == R.NumOfVertices());
assert(G.NumOfCells()    == R.NumOfCells());
typedef grid\_types<Complex2D> gt; // 'namespace' for Complex2D-related types 
for(gt::CellIterator c(G), ! c.IsDone(); ++c)
  cout << "Cell " << c.handle() << "  "
       << "has "  << (*c).NumOfVertices() << " vertices";
\end{example}

\algosubsection{Uses}

\algosubsection{Used by}
The operation of copying is so fundamental that
it is used virtually everywhere a grid has to be created.
For example, the algorithm is used for implementing 
the member function \code{Complex2D::read(istream\&)}, 
defined in \gralfilelink{complex2d}{C}{complex2d}
Here an adapter \sectionlink{\type{IstreamComplex2DFmt}}{istream-complex2d-fmt}
for a grid stored in a file is used.
\algosubsection{Notes}

\algosubsection{See also}
\sectionlinkUNDEF{EnlargeGrid}{EnlargeGrid}


%%\input{algo-find-boundary-component-germs}
%%\input{algo-adjacency-hull}

%%\noindent
%%{\bf Geometric algorithms}
%%\subsection{Geometric algorithms}

\section{Adapters to other libraries}
%%\input{cgal-adapter}
%%\input{ggcl-adapter}
%%\input{graph-adapter}

\section{Distributed Grids}
\subsection{Data structures}
%%\input{overlapping-grid}
%%\input{distributed-grid}
%%\input{composite-grid}
%%\input{global-grid}
%%\input{overlapping-gridfunction}
%%\input{distributed-gridfunction}
%%\input{composite-gridfunction}
%%\input{global-gridfunction}
\subsection{Algorithms}

%\xname{ConcreteExamples}
\begin{Label}{ConcreteExamples}
   \introsection{Concrete examples}
\end{Label}
 \needswork{This section needs work!}

There are several grid classes implemented, 
that conform to the interface described above, 
presented here with varying degree of ``iteration power'':
\begin{itemize}
\item  \sectionlink{\type{RegGrid2D}}{RegGrid2D}, a 2-dimensional
  regular grid, model of \sectionlink{\concept{FullGrid}}{FullGrid},
\item  \sectionlink{\type{Complex2D}}{Complex2D}, a  2-dimensional
  unstructured grid with arbitrary cells, model of \sectionlink{\concept{FullCellGrid}}{FullCellGrid},
\item \sectionlink{\type{Triang2D}}{Triang2D}, a  2-dimensional triangulation
\end{itemize}
Some (not very useful) examples of the iterator features
can be found in a  
\xlink{test routine}{\NMWRROOT/test/Grids/cartesian} 
for the cartesian grid, 
and a  
\xlink{test class}{\NMWRROOT/src/Grids/Complex2D/test-complex.C}
for  the {\tt Complex2D} grid.
  

\T\end{small}

\xchapter{Glossary of Grid Category Terms}{Glossary}
% \xname{GridGlossary}
%\begin{Label}{GridGlossary}
%\section{Glossary of Grid Category Terms}
%\end{Label}

\begin{Glossar}
\item[adjacent]
  Two cells are adjacent if they are \Glossarylink{incident} 
  to a common facet.
  Two vertices are adjacent if they are incident to a common edge.
  For other types of elements, we do not define adjacency.

\item[associative copy]
  An associative copy between two grids is a copy operation
  which preserves (outputs) a mapping between the elements of the first and
  those of the second grid, such that this mapping is (part of) a 
  \Glossarylink{grid morphism}.

\item[boundary facet] See \Glossarylink{interior facet}.

\item[bound to a grid] 
  We say that a grid entity \variable{e} of type \type{E}
  is bound to a grid \variable{g}
  if \variable{g} is an object of type \type{E::grid\_type} 
  and \code{\&g == \&(e.TheGrid())}.
  Here \type{E} might be a \Glossarylink{model} of 
  \conceptlink{Grid Element}{GridElement},
  \conceptlink{Grid Function}{GridFunction} or the like.

\item[category]
  is used to denote a group of related components
  which implement concepts from the same problem domain, such
  as grid-related components (grid category), which has 
  sub-categories combinatorial grids, grid functions (or data association),
  and grid geometries.

\item[component]
  A component is any piece of code which can be used elsewhere,
  for example a class template, a single class, a set of related classes,
  or a function implementing an algorithm.

\item[concept]
  Following to the 
  \xlink{STL}{\STLURL} parlance
  a concept is a set of type requirements.
  Generic algorithms require their arguments to be \glossarylink{models}{model}
  of certain concepts.
  The STL documentation explains further:
  \begin{quote}
    Concepts are not a part of the C++ language; there is no way to declare a concept in a
    program, or to declare that a particular type is a model of a concept.
  \end{quote}

\item[CW-complex]
A finite CW-complex $C$ of dimension $d$ 
is a set of topological open $k$-cells with $0 \leq k \leq d$.
(An open k-cell is a set homeomorphic to the open k-ball in $R^k$.)
The boundary of each $k-cell$ in $C$ must be formed by
the union of cells of lower dimension
which are also contained in $C$.

This is a very general definition, which must often be refined
for practical purposes.
However, the case of cells with holes (sometimes used in geometric modeling)
is \emph{not\/} covered by the definition.

\item[dimension-homogeneous]
  A grid element is called {\sl principal\/} if it is not incident to a higher dimensional element.
   A grid is dimension-homogeneous, if each principal element is of
   the highest dimension (namely, the dimension of the grid, $d$).

   This property excludes for example isolated vertices,
   or `free' edges (if the dimension is at least $2$).

\item[element]
We use the term element to denote any k-cell of a grid:
A vertex is a 0-element, an edge a 1-element, and a cell a $d$-element
(if the grid is $d$-dimensional). 
See also \xlink{Element subsection}{CombinatorialLayer.html#element-intro}

\item[generic programming]
Generic programming has been called ``programming with concepts''.
It aims at implementing algorithms as general as possible, 
without sacrificing efficiency by doing overgeneralization or
introducing undue amounts of runtime overhead.
More information can be found \xlink{here}{\BERTIURL/generic}.

\item[geometric realization]
A geometric realization of a combinatorial (or abstract) complex
(which is given by a graded \Glossarylink{poset} or a \Glossarylink{lattice})
is a CW-complex which has the same (or isomorphic) 
incidence poset.


\item[grid]
Grid is used as synonym for a 
finite 
\Glossarylink{dimension-homogeneous}
\Glossarylink{CW-complex}.

\item[grid morphism]
A grid morphism is a mapping between the 
\glossarylink{posets}{poset} of two grids
which respects the partial order.
It is an isomorphism if it is bijective.

\item[incident]
 Two elements are incident if one is contained in the boundary 
 of the other. Thus, elements of the same dimension cannot be incident,
 only \Glossarylink{adjacent}.

\item[interior facet] A facet $f$ is called interior with respect to a cell set 
  $C$ (which is assumed to be a sub-range of a manifold-with-boundary grid),
  if there two different cells in $C$ which are incident to $f$.
  If there is exactly one cell in $C$ incident to $f$, then $f$ is a
  {\sl boundary facet\/} of $C$. If there is no such cell, $f$ is not a facet 
  of $C$.


\item[lattice]
  A lattice is a graded, bounded \Glossarylink{poset}
  with the additional property that given two element $a$ and $b$,
  there is a unique minimal upper bound $a \vee b$ (the {\sl join\/})
  greater then $a$ and $b$,
  and a unique maximal lower bound $a \wedge b$ (the {\sl meet\/}).
  For example, two edges both incident to the same 2D cell have
  as join that cell, and as meet either a common vertex (which must be unique)
  or the empty set.

\item[manifold-with-boundary]
  A combinatorial grid of dimension $d$ is a manifold-with-boundary grid,
  if it has a {\sl \Glossarylink{geometric realization}} which is a manifold with boundary.

\item[model]
  A concrete type is a model of a \Glossarylink{concept},
  if it fullfils the requirements of the concept.

\item[partial specialization] refers to binding a part of the parameters 
  of some generic component (class template) to some more specialized type.
  For example, consider the fully generic template frame for 
  \conceptlink{total grid functions}{TotalGridFunction}:
  \begin{example}
  template<class E, class T>
  class grid_function<E,T> \{\};
  \end{example}
  In generic algorithms, this serves as a placeholder for the actual implementations of 
  \type{grid\_function} for concrete element  types \type{E}.
  For example, total grid functions for the \sectionlinkfoot{\type{Complex2D}}{Complex2D}
  grid type are implemented by using vectors for the element types \type{Complex2D::Vertex}
  and \type{Complex2D::Cell}, and by using hash tables for the element type \type{Complex2D::Edge}:
  \begin{example}
  template<class T>
  grid_function<Complex2D::Vertex, T> 
    : public \sectionlinkweb{grid\_function\_vector}{grid-function-vector}<Complex2D::Vertex, T> \{
    // repeat constructors
  \};

  template<class T>
  grid_function<Complex2D::Cell, T> 
    : public \sectionlinkweb{grid_function_vector}{grid-function-vector}<Complex2D::Cell, T> \{
    // repeat constructors
  \};

  template<class T>
  grid_function<Complex2D::Edge, T> 
    : public \sectionlinkweb{grid_function_hash}{grid-function-hash}<Complex2D::Edge, T> \{
    // repeat constructors
  \};
  \end{example}

\item[poset]
A (strict) poset ${\cal S}$ is a {\bf p}artially {\bf o}rdered finite {\bf set}, 
that is, a set with a partial order $<$ 
which is antisymmetric and transitive. 
The poset of a grid consists of the grid's elements,
partially ordered by inclusion.

A poset is {\sl bounded\/} if there are unique minimal and maximal elements
$\hat{0}$ and $\hat{1}$.
A {\sl chain\/} is a totally ordered subset of a poset. A bounded poset
is called {\sl graded\/} if every maximal chain has the same {\sl length}
(its number of elements minus 1). 
For $ a \leq b$, the {\sl interval\/} $[a,b]$ 
is the set of all elements in between:
\[
  [a,b] = \{ c \in {\cal S} | a \leq c \leq b \}
\]
If ${\cal S}$ is graded, the {\sl rank\/} of $a \in {\cal S}$ is the length
of a maximal chain in $[\hat{0},a]$.

The poset of a grid ${\cal G}$ can be made into a bounded one by adjoining
the {\sl improper elements\/}  $\hat{0} = \emptyset$
and $\hat{1} = \|{\cal G}\|$, with dimensions $-1$ and $d+1$.

  
\item[publishing a type] means a standard way of associating types with classes.
  One way to do it is to use nested typedefs within the class, like in the following
  example from the STL:
  \begin{example}
   template<class T>
   class vector \{
   public:
     typedef T  value_type;
     typedef T* iterator;
     ...
   \};
  \end{example}
  Here the class template \type{vector} publishes \type{value\_type} and \type{iterator},
  which can be used in another component:
  \begin{example}
    template<class Container>
    void f(Container const& C, Container::value_type const& t) {
      Container::iterator i = C.begin();
      // ...
    }
  \end{example}
  Another way to look at it is to say that \type{vector<T>} defines
  a mapping from type \type{vector<T>} to a set of associated type like
  \type{vector<T>::iterator}.

  A different and somewhat more flexible way of achieving this is using so called
  \webref{traits}{\traitsURL}{Meyers95}
  classes, which makes use of several different such mappings
  for a given type possible. For this purpose, responsibility for the type definitions
  is delegated to a separate class, the {\sl traits class}.
  \begin{example}
   template<class C>
   struct container_traits \{\}; // default: empty

   // specialize for vectors
   template<class T>
   struct container_traits< vector<T> > \{
     typedef T  value_type;
     typedef T* iterator;
     // ...
   \};
  \end{example}
  If the algorithm {\tt f} above continues to be parameterized by {\tt C} alone,
  not much changes, only the occurences of {\tt C::iterator} has to replaced by
  {\tt container\_traits<C>::iterator}.
  On the other hand, one might imagine a {\sl counted iterator}, which counts the number
  of increments. It would not be easy to introduce this without traits.
  However, we can use an additional traits parameter to the algorithm {\tt f}:
  \begin{example}
    template<class Container, class Traits>
    void f(Container const& C, Container::value_type const& t, Traits const&) {
      Traits::iterator i = C.begin();
      // ...
    }
  \end{example}
  Now it is possible to introduce counted iterators, without any change to the algorithm
  implementation:
  \begin{example}
  template<class C>
  struct counted_traits {
    typedef C::value_type value_type; // use same value type
    typedef counted_iterator<C::iterator> iterator;
    // ... 
  };

  // use f, count increments
  MyContainer myc;
  typedef counted_traits<MyContainer> my_traits;
  
  f(my_c, t, my_traits());
  \end{example}
  In the grid component framework, the template \type{grid\_types<>}
  plays exactly the r{\^o}le of \type{container\_traits<>} above.

  A similar effect could be achieved by deriving from the container class,
  or defining a wrapper class with a delegation mechanism, which would contain
  the changed typedefs. However, this would not work for built-in types,
  and also not for aggregations (containers of containers), because there is no
  way of changing the type of contained items.
\end{Glossar}

  




\end{document}