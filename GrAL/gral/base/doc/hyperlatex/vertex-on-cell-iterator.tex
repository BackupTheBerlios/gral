 \xname{Vertex-On-CellIterator}
\begin{Label}{Vertex-On-CellIterator}
\conceptsection{Vertex-On-Cell (-Facet, ...) Iterator Concept}
\end{Label}

\conceptsubsection{Description}
A {\em Vertex-On-Cell Iterator\/} refines the concept of 
\sectionlink{\concept{Grid Incidence Iterator }}{GridIncidenceIterator} :
It allows to access all vertices incident to a given cell.
Thus, it can also be seen as a 
\sectionlink{\concept{Grid Sequence Iterator}}{GridSequenceIterator} 
over the 
\sectionlink{\concept{ Vertex Grid Range}}{VertexGridRange} of that cell.

Similar iterators can be defined for other anchor elements, most notably facets.
For these, just replace {\em cell\/} with {\em facet}.
\conceptsubsection{Refinement of}
\sectionlink{\concept{ Incidence Iterator}}{GridIncidenceIterator}
\\
%\sectionlink{\concept{Sequence Iterator}}{GridSequenceIterator}

\conceptsubsection{Notation}
{\tt V} is a model of Vertex-On-Cell Iterator
\\ 
{\tt v} is an  object of type  {\tt V}
\\
{\tt c} is an  object of type  {\tt V::Cell}

\conceptsubsection{Associated types}
The types {\tt V::element\_type} and {\tt V::anchor\_type}
can now be named more  specifically. The names from the incidence iterator concept
remain valid.

\noindent
\begin{tabular}{llp{7cm}} 
  \\
  \hline
  \bf  Name &\bf  Expression  &\bf  Description 
  \\ 
  \hline
  Vertex type  &
  {\tt V::Vertex} &
  synomym to {\tt V::element\_type} 
  \\ 
  Cell type  &
  {\tt V::Cell} &
  the cell type {\tt V} operates upon,
  \par model of \sectionlink{\concept{ Grid Cell }}{GridCell}
  \\ 
  Anchor type  & 
  {\tt V::anchor\_type} &
  synonym to  {\tt V::Cell},
  \par model of \sectionlink{\concept{Vertex Grid Range}}{VertexGridRange}  
  \\ 
  \hline
  \\
\end{tabular}
    
\conceptsubsection{Valid Expressions}

\begin{tabular}{llll} \\
  \hline
  \bf  Name  &\bf  Expression  &\bf  Type requirements  & \bf  return type  \\ 
  \hline
  anchor cell  &
  {\tt v.TheCell();} &
  ~ &
  {\tt V::Cell const\&} 
  \\ 
  \hline
  \\
\end{tabular}


\conceptsubsection{Expression semantics}

\begin{tabularx}{14cm}{RRRRR} \\
  \hline     
  \bf  Name     &
  \bf  Expression &
  \bf  Precondition&
  \bf  Semantics &
  \bf  Postcondition
  \\
  \hline
  anchor cell  &
  {\tt Cell C = v.TheCell();} &
  {\tt v is not singular} &
  get the anchor of {\tt v} & 
  {\tt c} is a valid cell; 
  \par
  {\tt c == v.TheCell()}
  \\ 
  \hline
  \\
\end{tabularx}

\W\conceptsubsection{Refinements}

\conceptsubsection{Models}
\sectionlinkUNDEF{{\tt Complex2D::VertexOnCellIterator}}{Complex2DVertexOnCellIterator}, 
defined in  
\xlink{{\tt vertex-on-cell2d-it.h}}{\NMWRROOT/include/Grids/Complex2D/vertex-on-cell2d-it.h } 

\W\conceptsubsection{Notes}


\conceptsubsection{See also}
\sectionlink{\concept{ Grid }}{Grid} ~
\sectionlink{\concept{ Grid Range }}{GridRange} ~
\\
\sectionlink{\concept{ Grid Element }}{GridElement} ~
\sectionlink{\concept{ Grid Vertex }}{GridVertex} ~
\sectionlink{\concept{ Grid Cell}}{GridCell} ~
\\
\sectionlink{\concept{Grid Sequence Iterator }}{GridSequenceIterator} ~
\sectionlink{\concept{Grid Incidence Iterator }}{GridIncidenceIterator} ~
  

