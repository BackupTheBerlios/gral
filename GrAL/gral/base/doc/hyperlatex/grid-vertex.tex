\xname{GridVertex}
\begin{Label}{GridVertex}
\conceptsection{Grid Vertex Concept}
\end{Label}

\conceptsubsection{Description}
A  {\em Grid Vertex\/} represents the mathematical concept of a vertex 
--- a 0-dimensional entity in a \sectionlink{\concept{ Grid }}{Grid}.

\conceptsubsection{Refinement of}
\sectionlink{\concept{ Grid Element}}{GridElement}
 
\conceptsubsection{Notation}
{\tt V} is a type which is a model of {\em grid vertex\/}
\\
{\tt v} is an object of type {\tt V}
\\
{\tt g} is an object of type {\tt V::grid\_type}
\\
{\tt h} is an object of type {\tt V::vertex\_handle}
\\
{\tt ci} is an object of type {\tt V::CellIterator}


\conceptsubsection{Associated types}

\noindent
{\em NOTE:\/} The types and expression involving 
\sectionlink{\concept{ Incidence Iterators}}{GridIncidenceIterator} are given below for the
case of \sectionlinkUNDEF{cell-on-vertex iteration}{CellOnVertexIterator}.
Analogous types and expressions can be defined for the other element types,
like edge, facet, or cell. 
The tables are to be understood in the following sense:
\\
{\em If\/} a vertex defines the incidence iterator over cells,
{\em then\/} the requirements under {\em Optional part\/} apply.
Analogous requirements take effect if `cell' is replaced by another element type.

\noindent
\begin{tabularx}{12cm}{llX} 
  \T \\  \hline
  \bf  Name  &\bf  Expression  &\bf  Description   \\ \hline
  handle type & 
  {\tt V::vertex\_handle} &
  type of the corresponding \sectionlink{\concept{Vertex Handle}}{GridVertexHandle} 
  \\ 
  \hline
  \multicolumn{3}{c}{\bf \em Optional part (as example) }
  \\
  \hline
  cell-on-vertex iterator &
  {\tt V::CellIterator}&
  type of the corr. CellOnVertexIterator
  \T \\ \hline
\end{tabularx}
    
\conceptsubsection{Valid Expressions}

\noindent
\begin{tabularx}{14cm}{Xlll} 
%  \\ 
  \T \hline
  \bf  Name  &\bf  Expression  &\bf  Type requirements  & \bf  return type \\ 
  \hline
  handle & 
  {\tt v.handle()} &
  ~ &
  {\tt V::vertex\_handle} 
  \\
  \hline
  \multicolumn{3}{c}{\bf \em Optional part (as example) }
  \\
  \hline
  cell-on-vertex iteration start & 
  {\tt v.FirstCell()} &
  ~ &
  {\tt V::CellIterator} 
  \\
  number of incident cells & 
  {\tt v.NumOfCells()} &
  ~ &
  {\tt int} 
  \T \\  \hline \\
\end{tabularx}
 
\T\begin{small}
\conceptsubsection{Expression semantics}
\begin{tabularx}{15cm}{XXXXX} 
  \T \\ \hline
  \bf  Name     &
  \bf  Expression &
  \bf  Precondition&
  \bf  Semantics &
  \bf  Postcondition
  \\ 
  \hline
  handle &
  {\tt h = v.handle();} &
  v is \footlink{valid}{valid} &
  shorthand for {\tt h = v.TheGrid(). handle(v)} &
  {\tt v == v.TheGrid(). vertex(h)}  
  \\ 
  \hline
  \multicolumn{3}{c}{\bf \em Optional part (as example) }
  \\
  \hline
  cell-on-vertex iteration start & 
  {\tt ci = v.FirstCell()} &
  {\tt v} is \link{valid}{valid}  &
  let {\tt ci} point to  the first cell incident to {\tt v}  & 
  {\tt ci.TheVertex() == ci.TheAnchor() == v} 
   and 
  {\tt ci.TheGrid() == v.TheGrid()}
  \\ 
  number of incident cells & 
  {\tt n =  v.NumOfCells()} &
  {\tt v} is \link{valid}{valid}  &
  n is the number of cells that are incident to {\tt v} &
  ~ 
  \T \\   \hline  \\
\end{tabularx}
\T\end{small}
    
\conceptsubsection{Complexity guarantees}
All operations are amortized constant time\noteref{note-vtx-amort}.

\W\conceptsubsection{Refinements}

\conceptsubsection{Models}

Vertex classes of 
\gralmodulelink{complex2d} (see \gralclasslink{Vertex2D}{complex2d}),
\gralmodulelink{complex3d},  
\gralmodulelink{cartesian2d},
\gralmodulelink{cartesian3d}, or
\gralmodulelink{triang2d}.

\conceptsubsection{Notes}
\begin{enumerate}
\item \notelabel{note-vtx-amort}
  Amortization is understood to involve calling the operations for all
  vertices of a grid.
\end{enumerate}

\conceptsubsection{See also}
\sectionlink{\concept{Grid}}{Grid} ~
\sectionlink{\concept{Grid Element Handle}}{GridElementHandle} ~
\sectionlink{\concept{Vertex Handle}}{GridVertexHandle}~
\sectionlink{\concept{Grid Element }}{GridElement} ~
\sectionlink{\concept{Grid Cell}}{GridCell} ~
\sectionlink{\concept{Sequence Iterator}}{GridSequenceIterator} ~
\sectionlink{\concept{Incidence Iterator}}{GridIncidenceIterator} ~

  

