\xname{GridEdge}
\begin{Label}{GridEdge}
\conceptsection{Grid Edge Concept}
\end{Label}

\conceptsubsection{Description}
A  {\em Grid Edge\/} represents the mathematical concept of an edge 
--- a 1-dimensional entity in a \sectionlink{\concept{Grid}}{Grid}.


\conceptsubsection{Refinement of}
\sectionlink{\concept{ Grid Element}}{GridElement}
 
\conceptsubsection{Notation}
{\tt E} is a type which is a model of \concept{Grid Edge}
\\
{\tt e} is an object of type {\tt E}
\\
{\tt V} is  a shorthand for the {\tt E::Vertex} type
\\
{\tt v} is an object of type {\tt V}
\\
{\tt g} is an object of type {\tt E::grid\_type}
\\
{\tt h} is an object of type {\tt E::edge\_handle}

\conceptsubsection{Associated types}

{\em NOTE:\/} The types and expression involving 
\sectionlink{\concept{ Incidence Iterators}}{GridIncidenceIterator} are given below for the
case of \sectionlink{cell-on-edge iteration}{Cell-On-EdgeIterator}.
Analogous types and expressions can be defined for the other element types,
like edge, facet, or cell. 
The tables are to be understood in the following sense:
\\
{\em If\/} a edge defines the incidence iterator over cells,
{\em then\/} the requirements under {\em Optional part\/} apply.
Analogous requirements take effect if `cell' is replaced by another element type.

\noindent
\begin{tabularx}{12cm}{llX} 
  \\
  \hline
  \bf  Name  &\bf  Expression  &\bf  Description   \\ \hline
  handle type & 
  {\tt E::edge\_handle} &
  type of the corresponding \sectionlink{\concept{Edge Handle}}{GridVertexHandle} 
  \\ 
  vertex type &
  \code{E::Vertex} &
  Vertex type of the edge, 
  short for \code{E::grid\_type::Vertex}.
  \\ 
  \hline
  \multicolumn{3}{c}{\bf \em Optional part (as example) }
  \\
  \hline
  cell-on-edge iterator &
  {\tt E::CellIterator}&
  type of the corr. CellOnEdgeIterator
  \\ 
  \hline
  \\
\end{tabularx}
    
\conceptsubsection{Valid Expressions}

\begin{tabularx}{12cm}{Xlll} \\ 
  \hline
  \bf  Name  &\bf  Expression  &\bf  Type requirements  & \bf  return type \\ 
  \hline
  handle & 
  {\tt e.handle()} &
  ~ &
  {\tt E::edge\_handle} 
  \\
  first vertex &
  \code{v = e.V1()} &
  ~ &
  \type{Vertex}
  \\
  second vertex &
  \code{v = e.V2()} &
  ~ &
  \type{Vertex}
  \\
  \hline
  \multicolumn{3}{c}{\bf \em Optional part (as example) }
  \\
  \hline
  cell-on-edge iteration start & 
  {\tt e.FirstCell()} &
  ~ &
  {\tt E::CellIterator} 
  \\
  number of incident cells & 
  {\tt E.NumOfCells()} &
  ~ &
  {\tt int} 
  \\
  \hline
  \\
\end{tabularx}
 
\T\begin{small}
\conceptsubsection{Expression semantics}
\begin{tabularx}{15cm}{XXXXX} \\
  \hline
  \bf  Name     &
  \bf  Expression &
  \bf  Precondition&
  \bf  Semantics &
  \bf  Postcondition
  \\ 
  \hline
  handle &
  {\tt h = e.handle();} &
  e is \footlink{valid}{valid} &
  shorthand for {\tt h = e.TheGrid(). handle(e)} &
  {\tt e == e.TheGrid(). edge(h)}  
  \\ 
  \hline
  \multicolumn{3}{c}{\bf \em Optional part (as example) }
  \\
  \hline
  cell-on-edge iteration start & 
  {\tt ci = e.FirstCell()} &
  {\tt e} is \link{valid}{valid}  &
  let {\tt ci} point to  the first cell incident to {\tt e}  & 
  {\tt ci.TheEdge() == ci.TheAnchor() == e} 
   and 
  {\tt ci.TheGrid() == e.TheGrid()}
  \\ 
  number of incident cells & 
  {\tt n =  e.NumOfCells()} &
  {\tt e} is \link{valid}{valid}  &
  n is the number of cells that are incident to {\tt e} &
  ~ 
  \\ 
  \hline
  \\
\end{tabularx}
\T\end{small}
    
\conceptsubsection{Complexity guarantees}
All operations are amortized constant time\noteref{note-edge-amort}.

\conceptsubsection{Refinements}

\conceptsubsection{Models}
\sectionlink{{\tt Complex2D::Edge}}{Complex2DEdge} 
defined in
\gralfilelink{edge2d}{h}{complex2d}

\conceptsubsection{Notes}
\begin{enumerate}
\item \notelabel{note-edge-amort}
  Amortization is understood to involve calling the operations for all
  edges of a grid.
\end{enumerate}

\conceptsubsection{See also}
\sectionlink{\concept{Grid}}{Grid} ~
\sectionlink{\concept{Grid Element Handle}}{GridElementHandle} ~
\sectionlink{\concept{Edge Handle}}{GridVertexHandle}~
\sectionlink{\concept{Grid Element }}{GridElement} ~
\sectionlink{\concept{Grid Cell}}{GridCell} ~
\sectionlink{\concept{Sequence Iterator}}{GridSequenceIterator} ~
\sectionlink{\concept{Incidence Iterator}}{GridIncidenceIterator} ~

  

