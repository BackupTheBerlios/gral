\xname{grid-function-vector}
\begin{Label}{grid-function-vector}
\datasection{grid\_function\_vector<E,T> }
\end{Label}

\datasubsection{Declaration}
\begin{example}
template<class E, class T>
class grid_function_vector;
\end{example}
\datasubsection{Description}
The class template \type{grid\_function\_function} 
is an implementation 
of the \conceptlink{Total Grid Function}{TotalGridFunction} concept.
It works for any element type whose handles are 
consecutively ordered integers \noteref{note-gfv-consecutive}.
\datasubsection{Model of}
\conceptlink{Total Grid Function}{TotalGridFunction}
\datasubsection{Definition}
Defined in \nmwrcodelink{grid-function-vector.h}{Grids}

\datasubsection{Template parameters}
\begin{tabular}{lll} \hline
  \bf Parameter & \bf Description & \bf Default \\
  \hline
  \type{E}  & the element type  & ~ \\
  \type{T}  & the value  type  & ~ \\
  \hline
\end{tabular}

\datasubsection{Type requirements}
\type{E} must be a model of \conceptlink{Grid Element}{GridElement}.\\
\type{T} must be a model of STL \Stllink{Assignable}.

\datasubsection{Public base classes}
\type{grid\_function\_hash\_base<E,T>}, defined in
\nmwrcodelink{grid-function-hash.h}{Grids}.

\datasubsection{Members}
\datasubsection{New members}

\datasubsection{Example}
\begin{example}
 a_grid_type g;
 
 typedef grid_types<a_grid_type> gt;
 partial_grid_function<gt::Vertex, int> color(g,black);
 mark_white_vertices(g,colors);
\end{example}
\datasubsection{Known uses}
Used for implementation of total grid functions for most grids,
for example \Sectionlinkshort{RegGrid2D}  and
total grid function for the element types \type{Vertex} and \type{Cell}
of \Sectionlinkshort{Complex2D}.

These implemenentations rely on the technique of
\Glossarylink{partial specialization}.

\datasubsection{Notes}
\begin{enumerate}
\item \notelabel{note-gfv-consecutive}
Currently, it is not possible to obtain this information at compile
time.
\end{enumerate}
\datasubsection{See also}
\Sectionlinkshort{partial-grid-function-hash}

