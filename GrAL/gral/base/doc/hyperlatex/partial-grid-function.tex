\xname{PartialGridFunction}
\begin{Label}{PartialGridFunction}
\conceptsection{Partial Grid Function Concept}
\end{Label}

\conceptsubsection{Description}

The  {\em Partial Grid Function\/} concept refines the 
\sectionlink{\concept{ Container Grid Function}}{ContainerGridFunction} concept.
A partial grid function reserves storage only for those values which are
write-accessed explicitly.
This is in contrast to 
\sectionlink{\concept{ Total Grid Function }}{TotalGridFunction} 
which allocates storage for each value.

Partial grid functions are of particular importance for locally operating 
algorithms with sublinear runtime and memory requirements.

\conceptsubsection{Refinement of}
\sectionlink{\concept{ Container Grid Function}}{ContainerGridFunction}

\conceptsubsection{Notation}
{\tt F} is a type which is a model of  Partial Grid  Function 
\\
{\tt f} is an object of type  {\tt F}
\\
{\tt G} is shorthand for  {\tt F::grid\_type}
\\
{\tt g} is an object of type  {\tt G}.

\begin{ifhtml}
\conceptsubsection{Associated types}
None, exept those defined in
\sectionlink{\concept{ Container Grid Function}}{ContainerGridFunction}
\end{ifhtml}

\conceptsubsection{Valid Expressions}
\begin{tabular}{llll} 
  \T \hline
  \bf  Name  &\bf  Expression  &\bf  Type requirements  & \bf  return type  \\ 
  \hline
  Definition test  &
  {\tt f.defined(e);} &
  ~ &
  {\tt bool} 
  \\ 
  Definition undo &
  {\tt f.undefine(e);} &
  ~ &
  ~
  \\ 
  Default setting &
  {\tt f.set\_default(t);} &
  ~ &
  ~ 
  \\ 
  Default access &
  {\tt f.default();} &
  ~ &
  F::value\_type 
  \T\\ 
  \T\hline
  \T\\
\end{tabular}


\conceptsubsection{Expression semantics}
\begin{tabularx}{15cm}{RRRRR}
 \T  \hline      
  \bf  Name    &
  \bf  Expression &
  \bf  Precondition&
  \bf  Semantics &
  \bf  Postcondition
  \\ 
  \hline
    construction from grid & 
    {\tt F f(g);} &
    ~ &
    construct and bind {\tt f}  to {\tt g} &
    {\tt f} is bound to {\tt g} 
    \par write access is allowed 
    \par read access is undefined
    \par {\tt f.size()} is equal to the cardinality of the 
    \link{range}[\footnote{see \Ref}]{range} of {\tt f}
  \\ 
    construction and initialization & 
    {\tt F f(g,t);} &
    ~ &
    construct and bind {\tt f}  to {\tt g}, 
    \par initialize the default value to {\tt t} 
    &
    {\tt f} is bound to {\tt g} 
    \par write access is allowed 
    \par {\tt f(e)} is equal to {\tt t} for all elements {\tt e}
    in the range of {\tt f}.
    \par {\tt f.size()} is equal to the cardinality of the 
    \link{range}{range} of {\tt f}
  \\ 
    Binding to grid &
    {\tt f.set\_grid(g);} &
    f is unbound &
    bind {\tt f}  to {\tt g} &
    {\tt f} is bound to {\tt g} 
    \par write access is allowed, 
    \par read access is undefined
    \par {\tt f.size()} is equal to the cardinality of the 
    \link{range}{range} of {\tt f}
   \\
    Definition test &
    {\tt if(f.defined(e))} &
    ~ &
    true if {\tt f[e]} has been explicitely set.&
    ~ 
   \\
    Definition undo &
    {\tt f.undefine(e)} &
    {\tt f.defined(e)} &
    remove {\tt e} from the set of explicitely defined values &
    {\tt f(e)} is equal to {\tt f.default()} 
    \par {\tt f.defined(e) == false}
   \\
    Default setting &
    {\tt f.set\_default(t)} &
    ~ &
    set default value to {\tt t} &
    {\tt ! f.defined(e)} implies{\tt f(e)} is equal to {\tt t}
    \par {\tt f.default()} is equal to {\tt t}
   \\
    Default access &
    {\tt t = f.default()} &
    default has been set by constructor {\tt f(g,t)} 
    or by  {\tt f.set\_default(t)} &
    set {\tt t} to  default value of {\tt f} &
    {\tt t} is equal to {\tt f.default()} 
   \\
    write access &
    {\tt t = f[e];} &
    {\tt f} is bound to a grid
    \par {\tt e.TheGrid() == f.TheGrid()} &
    create storage for a value\noteref{note-create-storage}
    if  {\tt ! f.defined(e)};
    \par assign  {\tt f(e)} to  {\tt t}. &
    {\tt f(e)} is equal to {\tt t}
   \T\\
   \T\hline
   \T\\
 \end{tabularx}

 \conceptsubsection{Invariants}
 \begin{tabular}{ll} 
   \hline
   Default value &
   If not  {\tt f.defined(e)} then  {\tt f(e)} is equal to {\tt f.default()}  
   \T\\ 
   \T\hline
 \end{tabular}

 \conceptsubsection{Complexity Guarantees}
 Default construction takes constant time.
 \\
 Construction from grid and construction with initalization both
 take constant time.
 \\
 The {\tt set\_default(t)} operation takes constant time.
 \\
 The access operations {\tt f(e), f[e]} and the test  {\tt f.defined(e)}
 take at most logarithmic time, or amortized constant time, or expected constant time.
 \\
 The memory requirements are at most proportional to the total number of different
 elements {\tt e} for which {\tt f[e]} has ever been called.
 
 \conceptsubsection{Models}
 \sectionlink{{\tt partial\_grid\_function<E,T>}}{partial-grid-function-hash}
 --- a generic implementation of partial grid functions by hash tables,
 defined in
 \gralfilelink{partial-grid-function-hash}{h}{base}.

 \conceptsubsection{Notes}

 \begin{enumerate}
 \item \notelabel{note-create-storage}
   This means that storage is also added if the access is meant read-only!
   Therefore, one should use the syntax {\tt t = f(e)} in this case.
 \end{enumerate}
 
 \conceptsubsection{See also}
 \sectionlink{\concept{ Grid Element Function }}{GridElementFunction} ~
 \sectionlink{\concept{ Grid  Function }}{GridFunction} ~
 \sectionlink{\concept{ Mutable Grid  Function }}{MutableGridFunction} ~
 \sectionlink{\concept{ Container Grid  Function }}{ContainerGridFunction} ~
 \sectionlink{\concept{ Total Grid Function }}{TotalGridFunction} ~
  

