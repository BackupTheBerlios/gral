\documentclass{article}
\usepackage{hyperlatex}


\newcommand{\GRALROOT}{GRALROOT}
\newcommand{\GRALINC}[2]{\GRALROOT/#2/doxygen/html/#1.html}
\newcommand{\STLURL}{STLURL}
\newcommand{\STLPATH}[1]{\STLURL/#1}
\newcommand{\BERTIURL}{http://www.math.tu-cottbus.de/\~{}berti}
\newcommand{\GRALURL}{\BERTIURL/gral}
\newcommand{\traitsURL}{}



\newcommand{\gralclasslink}[2]{\xlink{#1}{\GRALINC{classGrAL_1_1#1}{#2}}}
\newcommand{\gralmodulelink}[1]{\xlink{{\tt #1}}{\GRALINC{index}{#1}}}
\newcommand{\gralfilelink}[3]{\xlink{{\tt #1.#2}}{\GRALINC{#1\_8#2}{#3}}}

\title{Control Device: External Parameter Control}
\htmltitle{Control Device: External Parameter Control}
\author{Guntram Berti}
\htmladdress{\xlink{Guntram Berti}{mailto:berti@ccrl-nece.de}}
\htmlattributes{TABLE}{BORDER}
\htmldepth{0}
\htmldirectory{html_dir}




\begin{document}
 \section{Control Device: External Parameter Control}
 
 A very important question is the control of parameters
 in a program. Almost every nontrivial class contributing to the
 overall behavior of a program contains some parameters, that should
 be modifiable at startup or at runtime, e.g. the size of a data structure,
 filename for input and output etc. All these classes play the role of a
 {\tt  controlable }, that is, they have some parameters to be 
 controled by some {\tt  \gralclasslink{ControlDevice}{external-control}}
 which is the counterpart.
 A {\tt  ControlDevice } could be, for example,

 \begin{itemize}
 \item  a parameter file,
 \item  a GUI-dialog,
 \item  an interactive shell like tcl or
 \item  command-line parameters.
 \end{itemize}

A controlable could be for example a simple class like 
\begin{verbatim}
  class square : public controlable {
  private:
    double a, b; // sides
  public:
    square(ap = 0.0, bp = 0.0) : a(ap), b(bp) {}
    // need some method for pulling in parameters a, b from file

    // traditional solution:
    void read(istream& in);

    // 'controlable' solution: Let a ControlDevice know the details
    // and do the work
    virtual void register_at(ControlDevice&);
  };
\end{verbatim}

\subsection{The interfaces}
The basic interface  for {\tt \gralclasslink{controlable}{external-control}} looks like this:

\begin{verbatim}
class controlable {
public:
  virtual void register_at(ControlDevice&,const string& prefix) = 0;
  virtual void register_at(ControlDevice&);
  virtual void notify()  {} // may be used to signal a change
  virtual ~controlable() {}
};
\end{verbatim}
    
The ``other'' side ({\tt \gralclasslink{ControlDevice}{external-control}}) 
which abstracts from the paramater input mechanism
has the following outlook:

\begin{verbatim}
class ControlDevice  {
public:
  ControlDevice(control_device_impl* imp = 0) : impl(imp) {}
  void add(const string& nm, Mutator* value_ref);
  void update();
  string name() const;
  void print_values(ostream&) const;
  ControlDevice getSubDevice(const string& name);
private:
  control_device_impl* impl;
};
\end{verbatim}
    
    Now how does it work? A {\tt  controlable } will {\tt  register\_at(..) }
    a given  {\tt  ControlDevice } (e.g. via the method {\tt  add(..) } for
    basic parameters). This {\tt  ControlDevice } will then be able to change
    the parameters, for example upon a call to {\tt  update() }, which would
    in the case of a parameter-file cause the file to be read in.
    Where's the big deal of it? The advantage of this abstract approach is
    twofold: First, the code does not get littered with statements to read in
    parameters. Second, one gains flexibility in choosing the user interface:
    The implementation of the classes with parameters  does not have to change, 
    wether we have a GUI-interface, a simple parameter file or something 
    completely different.
 
    How does it work? Looking at the interface of {\tt  ControlDevice }, 
    you see the function 
\begin{verbatim}
  void add(const string& nm,Mutator* value_ref);
\end{verbatim}
   This is a name-value-pair ({\tt nm,value\_ref }). The name is used
    to reference the right variable from the outside (e.g. in a file or as
    label in a window), the value reference is used to change the underlying
    variable. The interface of class {\tt  \gralclasslink{Mutator}{external-control} } follows:
    
\begin{verbatim}
class Mutator {
public:
  virtual void read (istream& in)  = 0;
  virtual void print(ostream& out) const = 0;
  virtual void print(ostream& out, const string& name) const = 0;
  virtual string description() const;
  virtual ~Mutator() {}
};
\end{verbatim}

    The value is actually changed via reading from a stream, which could be a string
    typed in at a window. A simple implementation of this interface is supplied
    by the following template:

\begin{example}
template<class T>
class TypedMutator : public Mutator {
protected:
  T& v;
public:
  \gralclasslink{TypedMutator}{external-control}(T& vv) : v(vv) {}
  T value() {return v;}
  vir1tual void read(istream& in)   { in >> v;}
  virtual void print(ostream& out) const 
    { out << v;}
  virtual void print(ostream& out, const string& prefix = "") const 
    { out << prefix << v;}
};
\end{example}

There are other {\tt  Mutators}, see  the \gralclasslink{Mutator}{external-control} documentation. 
You can attach a (simple) Mutator
to  a variable like this:

\begin{example}
ControlDevice Ctrl(GetFileControlDevice("myparameters.in"));
...
double x;
    Ctrl.add("x",GetMutator(x));
    // RegisterAt(Ctrl,"x",x) is a synomym for the above.
\end{example}

The file {\tt  myparameters.in } could then contain the name-value-pair
{\tt  x 0.43 }, and on {\tt  Ctrl.update() } the variable {\tt  x } 
would be set
to {\tt  0.43} . This example also shows how to create a {\tt  ControlDevice }
More examples of how this works are found in 
\gralfilelink{ex1}{C}{external-control}
together with a sample input file.
It shows also how to use other types of Mutators and how
to build hierarchical (i.e. nested) name-value-structures.


 \end{document}
