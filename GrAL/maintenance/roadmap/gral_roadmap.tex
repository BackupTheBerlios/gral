\documentclass[dvips,color,12pt]{refart}
%\usepackage{doublespace}
%\usepackage{showkeys}
%\usepackage[T1]{polski}
%\usepackage[latin2]{inputenc}
\usepackage{hangcaption}
\usepackage{amsmath}
\usepackage{psfrag}
\usepackage{graphicx}
\usepackage{enumerate}
\usepackage{url}
\usepackage{cite}
\usepackage{color}

\newcommand{\Revision}{$Revision: 1.1 $}
% indentation 
\settextfraction{0.9}
%\setlength{\topmargin}{-2cm}
%\addtolength{\textheight}{4.0cm}

%%%%%%%%%%%%%%% FIGURE RULES
\DeclareGraphicsRule{.ps.gz}{eps}{.ps}{zcat #1}
\graphicspath{{figures/}}

%%%%%%%%%%%%%%% TITLE PAGE
%
\author{Roman Putanowicz\\
%Institute of Computer Methods \\
%in Civil Engineering (L5)\\
%Cracow University of Technology\\
$<$putanowr@twins.pk.edu.pl$>$\\
%Instytyt Metod Komputerowych w\\
%In�ynierii L�dowej (L5)\\
%Politechnika Krakowska
%$<$putanowr@twins.pk.edu.pl$>$\\
}
\title{GrAL development roadmap}
\date{\Revision\\%
\today}
%%%%%%%%%%%%%%%%%%%%%%%%

\begin{document}
\maketitle
\tableofcontents
\section{GrAL overview}
\emph{FIXME: some general GrAL intro}
\section{Future directions}
\emph{FIXME: Short summary of GrAL future development}
\section{GrAL core functionality}
  \subsection{Grid functions algebra and calculus}
  Module providing algebraic operations on grid functions (+ - * /)
  and module providing basic calculus operation (gradient, divergence, curl,
  etc.)
  \subsection{Support for curved geometries}

\section{GrAL plug-ins}
\subsection{Grid generators}
A valuable extension of GrAL library would be a module providing interface
to some grid generation tools, e.g. Triangle, NetGen, GRUMMP. 

\subsection{Storage formats}
One point where GrAL library can be visibly enhanced is wider support
for grid file formats for persistent grid storage. Various grid manipulation
libraries offer their own I/O capabilities and writing adapters to their
data structures enhances GrAL I/O capabilities. There are however some useful
storage formats which are not tightly coupled with any particular grid data
structures. Moreover, providing interface to a particular grid file format
can be often easier than providing a fully fledged grid data structure adapter.
The more file formats is supported by GrAL library, the bigger chances that 
more users will pick it as a basis for their projects.

Some formats to be supported:
\begin{itemize}
 \item HDF5
 \item XML based format proposed by Adaptive Software Project
 \item formats for FEM packages: ANSYS, FEAP
\end{itemize}

\subsubsection{Grid file format recognition tools}
One useful addition to GrAL library would be a facility to recognize file
formats. This facility would allow an automatic choice of a proper module for
I/O operations.

\subsubsection{Improvement of GrAL complex2d-format}
GrAL complex2d-format comes in two flavours: 0-bases and 1-based numbering.
The information about the numbering is external to a file and must be
given as a parameter to grid reading function. Using wrong file type can result
in strange behaviour or program crashes. Much improved interface can be
provided by encoding the format type in the file itself, for instance
in the first line of file as a kind of file meta-data:
\begin{verbatim}
# GrAL 0.2 0-based
12 15
.......
\end{verbatim}
\subsubsection{GrAL complex3d-format}
Provide classes OstreamComplex3D and IstreamComplex3D.
\subsubsection{Tutorial}
Much needed would be a tutorial explaining step by step programming classes
for I/O (especially archetype concept).
\subsection{Grid structure adapters}

Though GrAL library provides lot of functionality for grid manipulations,
it is rather unrealistic to expect it will provide coverage and expected
performance in all application areas. However, GrAL generic nature makes
it a perfect link between various specialized libraries.

Grid structure adapters are special classes which allow to transfer grid
data both direction between GrAL data structures and specialized data
structures of the particular library. Thanks to that, some highly specific
grid processing can be delegated to more efficient implementation,
while generic processing can be still done in GrAL.

Candidates for grid structure adapters:

\begin{itemize}
  \item VTK (Visualization Toolkit) grid classes
  \item OpenDX data structures
  \item GTS (GNU Triangulated Surface) data structures
  \item Boost Graph Library
  \item CGAL ?
\end{itemize}

\section{GrAL development tools}
This point concentrates on software engineering side of building GrAL.
\subsection{Templates for new modules}
Adding new modules can be made more automatic. A script can prepare suitable
directory structure, makefiles, documentation files, etc.
\subsection{Revision of current module management process}
The module management process need some smoothing. 
In current version it requires updating configuration files in each
module directory.
\subsection{Meta-programming based on GrAL concepts}
GrAL's grid data structures implementation is firmly and consistently
based on a hierarchy of abstract concepts. The concepts form an abstract grid
interface. The description of each concept includes detailed specification
of the methods and properties which must be provided in order to call an
implementation a model of that concept.

Combination of C++ templates and object oriented techniques minimizes the
effort of providing required interface, but there is still 
room for greater automation of that process.
There should be possible to provide tools (some sort of IDE compiler; IDE --
Interface Definition Language) which generates more or less elaborate code 
skeletons for new GrAL modules.

When programing a new grid data structure or an adapter to an existing one,
programmer will specify what concepts is the class a models of.
Then a "concepts compiler" will output code skeleton which will be subsequently
filled and adapted. This "meta-programming" would make easier writing 
traits classes, rudimentary documentation and would make the design decisions
more visible.

\section{Other language interfaces}
\emph{FIXME: section explaining the need for scripted interface to GrAL
functionality}
\subsection{Python}
\subsection{Java ??}
\section{GrAL satellite technology}
Success of GrAL will depend in large extent on devoted users base. To gain
popularity GrAL must be available not only to developers but somehow also 
to "ordinary" users facing gird manipulation problems. Some possible ideas of 
providing GrAL "to masses" are listed below.

\subsection{Unix-style tools}
One end of the spectrum of grid technology consumers consists of developers
building complex simulation system which handle large (hundred thousands and
more cells) distributed and heterogeneous meshes.
The other end consists
of researchers who occasionally want to perform seemingly simple mesh
manipulations like extracting mesh boundary, generating meshes in standard
geometries (boxes, spheres, L-shapes, etc), mapping mesh geometry through
simple mappings like rotations, scaling, or translating meshes between various
formats. We say "seemingly simple" because after a while the algorithms
involved in the above tasks get quite complicated, especially if one wants
to handle general cases and because there is lack of standard tools for such
manipulations. The second reason is our primary concern in this section.

One may wonder if GrAL does not fit the description of a standard tool for 
grid manipulations. Of course it does, but GrAL scope is somehow different. To
better illustrate this point let us consider two situations.
\begin{itemize}
\item The first one is when a researcher having a finite element mesh wants to 
extract mesh boundary in order to feed it into a boundary element method
program. What he also wants is to keep track between corresponding nodes and
elements in these two meshes. 
\item The second case is when a user wants to calculate average value of some
cell based grid function for the cells incident to a given node.
\end{itemize}
Knowing GrAL library these tasks are simple not to say trivial to program
but the users in question does not know GrAL (or for the fact any other
similar system) nor has he incentive or resources to learn it. What he wants,
instead, is to use a simple tool, most likely command line based so his job
can be automated if necessary. The usual solution to the above problems is
that a) either the needed functionality is found in an all-in-one program or
b) a local "grid toolsmith" prepares the needed tool or c) the
user programs the task in ad-hoc manner which often precludes reuse. 

The problem with all-in-one programs is that the more functionality they
encompass the less convenient they are, not mentioning the increasing
costs of maintaining them. Next, not all researchers work with skillful
programmers and most of the time the researchers are building their own tools.
However, instead of creating the tools from scratch, they should use convenient
building blocks as the starting point. 

There is an analogy between grid manipulations and text processing.
Text processing is ubiquitous in research practice and is probably the most
often performed activity. In some operating systems (most notably 
UNIX and Linux of various flavours) text processing lies at the core of their
design philosophy. Thus most of the information is stored in human readable 
text files. Also many high level protocols are text based. Now, instead of
providing specialized programs for each bit of different data, the users
are given very generic set of text processing tools ({\tt tr, cat, tac, merge,
paste, sort, uniq, grep, cut} and so on). What is important, the users are also
given a mechanism -- unix pipes -- for connecting tools. The tools go beyond
text processing capabilities found in most programming languages but are
also generic 
so they can be used in varying context. In authors' opinion this
processing paradigm has been unmatched by anything else yet.

The analogy between grid and text processing is such that for grids one can
provide set of generic tools too. The tools will be canned pieces of GrAL
functionality, relieving the user of the need to program in C++ with GrAL.
Being small and simple such tools are easier to use and to maintain. 
To complete the picture one need the way to interconnect the tools. 
This it the reason why the tools should be command line and text oriented
as shell and pipe mechanism is already in place. 

Here we touch another common point between grid and text processing. Grids are
most often saved in human readable text files and the standard text processing
tools are often used to program various grid manipulations (e.g. searching for
cell or node related data, splitting the grid, counting elements, etc.).
However, standard text processing tools work well only with content which is
organized in a linear fashion, and with algorithms sequential in nature. 
That is often not the case of grid manipulation and one has to resort to
programming in languages like C, C++, Fortran. 

If the nonlinear and complex algorithms are wrapped as self contained,
black box tools, then the whole potential of pipelined text processing can
be fully utilized for grid manipulations.
Let us illustrate the above concept with the already mentioned problem
of calculating grid function average over cells incident to a given node.
Suppose we have a grid file {\tt mesh.dat} with standard FEM incidence data 
(i.e. the file
lists nodes incident to each element). The format of this file is irrelevant
at the moment. The grid function is specified in file {\tt gf.dat}
in such way that each record contains element number and grid function value.
The node has number 3. With such setup the solution may look like:
\begin{tt}
\begin{verbatim}
  nodetocells mesh.dat 3 | sort -n | records gf.dat | average 2
\end{verbatim}
\end{tt}
The key component of the above pipeline is program {\tt nodetocells}. 
It takes mesh data file and node number as arguments and returns list
of elements incident to the node. The list is the sorted using standard
{\tt sort} utility. The {\tt records} selects specified records from file
{\tt gf.dat} and {\tt average} calculates the average of values from column 2.
Sorting tools like {\tt sort} are by default in all systems while 
{\tt records} and {\tt average} can be programed within minutes as shown 
in appendix \ref{app:tools}.
The only problematic bit is {\tt nodetocells} as it has to invert the
cell to nodes mapping stored in {\tt mesh.dat}. Though this is conceptually
simple, practically may be not if different grid data structures or
different file formats are considered. Moreover why should the wheel be
invented over and over again?

Because of its generic nature GrAL is perfect for building grid tools like
the one presented above. Such tools can be written using general grid 
abstractions provided by GrAL. The concreted data structures, file formats,
etc. will be hidden behind GrAL interface.
\subsubsection{To do}
\begin{enumerate}
  \item Recognize what kind of grid manipulations can be encapsulated as a
  separate command line tools. Possible groups of tools include:
  \begin{itemize}
    \item format translators -- translate one grid file format to another as
    closely as possible, 
    \item grid characteristic queries -- display information on characteristic
    grid properties -- number of nodes, cells, etc., 
    \item grid archetype queries -- display information on number and type of
    gird cell archetypes,
    \item grid geometry queries -- display information about various measures
    of grid geometric embedding (i.e. volume, surface area, etc.); geometric
    measures of each grid element; gird gravity center; grid inertia axes; 
    grid bounding box, 
    \item grid geometry transformations -- apply basic affine transformations
    (rotations, translations, scaling) to grid geometries as well as to single
    grid points, 
    \item grid incidence queries -- all sort of incidence queries
    \item miscellaneous -- grid splitting and gluing
  \end{itemize}
  \item Design the tools interface with particular attention paid to use of the
  tools as elements of text processing pipeline.
  \item Build a common implementation skeleton (the tools will share command
  line handling, error handling, etc.)
  \item Documentation. Besides all other formats there should be documentation
  in UNIX man page format (it is indispensible for this kind of tools).
  \item Test cases.
  \item Implementation.

  \subsubsection{Critical view and polemic}
  Possible criticism of the idea above may go along the following lines:
  \begin{itemize}
   \item {\it If the tools are basic, every one can implement them easily and
   according to his need once having access to GrAL library.}\\
   This approach would multiply cost and make harder the exchange of other
   utilities base on these tools.
   \item {\it The tools cover what would be equally well provided by wrapping
   GrAL in a scripted language}.\\
   Basically true, however command line approach
   is programming language neutral. And still then, the command line tools
   are a bit higher level than interface provided by scripted language mapping.
   On the other hand if a scripting language interface was ready such tools
   could be implemented in that language instead C++. 
   \item {\it Instead of command line there should be graphical user
   interface.}\\ Command line tools are cheaper to implement. Besides the whole
   point behind the command line tools is that they can be easily combined
   together. Achieving the same in graphical interface is much harder. 
   One possible solution along the GUI line would be to provide OpenDX modules
   specialized for grid processing and to use OpendDX graphical programming 
   capabilities.  This way however one confines himself to a single environment.
   \item {\it Command line is outlandish.} Not necessarily. There are examples
   of successful command line environments - the already mentioned t{\tt
   textutils} or closer to GrAL GMT (The Generic Mapping Tools)
   \url{http://gmt.soest.hawaii.edu/}.
   \item{The frequency of use of such tools will be low and their usefulness
   dubious. They may end up playing an ornamental role.}\\ Well, these tools are
   not a panacea for grid processing. The whole point in this exercise lies in
   providing suitably grained environment for grid processing, as illustrated
   on figure:
   \begin{center}
      \includegraphics{toolspyramid.eps}
   \end{center}
   Each level makes the picture complete and it is up to the user to chose the 
   level he wants to operate on.
   Command line tools can be very helpful when writing test scripts, in
   debugging, and in rapid prototyping.
  \end{itemize}
\end{enumerate}
\subsection{Lightweight web services base on GrAL}
This point is more related to possible, though indirect, use of GrAL than to its
development.

Among the users of grids technology supposedly a large group are those ones
who use grid generators, grid partitioners or grid format translators very
sparingly, just to prepare an input for their programs and then they forget
about the whole thing. Further, among them are users who work in environment
prepared for them, as well as those, who build the environment for themselves.
Whatever painless is (and often it is not) the installation and maintenance 
of grid manipulation software in comparison with infrequency of its use
renders it expensive. A possible cost cutting solution would be to use web
based services for accessing tools like grid generators, partitioners,
format translators or visualization.

The whole idea is not knew and it is at the heart of the Grid technology.
Here however, we talk about the really lightweight version of it (based 
just on http protocol na CGI mechanism). On client side on has basically 
everything what is needed that is web browser. One convenient addition could
be a set of http enabled command line tools for processing grids in batch
mode. 

On server side, in turn, we have users who write grid processing tools.
Very often they provide their codes as open source projects. However, also
often they do not have resources, knowledge or time to provide their programs
as services (very often we just talk about single person).

Now, assume that the whole process of establishing web service for a particular
tool can be made very simple (i.e. amounting to asking an administrator for
CGI privilege and then to follow step by step an instruction filling some
script templates). That should encourage people to publish their programs
as web services because in that case possible gains outweigh the costs.
The benefits include:
\begin{description}
  \item[publicity] -- It might be important to show people interest and project
  usefulness in order to get founding. The service framework may provide means
  for gathering usage statistics.
  \item[availability] -- If the program depends on specific hardware or
  software resources, it might be hard to publish it in a way other than a web
  service.
  \item[portability] -- Services are portable.
  \item[feedback] -- Service users are automatically program testers. The
  service framework can provide means for logging the user actions for
  debugging purpose.
\end{description}

Where is role a place for GrAL in the above? 
GrAL based tools can be used as grid file
translators, grid data structures translators, data validation tools. In
reverse, easier transfer or grid data thanks to web services will increase
demand for such grid tools.

Web services might be used by organizations working on GrAL to build users
community.
\subsection{GrAL + Corba for distributed grid processing}
The aim of ths point is to investigate the use of GrAL with Corba 
for building distributed grid processing programs. Justification for this
is somehow similar as for web based services, though this point will
serve a more illustratory purpose. It is aimed at users who know GrAL 
but do not know Corba and at users who know Corba but do not know GrAL
and shows how to use these two tools together.

\subsection{GrAL + MPI for parallel grid processing}
This point aims at providing better support for using GrAL with MPI for 
transferring grids and grid related data between different processors in
parallel programs. The idea is that calls to MPI services will be wrapped
as a higher level interface relieving the user of the need to deal with
common tasks or low level MPI internals. 
\begin{center}
\includegraphics{gps.eps}
\end{center}
An extension fo this project can be a generic interface to parallel services
regardless of it being provided  by MPI, PVM or OpenMP.
The interface would be parameterized by message passing mechanism giving
the users a lot more flexibility at constructing their applications.

\section{Users' support}

\subsection{Users documentation}

\subsection{Developers documentation}
\appendix
\end{document}
